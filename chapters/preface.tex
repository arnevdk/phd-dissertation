\chapter*{Preface}
\addcontentsline{toc}{chapter}{Preface}

The past four years have been a stimulating, instructive, and intense
journey of scientific and self-discovery. What started as a vague interest
in the applications of machine learning in the biomedical domain during
my master's studies rapidly evolved into a fascination with neural
signal processing, from the moment I came into contact with ith.
This was supercharged by the very high science-fiction factor
of Brain-Computer Interfaces.
There is just something about both the brain and computers, and
their beautiful complexity-from-simplicity, that I just can't
seem to let go of.

However, it is the potential impact on people's lives that first drew
me to the field. The biggest advantage my PhD project had to offer was
its holistic, end-to-end nature, involving everything from signal
processing and innovating with machine learning under the hood to EEG
experimentation and, eventually, working with end-user patients. I have
always considered this PhD, first and foremost, a learning experience—
one that has provided me, after a lot of trial and error, with the
toolkit to, hopefully, improve lives through advancements in biomedical
technology in the future.

For this, I am grateful to my thesis supervisors. The excellent and
complementary combination of the network, resources, advice, and
feedback you provided helped this project succeed. I would specifically
like to thank prof. Marc Van Hulle for the many opportunities you have
given me, prof. François Cabestaing for your close, active, and
involved supervision, and prof. Hakim Si-Mohammed for the many
sharp-witted ideas and contributions you brought to the table. I also
want to thank the members of the jury, prof. Andrea Kübler, prof.
Fabien Lotte, prof. Reinhold Scherer, prof. Maarten De Vos, and prof.
Adalberto Simeone, for the time and effort you have spent evaluating
this thesis and for your insightful feedback.

Furthermore, I would like to explicitly thank everyone else who helped
in realizing the work presented in this thesis. Juliette, your
proactive attitude and multidisciplinary perspective—and, of course,
your native command of the French language—were an immense help in
data collection. Without you, this thesis would probably be missing its
last chapter, and I hope you enjoy your well-deserved upcoming PhD
career. Working with participants with physical impairment would also
not have been possible without the medical advice of prof. Etienne
Allart from the Neurorehabilitation Unit at CHU Lille. I would also
like to thank the others involved in recruiting and interacting with
these participants: Alixe and dr. Edward Schietekatte at TRAINM Neuro
Rehab Clinics, prof. Kristl Claeys and prof. Philip Van Damme from the
Neuromuscular Reference Center at UZ Leuven, Marie and the team at
Fondation Partage et Vie, and all colleagues who helped me in data
collection and processing. Most importantly, I am grateful to everyone
who participated in our study, and I am sincerely moved by the positive
reception among those of you who could benefit from it. Of course, I
would also like to thank all students, colleagues, and friends who
submitted themselves to having their hair messed up in yet another EEG
experiment.

Since my thesis was a joint PhD between two labs, one at KU Leuven in
Belgium and one at the University of Lille in France, I got to
experience working at two different research institutions, which formed
an invaluable enrichment of my mini-career so far. This also means I've
had the amazing luck of having double the amount of wonderful
colleagues.

In Lille, I've very much appreciated working with Gaël. Your presence
never failed to lighten up the mood, and I wish you all the best luck
in finishing up your own PhD. Next, I want to thank Jimmy for showing
me the way through Lille and its campus life, and for all the tea. I
also enjoyed discussing with José, Jean-Marc, and Marie-Hélène. Finally,
I'm also glad that the SIGMA team was around to keep me in touch with
my computer science roots. Even though I was not always physically
present in Lille, I've always considered all of you as very good
colleagues, and I'm glad I've had the opportunity to revive my high
school French, which was in dire need of a refresher.

In Leuven, I want to begin with a special thanks to those of you who
guided me in the beginning of my PhD: Benjamin, for introducing me to
the wonderful world of ERPs and linear decoding, and Axel and Arno for
powering through COVID with me. As for the rest of you lot—Aline,
Anahit, Aurélie, Barbara, Bob, Bob, Elvira, Eva, Li Ang, Liuyin, Mani,
Mani, Mansoureh, Qiang, Tjaša, Valentina, and Yide—you've made my days
brighter in many ways, and I could always count on you for advice,
support, or help. To Dilara, Gijs, Lunky, Reniflal, and Zoë, master
students and interns that I was fortunate enough to supervise: thank
you for your dedication, and good luck with your future adventures. I
would also like to mention the great work of the students at NeuroTech
Leuven—you never fail to impress me with wonderful ideas and excellent
execution. A final shoutout to everyone in and around the
Neurophysiology lab: Alessia, Astrid, Christophe, Inez, Jonathan,
Kenneth, Noa, Saba, Sara, Sofie, Stijn, Wouter, and all others.

On a more personal note, I probably wouldn't have gotten here if it was
not for the support (and proofreading) I have \emph{always} experienced
from my family and friends.
Cato, Loes, Mama, and Papa, I think you've provided about the
best nest I could imagine to grow up in. To Aderik, Joran, Siebe: thank you for
carrying me throughout my university studies.
To Hanne, Ingrid, Jonah, Kiefer, Joshua, Mathias, Vik, and Yana, I have but one
thing to say: Snek is love, Snek \emph{is} life.

Some people say whoever is mentioned first on a paper is the most
important, since they're supposed to have done done all the work.
Without the last author on there, however, most scientific publications would
either never have materialized or go unnoticed. Therefore, I would like to
mention my partner Liesel here at the end.
Liesel, thank you for everything you have done
for me. This does not only include the last few months, taking care of
practical arrangements, proofreading, managing, and putting up with me as I was
focused -- maybe a bit too much -- on work, \ldots, but much, much more than
that.
I still cannot believe how lucky I am to have you by my side.

As you turn this page and start reading the first chapter, I will also
start a new chapter in my life. You, on the other hand, will now dive
into about 200 pages of algebra, convoluted procedures, and tables with
numbers. Enjoy!
