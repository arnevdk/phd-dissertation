\chapter*{R\'esum\'e}
Les individus ayant des handicaps moteurs sévères, tels que ceux atteints de
syndrome d'enfermement ou de dysfonctionnement oculomoteur, rencontrent des défis considérables
lors de l'utilisation d'interfaces cerveau-ordinateur (BCI)  traditionnelles.
Ces systèmes nécessitent que les utilisateurs dirigent leur regard vers des
cibles spécifiques, une tâche impraticable pour les personnes ayant un contrôle oculaire limité ou inexistant.
Ce travail vise \`a palier cette limitation par améliorer des méthodes de BCI
indépendantes du regard et la décodage de la attention visuo-spatiale (VSA) cachée, où les utilisateurs peuvent diriger leur attention vers une cible sans mouvements oculaires correspondants.
Une contribution clé de ce travail est la compensation pour la compensation de
la variabilité de latence des potentiels évoqués (ERP), laquelle impacte
la performance de décodage de VSA cachée.
En gérant cette variabilit\'e, la méthode proposée améliore la précision de la communication
sans nécessiter de contrôle du regard, rendant les BCI plus utilisables pour les personnes ayant des handicaps moteurs.

Au-delà de l'indépendance du regard, le travail fait également progresser les
techniques générales de décodage ERP en affinant la structure des estimateurs
linéaires et multilinéaires.
Ces méthodes améliorent la précision de classification dans une gamme de
conditions BCI, en particulier lorsque les données d'entraînement sont limitées.
L'introduction de régularisation structurée dans les modèles linéaires et
multilinéaires améliore l'interprétabilité des classificateurs et réduit le temps d'entraînement ainsi que la complexité computationnelle.
Cela permet un entraînement plus efficace et contribue à la generalisation, ce qui
permets le d\'evelopment des systèmes plus fiables qui s'adaptent à divers
contextes et besoins des utilisateurs.

Les méthodes proposées ont été validées lors d'expériences impliquant à la fois
des individus en bonne santé et sept individus avec des handicaps physiques sévères et un contrôle oculomoteur altéré.
Ces expériences ont démontré la robustesse des nouvelles méthodes de décodage, montrant que le système pouvait efficacement décoder l'attention cachée même lorsque l'observation directe de la cible  était impossible.

\paragraph{Mots-clés :}
interfaces neuronales directes,
\'electroenc\'ephalographie,
potentiels \'evoqu\'es,
attention visuo-spatiale cachée,
decodage (multi-)linéaire,
handicap\'es moteurs
