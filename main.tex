% ATTENTION: Don't build this, either build main_booklet.tex or
% main_manuscript.tex
%
% Preamble ====================================================================
\pdfminorversion=7
\usepackage{todonotes}
\usepackage{lipsum}
\usepackage{tabularx}
\usepackage{makecell}
\usepackage{enumitem}
\usepackage{hyperref}

% Bibliography
\usepackage[backend=biber]{biblatex}
\addbibresource{references.bib}
\addbibresource{publications.bib}

% Title
\title{\sffamily An EEG-based visual Brain-Computer Interface for gaze-free communication}
\author{Ir. Arne Van Den Kerchove}


\begin{document}

% Front matter ================================================================
\pagestyle{front}
\frontmatter

\chapter{Preface}
\lipsum[1]
\chapter{Abstract}
\lipsum[2-4]
\chapter{Samenvatting}
\lipsum[5-7]
\tableofcontents
\listoffigures
\listoftables



% Main matter =================================================================
\mainmatter
\pagestyle{main}


\part{Introduction \& Literature}

\chapter{Brain-Computer Interfaces}
\begin{refsection}
	\section{Rationale}
	\section{Overview \& state-of-the-art}
	different recording technologies and level of invasiveness + infographic
	active/reactive/passive + infographic
	\section{Visual Event-Related Potential based BCIs}
	\todo{Check provisionary and final doctoral plan for motivation}
	\printbibliography[heading=subbibliography]
\end{refsection}

\chapter{Linear ERP decoding}
\section{LCMV-beamforming}
proof equivalent to LDA, better results with balanced and correctly centered
covariance matrix
\section{Toeplitz-LDA}
\section{Other methods}

\chapter{Gaze-independent BCIs}
\section{Oculomotor deficits in Locked-in Syndrome}
\section{Gaze-independent visual BCIs}
\section{Benefits \& drawbacks of covert attention in BCI operation}
\section{Electrophysiological correlates of covert attention}
\section{Compensating for covert attention}

\chapter{Problem statement \& rationale}
\emph{Goal:} Enable communication for eye-motor impaired patients

\emph{Method:} Design a comfortable interface that allows them to maximally exploit
their residual gaze capabilities, by leveraging a non-invasive high-ITR
visuospatial ERP-based BCI and improving ERP decoding performance (in general
and specifically in gaze-independent settings).

\emph{Novelty:}


\part{Algorithms \& Decoders}
\chapter{Kronecker-structured discriminant analysis}


Part of the work in this chapter was published in~\cite{VanDenKerchove2022}.
\todo{Rename to LDA}

\section{Introduction}

Brain-computer interfaces (BCIs) establish a direct communication pathway
between the brain and an external device~\cite{Wolpaw2002}.
Severely disabled patients with impaired or absent communication capabilities
can benefit from BCIs to restore normal functioning~\cite{Naci2012,Chaudhary2016}.
BCIs can be implemented in multiple ways, using non-invasive recording
techniques such as electroencephalography (EEG)~\cite{Abiri2019},
magnetoencephalography (MEG)~\cite{Mellinger2007},
functional Near-Infrared Spectroscopy (fNIRS)~\cite{Hong2015}, and Optically Pumped
Magnetometers (OPM MEG)~\cite{Paek2020}, or semi-invasive and invasive methods such as
electrocorticography (ECoG)~\cite{Schalk2011} or microelectrode
arrays~\cite{Maynard1997} which require surgery to implant a recording device.
While invasive BCIs yield the highest information transfer
rate~\cite{Willett2021}, non-invasive BCIs are preferable for short-term use
since they are not susceptible to the risks that come with surgery.
Of the non-invasive options, EEG is the most cost-effective and practical as it
is not limited to the same controlled settings as MEG and OPM MEG.

Besides the recording method, BCIs differ in the communication paradigms used
for communication~\cite{Abiri2019}.
A popular class of BCI paradigms relies on the evocation
of Event-Related Potentials (ERPs) in the brain in response to visual, auditory, or tactile stimulation, given their low decoding cost and generally short
calibration time before usage~\cite{Gao2014, Kapgate2015}.
The study we report on focuses on the visual P300 oddball ERP in response to a
rare but attended visual stimulus.
The decoder detects whether this ERP is present to determine which stimulus
the user attended.
The P300 paradigm has been used extensively in BCI development and is easy to
set up~\cite{Farwell1988, Sellers2006, Barachant2014, Philip2020}.

There are multiple state-of-the-art P300 classification methods, like Support
Vector Machines (SVMs)~\cite{Tayeb2014}, deep
learning models~\cite{vavreka2020evaluation,Borra2020}, and Riemannian Geometry
classifiers~\cite{Barachant2014}.
While these models often return a high classification accuracy, there is a need
for lightweight models -- lightweight models lead to fast off-line analyses and can be transferred to consumer-grade hardware.
When moving towards plug-and-play solutions, BCI calibration sessions should be short and model training times low.
The spatiotemporal beamformer~\cite{VanVliet2015, Wittevrongel2016} belongs to
this class of ERP decoding models as it achieves state-of-the-art performance and is fast to train.
Earlier work has shown that it is
possible to apply the spatiotemporal beamformer to multiple time-locked visual
BCI paradigms, including the P300 oddball paradigm,
Steady-State Visually Evoked Potentials (SSVEP), code-modulated Visually Evoked Potentials (cVEP)~\cite{Wittevrongel2017a} and
motion-onset Visually Evoked Potentials (mVEP)~\cite{Libert2021}.

This work shows that the original spatiotemporal
beamformer~\cite{Wittevrongel2016} can fall short in performance when BCI
calibration data are restricted.
We also show that the spatiotemporal beamformer does not scale well for
higher spatial and temporal resolution cases.
As a response to these issues, we introduce a regularization method that
exploits prior knowledge about the spatiotemporal nature of the EEG signal to
improve the accuracy for low data availability settings and speed up the
classifier training time, thereby considerably reducing memory usage.
Similar structured regularization approaches have been applied to other linear
ERP classifiers~\cite{gonzalez2017spatio, Vliet2020} and have shown
significant increases in performance.
Additionally, we show that regularization results in an interpretable
classification model, which can aid in analyzing and developing spatiotemporal beamformer-based classifiers.

\section{Materials \& methods}
\subsection{Notation}
We represent matrices with cursive capital letters, vectors with bold
lowercase letters, and scalars with cursive lowercase letters.
The epoched EEG data with $n$ epochs, $c$ channels, and $s$ samples are
represented in epoch format as $\{X_i\in\mathbb{R}^{c\times s}\}^n_{i=1}$ or flattened vector format by concatenating all channels for each epoch.
Flattening results in $\{\mathbf{x}_i\in\mathbb{R}^{cs}\}^n_{i=1}$ such that $\mathbf{x}_i = \text{vec}(X_i)$.
The real covariance matrix of the epochs in vector format is
denoted by $C$, estimators thereof as $\hat{C}$.

\subsection{Spatiotemporal beamforming}
LCMV-beamforming was initially introduced to EEG analysis as a filter for
source localization~\cite{VanVeen1997} to enhance the signal-to-noise ratio
(SNR).
Van Vliet et al.~\cite{VanVliet2015} first applied the spatiotemporal
LCMV-beamformer as a method for the analysis of ERPs.
The extension to the combined spatiotemporal domain~\cite{VanVliet2015} and the
data-driven approaches proposed by Treder et al.~\cite{TREDER2016279} and
Wittevrongel et al.~\cite{Wittevrongel2016} allow for its application to classification problems.

For the following analyses, we assume that all EEG channels are normalized with zero mean and unit variance without loss of generality.
Solving \autoref{eq:minimum_variance} under the linear constraint given by
\autoref{eq:linear_constraint} returns the filter weights $w$ defining the spatiotemporal LCMV-beamformer.
	\begin{equation}
		\underset{\mathbf{w}}{\arg\min}\mathbf{w}^\intercal C
		\mathbf{w}^\intercal
		\label{eq:minimum_variance}
	\end{equation}
	\begin{equation}
		\mathbf{a}^\intercal\mathbf{w} = 1
		\label{eq:linear_constraint}
	\end{equation}

These weights minimize the variance of the output of the filter while enhancing
the signal characterized by the constraint.
$\mathbf{a} = \text{vec}(A)$ is the data-driven activation pattern, a template
of the signal of interest maximizing the difference between two classes of
epochs, determined as follows:

	\begin{equation}
		\mathbf{a} =
		\frac{1}{|\text{class 1}|}\sum_\text{class 1}\mathbf{x_i} -
		\frac{1}{|\text{class 2}|}\sum_\text{class 2}\mathbf{x_i}
		\label{eq:activation_pattern}
	\end{equation}

The method of Lagrange multipliers then gives the closed-form solution to the minimization problem posed by
\autoref{eq:minimum_variance} and \autoref{eq:linear_constraint} as:

	\begin{equation}
		\mathbf{w} =
		\frac{C^{-1}\mathbf{a}^\intercal}
		{\mathbf{a}C^{-1}\mathbf{a}^\intercal}
		\label{eq:closed_form}
	\end{equation}



The beamformer can be applied to epochs (unseen or not) as:

	\begin{equation}
		y_i = \mathbf{w}\mathbf{x}_i
		\label{eq:apply_beamformer}
	\end{equation}

resulting in a scalar output per epoch.
The linear constraint in \autoref{eq:linear_constraint} ensures that the
beamformer maps epochs containing a target response to a score close to one
and, conversely, epochs not containing a target response to a score close to
zero.

\subsection{Covariance matrix regularization}
While the spatiotemporal beamformer, in theory, achieves optimal separation
between target and non-target classes, in analogy to linear discriminant analysis~\cite{TREDER2016279}, it does not always perform well on unseen data.
The main challenge is to find a good estimator for the inverse covariance matrix $C^{-1}$ since the real underlying covariance matrix generating the data is, in principle, unknown.

\subsubsection{Empirical covariance estimation}
\label{sec:empirical_covariance}
Earlier spatiotemporal beamformer studies~\cite{Wittevrongel2016,
Wittevrongel2016a, Wittevrongel2017, Wittevrongel2017a} use the empirical
covariance and inverse covariance calculated as follows:

	\begin{equation}
		\hat{C}_\text{emp} =
		\frac{1}{n-1}\sum^{n}_{i=1}\mathbf{x}_i\mathbf{x}_i^\intercal
		\label{eq:empirical_covariance}
	\end{equation}


	\begin{equation}
		\widehat{C^{-1}}_\text{emp} = \hat{C}_\text{emp}^+
		\label{eq:empirical_inverse_covariance}
	\end{equation}

The Moore-Penrose pseudoinverse $^+$, ensures a solution exists when $\hat{C}_\text{emp}$ is singular.
\autoref{fig:covs}a and \autoref{fig:covs}b respectively show examples of the
empirical estimators of the covariance and the inverse covariance matrices.
The empirical estimator suffers from performance and
stability issues if the number of epochs $n$ used or estimation is not much larger than the number of features $cs$~\cite{Stein1956,Khatri1987}.

\subsubsection{Shrunk covariance estimation}
\label{sec:shrunk_covariance}
The shrinkage covariance estimator creates a better-conditioned inversion matrix problem and generally performs better when applied to unseen data.
The estimators for the covariance and inverse covariance are given by:

	\begin{equation}
		\hat{C}_\alpha =
		(1-\alpha) \hat{C}_\text{emp}
		+ \alpha\frac{\text{Tr}(\hat{C}_\text{emp})}{cs}\mathbb{I}
		\label{eq:shrinkage}
	\end{equation}


	\begin{equation}
		\widehat{C^{-1}}_\alpha =
		\hat{C}^+_\alpha
		\label{eq:shrinkage_inverse}
	\end{equation}

with $0<\alpha<1$.
Analogous to L2 regularization of the beamforming problem,
shrinkage reduces the ratio between the smallest and largest eigenvalues
of the covariance matrix by strengthening the diagonal.
\autoref{fig:covs}c and \autoref{fig:covs}d respectively show examples of the
shrunk estimator of the covariance and the inverse covariance matrices.

Earlier work~\cite{Libert2021} applied shrinkage regularization to ERP
decoding with the spatiotemporal beamformer and showed competitive performance
compared to other state-of-the-art decoding techniques like stepwise LDA or SVM.
The abovementioned work chooses the shrinkage coefficient $\alpha$ as a fixed hyperparameter.
However, its optimal value depends on the number of training epochs, the
covariance matrix's dimensionality, and the independence and variance of the
data, which can vary across evaluation settings and per session.
The optimal value for $\alpha$ can be found with a line search using cross-validation, but this can be a costly procedure.
Methods exist to estimate an optimal shrinkage value from the data directly.
Most notable among these are the Ledoit-Wolf procedure~\cite{LEDOIT2004365},
Rao-Blackwell Ledoit-Wolf~\cite{chen2010shrinkage}, and Oracle Approximating Shrinkage~\cite{chen2010shrinkage}.
A more recent estimation method~\cite{Tong2018} emulates a leave-one-out
cross-validation (LOOCV) scheme expressed by the data-driven closed-form
estimate:

	\begin{equation}
		\alpha =
		1-\frac{
			\frac{n}{n-1}\text{Tr}(\hat{C}_\text{emp}^2)
			- \frac{2}{cs}\left[\text{Tr}(\hat{C}_\text{emp})\right]^2
			+ \frac{1}{cs}\text{Tr}(\hat{C}_\text{emp}^2)
			- \frac{1}{n(n-1)}\sum_{i=1}^n||\mathbf{x}_i||_2^4
		}
		{
			\frac{n^2 -2n}{(n-1)^2}\text{Tr}(\hat{C}_\text{emp}^2)
			- \frac{2}{cs}\left[\text{Tr}(\hat{C}_\text{emp})\right]^2
			+ \frac{1}{cs}\text{Tr}(\hat{C}_\text{emp}^2)
			+ \frac{1}{n(n-1)^2}\sum_{i=1}^n||\mathbf{x}_i||_2^4
		}
		\label{eq:loocv}
	\end{equation}

We opt for the LOOCV shrinkage estimator because it avoids some of the
assumptions made by~\cite{LEDOIT2004365} and~\cite{chen2010shrinkage} and
because it generalizes to structured covariance estimation as described in
\autoref{seq:structured_estimation}.

\subsubsection{Spatiotemporal beamforming with Kronecker-Toeplitz structured covariance}
\label{seq:structured_estimation}
Exploiting prior knowledge about the spatiotemporal structure of the EEG signal leads to a more regularized estimator of the covariance.
When viewing the example of empirical spatiotemporal EEG covariance in
\autoref{fig:covs}a, it becomes clear that this matrix consists of a block pattern of repeated, similar matrices.
Due to the multi-channel nature of the signal, we assume that the covariance of spatiotemporal EEG epochs is a Kronecker
product of two smaller
matrices~\cite{Munck1992,DeMunck1999,huizenga2002spatiotemporal}, as expressed
by:

	\begin{equation}
		\hat{C}_\text{struct} = \hat{S} \otimes \hat{T}
		\label{eq:kronecker}
	\end{equation}

with $\otimes$ the Kronecker product operator.
$\hat{S} \in \mathbb{R}^{c\times c}$ and $\hat{T} \in \mathbb{R}^{s\times s}$ respectively correspond to estimators of the spatial and temporal covariance of the data.
Furthermore, because the temporal covariance of the EEG-signal is
stationary (i.e., it is only dependent on interval length between covarying
time samples)~\cite{Bijma2003}, it is assumed to have a Toeplitz-matrix structure:

	\begin{equation}
		\hat{T}_{i,j} = \hat{T}_{i+1,j+1}
		\label{eq:toeplitz}
	\end{equation}

\autoref{prop:inverse_kronecker} then leads to
\autoref{eq:cov_inverse_kronecker} to estimate the inverse
covariance.
\begin{property}
	$(U \otimes V)^+ = U^+ \otimes V^+$ for any non-singular
	matrices $U$ and $V$~\cite{Langville2004}.
	\label{prop:inverse_kronecker}
\end{property}

	\begin{equation}
		\widehat{C^{-1}}_\text{struct} = \hat{S}^+ \otimes \hat{T}^+
	\label{eq:cov_inverse_kronecker}
	\end{equation}

Finally, based on \autoref{prop:kronecker_multiplication},
\autoref{eq:closed_form} can be reformulated more efficiently as
\autoref{eq:closed_form_kronecker}.
\begin{property}
	$(U\otimes V)\cdot\text{vec}(W) = \text{vec}(VWU^\intercal)$
	for any matrices $U\in\mathbb{R}^{p\times p}$,
	$V\in\mathbb{R}^{q\times q}$ and $W\in\mathbb{R}^{p\times q}$~\cite{Loan2000}.
	\label{prop:kronecker_multiplication}
\end{property}

	\begin{equation}
		\hat{\mathbf{w}}_\text{struct} =
		\frac{\hat{S}^+A^T\hat{T}^+}
		{\mathbf{a}\cdot\text{vec}(\hat{S}^+A^T\hat{T}^+)}
		\label{eq:closed_form_kronecker}
	\end{equation}

Using \autoref{eq:closed_form_kronecker} removes the need to calculate the full, high dimensional Kronecker product $\hat{S}^+\otimes
\hat{T}^+$.
\autoref{fig:covs}e and \autoref{fig:covs}f respectively show examples of the
structured covariance and inverse covariance estimators,
consisting of a spatial Kronecker factor (\autoref{fig:covs}g and
\autoref{fig:covs}h) and a temporal component (\autoref{fig:covs}i and
\autoref{fig:covs}j).

\begin{figure}
	\begin{adjustwidth}{-\extralength}{0cm}
		\centering
	\includegraphics[width=\linewidth]{figures/covs.eps}
\end{adjustwidth}
	\caption{Different estimators of the covariance and inverse covariance
		of 100 epochs of data from \textit{Subject 01} for channels
\textit{Fz}, \textit{Cz}, \textit{Pz}, and \textit{Oz} and time samples between 0.1s and 0.6s.
Regularized estimators of the inverse covariance exhibit less extreme values and have a sparser structure.
(\textbf{a,b}) Empirical covariance and inverse covariance matrices.
(\textbf{c,d}) Shrunk covariance and inverse covariance matrices with $\alpha=0.14$ as
determined by the closed-form LOOCV method. (\textbf{e,f}) Kronecker-Toeplitz
structured covariance and inverse covariance matrices.
(\textbf{g,h}) Spatial Kronecker factor of the Kronecker-Toeplitz structured shrunk estimator and its inverse.
(\textbf{i,j}) Temporal Kronecker factor of the Kronecker-Toeplitz structured shrunk estimator and its inverse.}
	\label{fig:covs}
\end{figure}
\unskip

The Kronecker approach has shown significant performance yields in different linear spatiotemporal EEG and MEG
applications~\cite{de2002estimating,huizenga2002spatiotemporal,6408231,gonzalez2016kronecker,gonzalez2017spatio}.
Van Vliet \& Salmelin~\cite{Vliet2020} have applied a Kronecker-structured covariance estimator to ERP classification with linear models in a post-hoc fashion.
Our work goes further by embedding the Kronecker structure in the
spatiotemporal beamformer training process, using a data-adaptive shrinkage
method, and regularizing the covariance further by imposing a Toeplitz
structure on the temporal covariance.

\subsubsection{Kronecker-Toeplitz structured covariance estimation}
\label{sec:structured_covariance}
The question remains how to estimate $\hat{S}$ and $\hat{T}$.
While the Flip-Flop and Non-iterative Flip-Flop
algorithms~\cite{Lu2005, werner2008estimation, wirfalt2010toeplitz} can estimate Kronecker or Kronecker-Toeplitz structured covariances, new results show that a fixed point iteration is more efficient~\cite{Wiesel2012a,Wiesel2012}.
After each iteration, the spatial and temporal covariances matrices are scaled to unit
variance to ensure the fixed point iteration converges.
Finally, shrinkage can also be introduced in the Fixed Point Iteration to
improve stability and achieve more robust
regularization~\cite{Wiesel2012,Greenewald2014,6408231, Breloy2016}.
The spatial and temporal covariance matrices are shrunk at every fixed-point
iteration with shrinkage factors $\beta_k$ and $\gamma_k$ before matrix
inversion in the
next iteration.
Combined, this leads to the iterative estimation algorithm described by the
following equations:

	\begin{subequations}
	\begin{equation}
		\tilde{S}_{k+1} =
		\frac{1}{n}
		\sum^n_{i=1}X_i^\intercal\hat{T}_k^+X_i
		\label{eq:fpi_spatial}
	\end{equation}
	\begin{equation}
		\tilde{T}_{k+1} =
		\frac{1}{n}
		\sum^n_{i=1}X_i\hat{S}_k^+X_i^\intercal
		\label{eq:fpi_temporal}
	\end{equation}
\end{subequations}



	\begin{subequations}
	\begin{equation}
	\tilde{S}_{k+1}^{(\beta)} =
	(1-\beta_{k+1})\tilde{S}_{k+1}
	+\beta_{k+1}\frac{\text{Tr}(\tilde{S}_{k+1})}{c}\mathbb{I}
	\label{eq:fpi_spatial_shrunk}
	\end{equation}
	\begin{equation}
	\tilde{T}_{k+1}^{(\gamma)} =
	(1-\gamma_{k+1})\tilde{T}_{k+1}
	+\gamma_{k+1}\frac{\text{Tr}(\tilde{T}_{k+1})}{s}\mathbb{I}
	\label{eq:fpi_temporal_shrunk}
	\end{equation}
	\end{subequations}



	\begin{subequations}
	\begin{equation}
		\hat{S}_{k+1} =
		\frac{c}{\text{Tr}\left[\tilde{S}_{k+1}^{(\beta)}\right]}
		\tilde{S}_{k+1}^{(\beta)}
		\label{eq:fpi_spatial_norm}
	\end{equation}
	\begin{equation}
		\hat{T}_{k+1} =
		\frac{s}{\text{Tr}\left[\tilde{T}_{k+1}^{(\gamma)}\right]}
		\tilde{T}_{k+1}^{(\gamma)}
		\label{eq:fpi_temporal_norm}
	\end{equation}
\end{subequations}

$\hat{S}_0$ and $\hat{T}_0$ can be initialized to any positive definite matrix.
We choose to use the identity matrices $\mathbb{I}^{c\times c}$ and $\mathbb{I}^{s\times s}$.
After each iteration, all diagonals of $\hat{R}_{k+1}$ are set to their mean
values to ensure that $\hat{R}_{k+1}$ and $\hat{T}_{k+1}$ are Toeplitz structured.

Xie et al.~\cite{xie2021regularized} show that the LOOCV estimates for the
optimal values of $\beta_{k+1}$ and $\gamma_{k+1}$ also yield a closed-form
solution for the Kronecker fixed-point-iteration algorithm:

	\begin{subequations}
	\begin{equation}
		\beta_{k+1} =
		1-
		\frac{
			\frac{n}{n-1}\text{Tr}(\tilde{S}_{k+1}^2)
			- \frac{2}{c}\left[\text{Tr}(\tilde{S}_{k+1})\right]^2
			+ \frac{1}{c}\text{Tr}(\tilde{S}_{k+1}^2)
			- \frac{1}{n(n-1)}\sum_{i=1}^n
			\left[\text{Tr}(X_i\hat{T}_k^+X_i^\intercal)^2\right]
		}{
			\frac{n^2-2n}{(n-1)^2}\text{Tr}(\tilde{S}_{k+1}^2)
			- \frac{2}{c}\left[\text{Tr}(\tilde{S}_{k+1})\right]^2
			+ \frac{1}{c}\text{Tr}(\tilde{S}_{k+1}^2)
			+ \frac{1}{n(n-1)^2}\sum_{i=1}^n
			\left[\text{Tr}(X_i\hat{T}_k^+X_i^\intercal)^2\right]
		}
		\label{eq:spatial_shrinkage}
	\end{equation}
	\begin{equation}
		\gamma_{k+1} =
		1-
		\frac{
			\frac{n}{n-1}\text{Tr}(\tilde{T}_{k+1}^2)
			- \frac{2}{s}\left[\text{Tr}(\tilde{T}_{k+1})\right]^2
			+ \frac{1}{s}\text{Tr}(\tilde{T}_{k+1}^2)
			- \frac{1}{n(n-1)}\sum_{i=1}^n
			\left[\text{Tr}(X_i^\intercal\hat{S}_k^+X_i)^2\right]
		}{
			\frac{n^2-2n}{(n-1)^2}\text{Tr}(\tilde{T}_{k+1}^2)
			- \frac{2}{s}\left[\text{Tr}(\tilde{T}_{k+1})\right]^2
			+ \frac{1}{s}\text{Tr}(\tilde{T}_{k+1}^2)
			+ \frac{1}{n(n-1)^2}\sum_{i=1}^n
			\left[\text{Tr}(X_i^\intercal\hat{S}_k^+X_i)^2\right]
		}
		\label{eq:temporal_shrinkage}
	\end{equation}
\end{subequations}

The shrinkage parameters $0<\beta_{k+1}<1$ and $0<\gamma_{k+1}<1$ should be
re-determined after each iteration.
The Oracle Approximation Shrinkage method can also be used to determine
$\beta_{k+1}$ and $\gamma_{k+1}$~\cite{chen2010,xie2021regularized} but performs worse for spatiotemporal EEG data since not all assumptions are met.

\subsection{Dataset}
We use the dataset from~\cite{Wittevrongel2016}, containing P300 oddball EEG
recordings of 21 healthy subjects since it is a high-quality dataset with a high
number (32) of electrodes and concurrently recorded EOG responses for ocular artifact rejection.
Nine targets were arranged on a monitor before the subject during an
experimental session.
The subject was asked to pay attention to a cued target for a block
of stimulations.
The stimulations in a block are organized in 15 separate subsequent trials.
A trial is defined as 9 stimulations in which each target is flashed
precisely once per trial.
Each target was cued four times, resulting in a dataset consisting of 36 blocks
(4860 stimulations) per subject.
Each stimulation will correspond to a single epoch in the preprocessed dataset.
See~\cite{Wittevrongel2016} for a complete description of the dataset and the recording procedure.

\subsection{Software and preprocessing}
Data processing and classifier analysis were performed in Python using
Scikit-Learn (version 1.0.1)~\cite{pedregosa2011scikit} and SciPy (version
1.7.1)~\cite{2020SciPy-NMeth}.
The preprocessing pipeline was implemented using the MNE-Python toolbox
(version 0.24.0)~\cite{GramfortEtAl2013a}.
The dataset was converted to BIDS-EEG format~\cite{Pernet2019} and managed and
loaded with MNE-BIDS (version 0.9)~\cite{Appelhoff2019}.
The Riemannian classifier from \autoref{sec:riemannian} was implemented using
pyRiemann (version 0.2.7).
Statistical tests were performed in R (version 4.1.2).

The EEG recorded at 2048Hz was re-referenced off-line to the average of the mastoids.
The reference electrodes were dropped from the analysis.
Data were subsequently filtered between 0.5Hz and 16Hz using forward-backward
filtering with a fourth-order Butterworth IIR filter.
The EEG signal was corrected for ocular artifacts using Independent Component
Analysis (ICA) by rejecting components that correlated with the bipolar EOG channels vEOG and hEOG, according to adaptive Z-score thresholding.
Finally, epochs were cut from 0.1s to 0.6s after stimulus onset.
No baseline correction was performed since this affects the temporal covariance
of the data, violating the Toeplitz structure assumption~\cite{Bijma2003}.

\subsection{Classification}
\subsubsection{Cross-validation scheme per subject}
We use a variation of grouped fold cross-validation per subject to evaluate the classifiers.
We apply 4-fold cross-validation by splitting the blocks of each subject into
four continuous folds.
Unlike regular cross-validation, we only use a single fold to train the
classifiers while using the other three folds for validation.
This scheme results in a training set of 9 blocks of 135 epochs each.
We chose this approach since we are primarily interested in the performance of the classifiers in the case of low data availability.
The classification task is to determine the cued target for each block.
The fraction of correctly predicted cues provides the accuracy of a classifier.
Data from all trials are used in the training fold, while classifier validation
is performed multiple times per fold, each time using an increasing amount of
trials (i.e., using the first trial, using the first two trials, etc. until all 15 trials
are used).
For each of the 9 stimulated targets, the averages over the corresponding epochs across
the utilized trials are used to predict the cued target in that block.
The target with the maximal classifier score was then chosen as the predicted
cued target.
Before training the classifiers, a Z-score normalization transformation was
developed on the training data to scale all EEG channels to unit variance.
This transformation was then applied to the validation data.

\subsubsection{Spatiotemporal beamformer classifier}
Before calculating the spatiotemporal beamformer (STBF), the signal was downsampled to
32Hz or twice the low-pass frequency 16Hz, resulting in 17 time samples
between 0.1 and 0.6s. According to the Nyquist Theorem, more samples would not
contain more information hence the minimum temporal resolution is chosen to reduce
the dimensionality of the covariance and improve its covariance.
The activation pattern is the difference between the averages of epochs in response to cued targets and the averages of those in response to non-cued targets.
We constructed three variations of the spatiotemporal beamformer:
STBF with empirical covariance estimation (\textsc{stbf-emp}) as in
\autoref{sec:empirical_covariance}, STBF with
LOOCV-shrunk covariance estimation (\textsc{stbf-shrunk}) as in
\autoref{sec:shrunk_covariance}, and STBF with
Kronecker-Toeplitz structured covariance estimation (\textsc{stbf-struct} )with LOOCV shrinkage for
the Kronecker factors as in \autoref{sec:structured_covariance}.

\subsubsection{Riemannian geometry classifier}
\label{sec:riemannian}
We opted for a Riemannian
geometry-based classifier to compare our results.
The Riemannian model (x\textsc{dawn+rg}) uses the xDAWN spatial filter combined
with Riemannian geometry in tangent space as implemented by Barachant et
al.~\cite{barachant2014meg}.
This classifier uses four xDAWN spatial filters and each epoch's empirical spatial covariance matrix.
The target with the maximum score is the prediction of the cued target.
x\textsc{dawn+rg} was trained and validated without downsampling using epochs
at the original sample rate of 2048Hz.

\todo{check chapters/structure paper}
\todo{check structure Leuven.ai poster}
\section{The spatiotemporal EEG covariance}
\section{The single-KP covariance model}
\section{The sum-of-KP covariance model}
\section{Decoding performance}
\section{Discussion}
\subsection{Extension into the space-time-frequency domain}


\chapter{Tensor discriminant analysis}
\section{The tensor structure of EEG data}
\section{Higher-order Discriminant Analysis}
\section{Block-term Tensor Discriminant Analysis}
\section{Decoding performance}
\section{Discussion}
%\section{Tensorizations coping with latency jitter}

\chapter{Woody Classifier-based Latency Estimation}
\section{The role of jitter in gaze-independent ERP decoding}
\section{Latency estimation and correction}
\todo{check earlier draft alignment paper}
\todo{check deleted paragraphs in jitter paper}
\section{Classifier-based Latency Estimation}
\section{Classifier-based Latency Estimation with Woody iterations}
\section{Simulation study}
\section{Discussion}

\part{Experiments \& Applications}

\chapter{Interface design \& experimental setup}
\section{The Hex-o-Spell interface}
\section{Visuospatial attention conditions}

\chapter{Compensating jitter for gaze-independent decoding}
\section{Data collection \& preprocessing}
\section{Results}
\section{Discussion}
\todo{check chapters paper}

\chapter{Patient cases}
\section{Patient presentation}
\section{Data collection \& preprocessing}
\todo{Define structure patient results/discussion}
\section{Outcomes}
\section{Discussion}

\chapter{Integrating gaze-tracking data}

\chapter{Conclusions \& recommendations}
\section{General discussion}
\section{Limitations}
\section{Future directions}
\section{Working with patients}
\section{Conclusion}



\appendix
\backmatter
\chapter{Publications}
\chapter{Funding \& acknowledgments}
\lipsum[1]
\chapter{Personal contribution}
\lipsum[2]
\chapter{Conflict of interest}
The authors declare no conflict of interest.
% Structure

\newenvironment{cvlist}{
	\begin{description}[font=\normalfont]\setlength\itemsep{0em}}
		{\end{description}}

\newcommand{\cvitem}[2]{\item[#1,] #2}

%Preamble


%\begin{minipage}[b]{.7\linewidth}
%	\Large \textbf{Arne Van Den Kerchove}\\
%	\normalsize Doctoral Researcher Computational Neuroscience\\
%	KU Leuven, University of Lille
%  \bigskip
%
%  \texttt{arne.vandenkerchove@kuleuven.be \\
%  arne.vandenkerchove@univ-lille.fr \\
%  arne@vandenkerchove.com \\
%  +32 473 32 78 71 \\
%	\url{https://arne.vandenkerchove.com}\\
%	\url{https://orcid.org/0000-0002-9367-2986} \\
%  \url{https://linkedin.com/in/arnevdk} \\}
%
%%	\begin{tabular}{@{}c l}
%%		\faAt        & arne.vandenkerchove@kuleuven.be             \\
%%		\faAt        & arne.vandenkerchove@univ-lille.fr           \\
%%		\faAt        & arne@vandenkerchove.com                     \\
%%		\faPhone     & +32 473 32 78 71                            \\
%%		\faMapMarker & Leuven, Belgium                             \\
%%		\faMapMarker & Laboratory for Neuro- and Psychophysiology  \\
%%		             & KU Leuven-Campus Gasthuisberg             \\
%%		             & ON II-Herestraat 49-box 1021            \\
%%		             & BE-3000 Leuven                              \\
%%		\faMapMarker & UMR CRIStAL                                 \\
%%		             & Université de Lille-Campus scientifique   \\
%%		             & Bâtiment ESPRIT-Avenue Henri Poincaré     \\
%%		             & FR-59655 Villeneuve d'Ascq                  \\
%%		\faGlobe     & \url{https://arne.vandenkerchove.com}       \\
%%		\aiOrcid     & \url{https://orcid.org/0000-0002-9367-2986} \\
%%		\faLinkedin  & \url{https://linkedin.com/in/arnevdk}
%%	\end{tabular}
%\end{minipage}%
%\begin{minipage}[b]{.29\linewidth}
%	\includegraphics[width=\linewidth]{photo.jpg}
%\end{minipage}

% Curriculum vitae =============================================================
\chapter{Curriculum vitae}
\section*{About me}

I am a Ph.D. candidate\todo{change} at KU Leuven Laboratory for Neuro- and Psychophysiology
and the University of Lille Research Center for Informatics, Signal Processing,
and Control Science. My research is centered on developing brain-computer
interface (BCI) technologies with a primary focus on assisting individuals with
severe disabilities, especially those with motor and eye movement impairments.
My goal is to enhance their quality of life by creating accurate and
user-friendly BCIs. Combining my expertise in machine learning, signal
processing, and user interface design with insights from neurophysiology, my
PhD centers around the development of an event-related potential BCI system
using EEG for real-time communication. I strongly believe in the potential of
innovative neurotechnology to empower individuals and patients, enabling them
to lead more independent and fulfilling lives.

\paragraph{Keywords:} brain-computer interfaces, event-related
potentials, multilinear classification, covert visual attention

\subsection*{Education}

\begin{cvlist}
	\cvitem{Ph.D.\ in Biomedical Sciences}{KU Leuven, 2024}
	\cvitem{Ph.D.\ in Control Science and Signal Processing}{University of
		Lille, 2024}
	\cvitem{M.Sc.\ in Engineering Science: Computer Science}{KU Leuven, 2024}
	\cvitem{B.Sc.\ in Informatics}{KU Leuven, 2017}
\end{cvlist}

\subsection*{Publications}
\nocite{%
	VanDenKerchove2020,
	Libert2022,
	VanDenKerchove2022,
	VanDenKerchove2024,
}
\printbibliography[heading=none, resetnumbers=true]


\subsection*{Awards \& funding}
\begin{cvlist}
	\cvitem{Clinical Trial Sponsorship Program}{Zeto, 2024 \\
		Our research project titled ``A gaze-independent Visual Brain-Computer
		Interface for use by patients with limited or no eye control'' won a spot in
		the Zeto Clinical Trial Sponsorship Program, granting us the resources to conduct research
		with the Zeto EEG headset.
	}
	\cvitem{Student BCI competition 4th place worldwide}{NeuroTechX, 2022 \\
		Our NeuroTechLeuven BrainBrowsR project, a plug-and-play BCI system that lets you control
		social media applications  trough SSVEP-BCI was awarded 90 out of 100 points by
		the jury of the NeuroTechX international student club competition.
	}
	\cvitem{Global Ph.D. Partnership Grant}{KU Leuven and University, 2021--2024}
\end{cvlist}

\subsection*{Conferences and presentations}
\begin{cvlist}
	\cvitem{CORTICO Days}{CORTICO, Nancy, 2024 \\
		Poster presentation on covert visual attention BCIs for patients with
		oculomotor impairment.}
	\cvitem{10th International BCI Meeting}{BCI Society, Brussels, 2023\\
		Poster presentation on classifier-based latency estimation for
		gaze-independent BCIs.}
	\cvitem{Leuven.AI Scientific Workshop 2022}{Leuven.AI, Leuven, 2022\\
		Poster presentation on (multi-)Kronecker-structured linear discriminant
		analysis for low sample size event-related potential classification.}
	\cvitem{CORTICO Days}{CORTICO, Grenoble, 2022\\
		Presentation on Kronecker-structured LCMV-beamforming for event-related potential
		classification.}
	\cvitem{CORTICO Days}{CORTICO, 2021\\
		Presentation on a multi-component approach
		to spatiotemporal beamforming decoding of event-related potentials.}
	\cvitem{BCI \& Neurotechnology Spring School 2024}{g.tec, 2024}
	\cvitem{Closed Loop Neurotechnologies Autumn School}{NeurotechEU, Lille, 2023}
	\cvitem{CORTICO Days}{CORTICO, Paris, 2023}
	\cvitem{BCI \& Neurotechnology Spring School 2023}{g.tec, 2023}
	\cvitem{BCI \& Neurotechnology Masterclass Belgium}{g.tec, 2022}
	\cvitem{BCI \& Neurotechnology Spring School 2022}{g.tec, 2022}
\end{cvlist}

\subsection*{Teaching Experience}
\subsubsection*{Classes}
\begin{cvlist}
	\cvitem{Teaching Assistant Brain-Computer Interfaces}{KU Leuven, 2022--2024\\
		Teaching exercise sessions in BCI design and signal processing to students in
		the Master of Bioengineering and Advanced Master of Artificial
		Intelligence.}
	\cvitem{Teaching Assistant Fundamentals of Computer Science}{KU Leuven,
		2022 \\
		Teaching exercise sessions in Python programming and algorithmic reasoning to
		students in the Bachelor of	Engineering Sciences.}
\end{cvlist}

\subsubsection{Master theses and internships}
\begin{cvlist}
	\cvitem{Eye-tracker and ERP data fusion for gaze-independent visual
		BCI}{\ \\ Juliette  Meunier, University of Lille, 2024}
	\cvitem{LDA in	the combined
		space-time-frequency domain\\ for BCI decoding}{\ \\Lunkyadi Sucipto, KU Leuven,
		2023--2023}
	\cvitem{Tackling the Midas Touch problem in eye tracking with a
		BCI}{\ \\Reniflal Ebenezer Sundaralal, KU Leuven, 2023--2023}
	\cvitem{An EEG connectivity analysis of Alzheimer’s Disease
		and Frontotemporal Dementia}{Zoe Barinaga, KU Leuven, 2023--2024}
	\cvitem{Single-trial ERP latencies as a predictor for the mode
		of visual attention}{\ \\Yildiz Dilara Parry, KU Leuven, 2023}
	\cvitem{A hybrid P300-gaze BCI alternative for navigating virtual
		spaces}{\ \\Gijs Claes, KU Leuven, 2021--2022}
\end{cvlist}

\subsubsection*{Extracurricular}
\begin{cvlist}
	\cvitem{Project supervisor}{NeuroTech Leuven, 2022--ongoing\\
		Project supervisor and technical advisor for extracurricular
		neurotechnology student projects and a student team competing in the
		annual NeuroTechX BCI competition.}
	\cvitem{PAL tutor Principles of Computer Programming}{KU Leuven, 2015--2016\\
		Organizing and teaching peer-assisted learning sessions in Python programming for first year students.}
\end{cvlist}

\subsection*{Professional and volunteering Experience}
\begin{cvlist}
	\cvitem{Ambulance EMT}{Red Cross, FAST	vzw, 2022-ongoing}
	\cvitem{Python developer}{Mindspeller, 2019
		Python Flask developer in a KU Leuven spin-off that provides neuromarketing services based on
		scientifically validated neuroscience and AI research.}
	\cvitem{Freelance full-stack web developer}{Self-employed,
		2014--2020}
\end{cvlist}

\subsection*{Licenses and certifications}
\begin{cvlist}
	\cvitem{ICH Good Clinical Practice}{E6 TransCelerate BioPharma Inc.}
	\cvitem{Ambulance EMT}{FOD Volksgezondheid}
\end{cvlist}

\subsection*{Languages}
\begin{cvlist}
	\cvitem{Dutch}{Native}
	\cvitem{English}{Fully proficient}
	\cvitem{French}{Advanced}
	\cvitem{German}{Intermediate}
\end{cvlist}

\end{document}
