\chapter{Conclusion \& recommendations}
\todo{https://sci-hub.ru/https://ieeexplore.ieee.org/abstract/document/1642754}
\todo{https://www.sciencedirect.com/science/article/abs/pii/B9780444639349000020}
\todo{BCI principles and practice chapter 19}
\todo{https://www.nature.com/articles/s44222-024-00239-5}

\section{Contributions}

\todo{come back on hypothesis}


\subsection{Developed \Ac{erp} decoders}
We introduced a covariance estimator using adaptive shrinkage (STBF-shrunk) and an estimator
exploiting prior knowledge about the spatiotemporal nature of the EEG signal
(STBF-struct).
We compared these estimators with the original formulation of the
spatiotemporal (STBF-emp)
beamformer and a state-of-the-art Riemannian Geometry method (XDAWN+RG) in an off-line P3 detection task on
an existing dataset.
Our results show that the structured estimator results in higher accuracy in
general, and specifically when training data are sparsely available.
Results can be computed faster and with
substantially less memory usage.
Since these algorithms are not paradigm-specific, the conclusions can be
generalized to other ERP-based BCI settings.
These results have been published in~\cite{VanDenKerchove2022}.

We have tested our proposed methods and similar methods in gaze-independent and
covert VSA settings, but we did not observe a relative increase in performance
in these settings.

\todo{tensor}
\todo{wcble}

%* Preliminary results: BTTDA better than HODA, on par with other state of the
%art methods and in some situations better (data availability, noise structure)

\subsection{Gathered datasets}
The CVSA-ERP dataset consists of recordings of 15 healthy participants, mean age
$26.38\pm3.15$ years.
This dataset gives us insight in the ERP dynamics in overt, covert and the
novel split VSA condition, and confirms our hypothesis that P3 jitter has a
significant impact on performance in covert and split VSA, as well as that the
amplitude effect\todomvh{not the right word, maybe the effect of covert VSA on
ERP...} observed for covert VSA also holds for split VSA.

\todo{describe patients dataset}



\subsection{Investigated gaze-independence visual \acp{bci}}

We have evaluated our proposed wCBLE algorithm, the baseline CBLE algorithm and
a state-of-the-art algorithm (tLDA)~\cite{Sosulski2022} that uses similar techniques as our
proposed STBF-struct method on this dataset as well as a publicly available
dataset~\cite{Aloise2012} that also contains the overt and covert VSA conditions.
We evaluated the BCI decoding performance in a single-trial classification experiment,
as well as in a target selection experiment reflecting BCI operation.
Performance was significantly different between decoders, but this
result was significantly dependent on the VSA condition and the dataset.
While CBLE performed overall on par with the state-of-the art method for both datasets, wCBLE was
outperformed by tLDA in overt VSA decoding but outperformed
tLDA for covert VSA.
For the split attention conditions in the CVSA-ERP
dataset, wCBLE yielded a significant improvement over CBLE and the
state-of-the-art method only for d = 1 and d = 3.
These results were corroborated by analyzing selection accuracy and information
transfer rate, which showed similar behavior for both datasets, except in overt
VSA, where accuracy and information transfer rate were not harmed by the lower
single-trial selection performance of wCBLE.

To further study the gaze-independent performance of these algorithms, transfer
learning between VSA conditions is studied to simulate conditions where a patient
can end up in different VSA settings within a BCI operation session due to their
lack of eye motor control.
When trained and evaluated on overt VSA data, our proposed wCBLE algorithm results in a significant decrease in performance
over the state-of-the art tLDA decoder, consistent with the within VSA
condition results.
For all other pairs of training and evaluation VSA condition, however, wCBLE
never yields a significant decrease in performance and often significantly
outperforms the other methods.




These results show that there is an interest in developing a new class of ERP-BCI
interfaces for patients that prefer to rest their eyes on a chosen point on the
screen, so as to avoid the effort of redirecting their eye gaze at different spatial locations on
the stimulation screen. Furthermore, the performance gain in the split VSA conditions
and in the between-VSA condition transfer settings are promising for patients with even
less eye motor control, such as patients experiencing involuntary saccades or
fatigue. An accurate decoder would allow them to comfortably operate a BCI while
resting their eyes on whichever portion of the screen they prefer, even when there is
another target present at that location or this location varies during the course of the
experiment. Our results show that whenever the eye gaze is directed at a different
spatial location other than that of the mentally attended target, the wCBLE decoder can
more accurately discern the locus of mental attention, promoting in this way accurate
gaze-independent decoding. This comes, however, at the cost of a decreased performance
compared to overt VSA operation.

\todo{. Coincidentally, this also
builds towards a solution for the Midas Touch Problem in BCI, where a BCI user sometimes accidentally
selects a target while not wanting to give any input. Decoding of mental attention independent of eye
gaze, with the option of having visual attention without mental attention,
would counteract this.}


\section{Limitations}

hard to evaluate experience in long term home use
unify clinical side and bci development side

very small group of interest, but available right now, no implantation needed,

even if it is the most suited strategy for a patient, it will probably be
improved if a combination of eye tracking, residual motor control and other
biosensors are used.

true power in the future is probably focused on
invasive, active paradigms if recording technology and surgical methods
improve lowering the risk. orders of magnitude higher itr and more natural
communication.
characterized by publication date, but might indicate it is of interest to
heavily focus this research now on applied 'vraagstukken' with patients


\section{Recommendations}

\subsection{Decoding: keep it linear}

keep it linear
use linear methods, but impose specific structure on them reflecting properties
of the signal, either to make them better regularized or more robust to some
types of variation

\subsection{Working \emph{with} individuals \acs{sspi}}

this work focused on classification methods, and solving problem through
decoding

classification performance is not everything

ITR in on-line operation is the ultimate goal
, and, ultimately, user satisfaction, which depends on performance, comfort and perception
also do advance calculations of what the information transfer rate would be
given reasonable accuracy, so that you can go in with a reasonable interface
that has state-of-the-art ITR

classification performance is convenient since you can do a single data
collection and iterate offline.
this constricts you to decoder development
it is a better approach to only go there after optimizing the interface
experimentally and with patients with an existing
decoder, and only moving there once the options for progress are depleted

performance and abilities are probably so user dependent that it is not said that
findings from a population study with healthy controls will hold for a given
user with \ac{sspi}.
Incremental gains in performance in decoding for a specific paradigm will be
outweighed by how well that paradigm is adapted to the user, factoring in,
skills and abilities.

best approach should be everything working together, such as is advocated
by~\textcite{Pan2022} and \textcite{Fouad2020}

instead of decoder > data > patients > online
go other way
online > patients > data > decoder
this is unfortunately also the least practical way to do it for a 4-year
research project.

It is therefore important that research labs keep a working, long-term collaboration with
patient centers such that new projects can immediately be tested by the target
population, and that patients with bci experience can compare multiple systems
and guide development
The fact that a user has previous bci experience is often seen as a confounding
factor, while it is a strength that can help you gain insight into what works
and doesn't work for them, and where research efforts should be focused.

also maintain a working, on-line system that can be iterated upon

in a phase that is heavily dependent on user experimentation

not best approach for all research projects, but
Visual oddball bci field is advanced enough to do this


\subsection{Working with patients}
gaze independent visual, can be good current existing solution
take into accounrecommendations of \textcite{FriedOken2020}

immediately move to patients and on-line implementation, to be able to report
interpretable metrics
permanent working collaboration as a lab

In a mature field like visual bci, start from patients.
avoid situations where a problem is solved that doesn't pose itself in a
patient population
use all available capabilities and signals

Have attention for the personal aspect


\section{Current \& future work}
kronecker-sum lda
multicomponent alignment approach
hankel tensors to cope with jitter
select best classifier using eye tracking
expand on patient experiemnts

\section{Conclusion}
