\todo{mention partially published}
\todo{spatially vs temporally organized interfaces}
\todo{talk about covert shifts in spatial attention alpha}

\section{The gaze-dependence problem}

One of the goals of Brain-computer interfacing is establishing a communication
channel that does not rely on speech or
muscular activity~\cite{Naci2012,Chaudhary2016}, which in turn can provide
solutions to paralyzed individuals.
%The quest for performant and affordable solutions is most evident in
%visual BCIs using the electroencephalogram (EEG), where it has led to a gamut
%of increasingly sophisticated decoders, tailored to the needs
%of specific stimulation paradigms and usage contexts.

Traditional visual BCI scenarios require the user to overtly direct their
visuospatial attention (VSA) and gaze toward the screen target they intend to select.
In most settings, screen targets are overlaid with non-overlapping, transient
stimuli that evoke event-related EEG potentials (ERPs).
The selected target can be decoded from these ERPs, as is the case for the
oddball paradigm where observing a rare but attended stimulus evokes a P3 ERP
component.
However, a critical challenge arises when users rely solely or in part on covert
VSA, which involves directing visuospatial attention without corresponding eye gaze.
In these cases, classical solutions often fall short of the widely accepted
80\% target selection accuracy threshold deemed necessary for a comfortable user
experience~\cite{Brunner2010,Frenzel2011,Treder2010,Ron2019,Neeling2019},
calling for alternative, gaze-independent solutions.

In this work, we will use the term \emph{gaze-independent} meaning
\emph{‘dealing explicitly with the fact that a user cannot control their
gaze.'}
In the context of a visual BCI, this means that the user's
visuospatial attention and their gaze do not necessarily coincide.
Gaze-independent paradigms are particularly
promising for individuals with impaired eye motor control, such as those
suffering from specific types or stages of Amyotrophic Lateral Sclerosis (ALS),
Multiple Sclerosis (MS), stroke, or brain stem
stroke.
For these patients, gazing directly at a screen target may be uncomfortable, impractical,
or even impossible.
Hence, assistive devices that rely on eye tracking are often inefficient for
them.
Consequently, while BCIs hold great promise for these individuals, conventional
gaze-dependent BCI solutions do not meet their needs due to the absence of gaze
control. Therefore, the development of decoding strategies that account for covert
VSA becomes crucial in the pursuit of high-performance gaze independent
BCIs.

\section{Incidence of eye motor impairment}
Table~\ref{tab:incidence} reports the relatively high frequencies of
eye motor problems, which can range from involuntary eye movements to control
issues and even to partial or complete ophtalmoplegia.
\begin{table}
	\centering
	\begin{tabular}{@{}|l|rrrrr|r|@{}}
		\hline
		         & ALS  & MS   & Stroke  &
		DMD      & SMA  & LiS                           \\ \hline
		Minor    & 50\% & 31\% & 40-70\% & + & - &      \\
		Severe   & 33\% & 3\%  & +       & - & - & 98\% \\
		Complete & 17\% & -    & +       & - & - & 2\%  \\
		\hline
	\end{tabular}
	\caption{Incidence of eye motor impairment in Brain-Computer Interface target
		populations: (ALS: Amyotrophic Lateral Sclerosis, MS: Multiple Sclerosis,
		DMD: Duchenne's Muscular Dystrophy, SMA: Spinal Muscular Atrophy, LiS:
		Locked-In-Syndrome).}\label{tab:incidence}
\end{table}
\todo{document methodology}
\todo{find more references, fill out}
\todo{
Patients with
Totally locked in/locked in with eye motor issues due to
Brain stem stroke
ALS
MS
Motor and eye motor impairments due to neurotrauma (tetraplegia)
...
}

\todo{define eye motor issues}
\todo{systematic review of prevalences and prevalences of eye motor issues per disorder}
\todo{
cite{Guo2022}
cite{Prasad2010}
cite{Serra2018}
cite{Bogousslavsky1987}
cite{Castelnovo2016}
cite{Polet2020}
}

\section{Gaze-independent BCI paradigms}
\label{sec:gaze-independence-paradigms}
Gaze-independent ERP-based BCIs~\cite{Riccio2012, Aloise2012} can be realized in three
ways. Firstly, non-visual stimulation paradigms such as auditory and somatosensory paradigms
do not rely on gaze redirection but often result in lower information transfer
rates, increased mental effort and user-dependent variability~\cite{Reichert2020b}.
\todo{insert table from riccio in text as numbers, not as table}

Secondly, visual stimulation can be optimized, e.g. such that the stimuli are always
present in the field of view either overtly~\cite{Acqualagna2013, Won2018,
Lin2018} or covertly~\cite{Treder2010,Pires2011,Lees2018}. Some noteworthy examples include the GIBS Block Speller~\parencite{Pires2011},
the GeoSpell interface~\parencite{Aloise2012}, and the RSVP
speller~\parencite{Acqualagna2011}.
Non-spatial visual attention (feature attention) can also be exploited, such as
attention to stimulus color, shape or symbol~\cite{Zhang2010,Treder2011,Hwang2015}.
Alternative visual stimulation paradigms can modulate specific ERP components
that are more sensitive to stimulation in the visual periphery~\cite{Schaeff2012,Xu2022}.
Nevertheless, these methods, can suffer too from reduced information
transfer rates~\cite{Chennu2013}.
Furthermore, these systems typically require users to focus on a central fixation point while
selecting peripheral targets.
This entails that they still rely to some extent on
eye motor control, often necessitating central gaze fixation.
\todo{advantages disadvantages rewrite (requires gaze fixation, low itr, not tested with split attention)}
\todo{
Firstly, by avoiding visual
stimulation entirely and opting for auditory or somatosensory stimulation instead.
However, these alternatives often result in lower information transfer rates,
increased mental effort and variable outcomes for different users.
Secondly, gaze-independence can
be achieved by adapting the interface to display stimuli such that they
are always in the center of the field of view, by exploiting visual
non-spatial attention (feature attention), or a combination thereof.
These interfaces suffer equally from reduced information transfer rates.
}

Thirdly, stimuli can be presented in a standard BCI paradigm, but visuospatial attention
can be decoded separately from gaze direction.
Earlier work has made advances in decoding lateralized covert
VSA harnessing the N2pc ERP component~\cite{Thiery2016,Reichert2020b,Wang2022},
decoding covert VSA shifts from spectral content~\cite{Tonin2013}, or hybrid
stimulation paradigms~\cite{Egan2017}.
Since these methods often require a slower stimulation pace than the classical
ERP BCI paradigm, gaze-independent decoding in fast-paced ERP paradigm stimulation
remains underexplored.
\cite{Aloise2012b} aimed to bridge the performance gap between covert and
overt VSA decoding performance.
They compared classical linear and non-linear ERP classifiers on a covert
attention P3 ERP component dataset.
The results revealed no significant performance improvement in covert VSA
decoding for any of the investigated decoders.
\cite{Arico2014} observed higher variability in single-trial P3 peak
latencies relative to stimulus onset during covert VSA compared to overt VSA.
This latency variability contributes to reduced covert VSA decoding performance.
While they proposed an analysis and performance prediction method, they did not provide a
decoding solution.
They suggested that compensating for latency jitter could enhance covert VSA
decoding, but did not actually verify this hypothesis directly.
Additionally, \cite{hardiansyah2020single} developed a classifier for
covert VSA ERPs, exploiting single-trial latency features in combination with
amplitude features for classification with a support vector machine.
They demonstrated the positive influence of single-trial ERP component latency
features on covert VSA inference yet did not attempt to correct the amplitude
features for these latencies.
\todo{Should we already mention Frenzel here?}
\cite{Frenzel2011} introduced a similar protocol to our split VSA
setting.
They showed that it is possible to perform split VSA and that, in this
case, visuospatial attention and gaze direction can separately be decoded using
classical ERP techniques.
To the best of our
knowledge, this is the sole study that investigated split attention in ERP-based BCIs.
However, \cite{Frenzel2011} considered their interface only for the case
where the user actively intends to select both targets determined by the gaze
and the VSA.
In the split VSA setting considered in our work, we rather instruct the participant to ignore
the distractor and only attend the cued target, since we are interested in
decoding the visuospatial attention only.

P3 latency generally falls between 350ms and 600ms~\cite{Luck2014}, but this
value is heavily dependent on the subject and the task, and can vary from trial
to trial~\cite{Ouyang2017}.
The work of \cite{Arico2014}
illustrates that the variation in single-trial P3 latencies is important in
gaze-independent decoding and has been hampering covert VSA decoding performance.
In this work we aim to reprise their hypothesis stating that jitter compensation improves
covert VSA performance and extend it by developing a decoder and evaluating it
in a broader range of gaze-independent settings, including split VSA.
\todo{Check connection}
