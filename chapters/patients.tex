\section{Introduction}
\todo{rename patient to bci user and participant}

Communication BCIs have a target population consisting of patients in various
stages of paralysis or \ac{lis}, often suffering from oculomotor impairments.
This reduces their performance operating visual oddball \ac{bci}s (see
section\todo{ref section} for an overview), since they
are presumed unable to comfortably redirect their gaze at the desired target,
i.e., perform overt \ac{vsa}.\todo{cite some visual gaze-independent bci}
They might more naturally operate in covert \ac{vsa}, where the gaze and
\ac{vsa} do not coincide.
Several studies show that performance drops when not fixating the intended
target~\cite{Brunner2010, Frenzel2011, Treder2010, RonAngevin2019,
VanDenKerchove2024}, necessitating gaze-independent solutions.

Usually, they aim to improve gaze-independent \ac{bci} by optimizing
performance in cued covert \ac{vsa}.
These studies build on the assumption that patients with eye motor impairment
would feel comfortable operating an interface in pure covert \ac{vsa} with
central fixation.
One could argue that a \ac{bci} that is only verified to work when a central
fixation is maintained, could also be considered gaze-dependent.
This does not account for the residual eye motor capabilities of most severly
paralyzed or locked-in patients and the comfort they experience while
performing the gaze fixation.

Our previous study, presented in chapter \todo{ref chapter}  partially tried to account for this
and showed gaze-independent performance can be improved
improved in healthy subjects using a suited decoding strategy.
Yet, there is a strong need for verification of results obtained in healthy
samples.
It is a striking constatation that studies reporting on
gaze-independent visual \ac{bci} in patients that are actually eye-motor
impaired are very few.
in patients, cite different characteristics, differences
in setting, environment and equipment,
\todo{List conclusions of Peters2020 (severely eye motor impaired als patients
able to use a hex-o-spell)}

Eventually, one of the end goals of this research line is to develop
gaze-independent \ac{bci} for patients that
are fully locked-in and have no option left than to use a \ac{bci}.
However, this group is very small and it is often a challenge to recruit them
into a study and perform experiments with them~\cite{Wolpaw2006}.
Patients with less severe paralysis or in less progressed disease stages that struggle with
eye-tracking technology could also benefit from
solutions tailored to their specific situation.

Therefore, we aim to apply the concepts from earlier work an literature to
patients with various degrees motor impairment and various
degrees of eye motor impairment.
The objectives of this case study are as follows:
\begin{enumerate*}
  \item Explore capabilities and experienced comfort of eye motor impaired
    patients,
  \item evaluate the performance of a gaze-independent visual \ac{bci} in patients
  \item verify if this performance can be improved with a suitable decoding
    strategy.
\end{enumerate*}


\section{Materials \& methods}
\subsection{Patient recruitment}
Patients were recruited from across the Neuromuscular Reference center at
University Hospital Leuven (Leuven, Belgium), TRAINM Neuro Rehab Clinics
(Antwerp, Belgium), the Neurorehabilitation Unit at University Hospital Lille
(Lille, France) and a specialized care home (France).
Experiments were performed under the supervision of their treating physician.
Patients were recruited based on the following criteria.
To qualify for inclusion, patients must
\begin{enumerate}
	\item be at least 18 years old and no older than 60
	years,
  \item belong to class 2 or 3 according the \ac{bci}	patient selection criteria
    presented by~\textcite{Wolpaw2006},\label{item:patients/inclusion/wolpaw}
  \item have limitations to the extent or comfort of their eye motor control\label{item:patients/inclusion/oculomotor}
\end{enumerate}
Patients were excluded if they
\begin{enumerate}
  \item have a diagnosis of a major medical condition, including any major
    neurological or psychiatric disorder other than those of interest based on
    inclusion criteria~\ref{item:patients/inclusion/wolpaw}
    and~\ref{item:patients/inclusion/oculomotor}\label{item:patients/exclusion/medical}
  \item have a predisposition to or have a history of any kind of epileptic seizures,
    including photosensitive epilepsy,\label{item:patients/exclusion/epilepsy}
  \item have a severe loss in vision or hearing, that would significantly impair
        participation in the experiment,\label{item:patients/exclusion/vision}
  \item are currently using specific psychoactive medications or substances that could affect the out-
        come.\label{item:patients/exclusion/cognitive}
  \item be able to understand the experiment instructions and cooperate,
  \item have any other limitations preventing them from performing the given task.
\end{enumerate}

In total, 11 patients were contacted, of which 1 \ac{ms} patient was excluded based on
criterion~\ref{item:patients/exclusion/vision}, 1 \ac{tbi} patient on
both~\ref{item:patients/exclusion/epilepsy}
and~\ref{item:patients/exclusion/cognitive}, and one stroke patient based
on~\ref{item:patients/exclusion/medical}
One further stroke patient was excluded due to technical
difficulties during the experimental session.
Vision was assessed using a LogMAR chart~\cite{Bailey1976}.

Ultimately, 7 patients were retained.
Of these, one patient was diagnosed from bulbar-onset \ac{als}.
\ac{als} is a neurodegenerative disease affecting the motor neurons, leading to
progressive loss of motor function.
This initially results in general weakness and loss of muscle tone, but
eventually results in full body paralysis.\todo{cite}
Although speech and especially eye movements are usually preserved until the
later stages of the disease progression, the bulbar-onset variant is
characterized by an early loss of speech and an increased involvement of eye
motor symptoms~\cite{Guo2022}\todo{which?}.
Furthermore, one of the goals of \ac{bci} is to support \ac{als} patients whose
life span has been extended with life support with and assistive technology to
ensure quality of life.
In these very progressed stages of \ac{als}, eye movement will eventually also
be affected~\cite{Hayashi1991}.

Three other patients were diagnosed with \ac{fa}, a neurodegenerative
disease affecting the
spinal cord, peripheral nervous system and cerebellum.
\ac{fa} results in an impairing loss of muscle coordination.
In progressed stages, the disease can present with nystagmus, saccadic
intrusions and gaze dysmetria~\cite{Cook2017}.

The final three patients were stroke patients.
Brain stem and cerebellar stroke can often lead to forms of \ac{lis}, and
various eye motor disorders are common depending on the exact location
~\cite{Bogousslavsky1987, Moncayo2009}.
Stroke can lead to some of the more severe impairments like ophthalmoplegia or
partial ophthalmoplegia immediately from onset.

Table~\ref{tab:patients/patients} lists the included patients and their
diagnoses.

\todo{check Usability and Workload of Access Technology for People With Severe
Motor Impairment: A Comparison of Brain-Computer Interfacing and Eye Tracking}

\begin{table}[t]
  \centering
  \footnotesize
  \begin{tabular}{llllllllr}
  \toprule
  \textbf{ID}  & \textbf{Age} & \textbf{Sex} & \textbf{Hand.} &
  \textbf{Diagnosis}
  & \textbf{Speech}     & \textbf{Trach.} & \textbf{Communication}          &
  \textbf{Cls.} \\ \midrule
  PA1 & 58  & M   & L     & bulbar-onset \acs{als} & anarthric  & no          & tablet                 & 3  \\
  PB1 & 41  & M   & L     & \acs{fa} & dysarthric & no          & verbal                 & 3  \\
  PB2 & 43  & F   & R     & \acs{fa} & dysarthric & no          & verbal                 & 3  \\
  PB4 & 48  & M   & R     & \acs{fa} & dysarthric & no          & verbal                 & 3  \\
  PC2 & 43  & M   & R     & brainstem stroke & anarthric  & yes         & \makecell[l]{prompting\\+eye movement} & 2 \\
  PC3 & 43  & F   & R     & brainstem stroke & anarthric  & yes         & letterboard            & 2 \\
  PC4 & 54  & M   & R     & \makecell[l]{left cerebellar stroke \\ (trombosis of the basilar artery)} & anarthric  & yes & letterboard & 2 \\
  \bottomrule
\end{tabular}

  \caption[Presentation of included patients including their diagnosis and
  capabilities.]{Presentation of included patients including their diagnosis and
  capabilities.
  (Trach.: patient underwent a tracheotomy, Cls.: classification according
  to~\textcite{Wolpaw2006}).
  }
  \label{tab:patients/patients}
\end{table}
\todo{Replace with classification according to 40. Kübler A, Birbaumer N.
Brain-computer interfaces and communication in paralysis: extinction of goal
directed thinking in completely paralysed patients? Clin Neurophysiol.
2008;119:2658-2666.}
\todo{include year of diagnosis}

\subsection{Eye tracking and eye motor examination}

Self-reported eye motor abnormalities were recorded.
Patients were asked for, the paralysis or inability to move in a given
direction of one or both eyes, the inability to pursue an object to specific
positions, discomfort or fatigue fixating at specific positions or in general
when performing visual tasks and eye tremors while fixating, resting or
pursuing.

Additionaly, we implemented and performed the automated NeuroEye eye movement
test proposed by~\textcite{Hassan2022}.
This test is based on calibration-free eye tracking.
\todo{say which part we used and which part not}
\todo{motivation and text and references Juliette}
\todo{Limited in what it can tell us (check paper)}
\todo{include eye test document juliette}
Table~\ref{tab:patients/eye} details their eye motor impairments and vision of
included patients.
Patient PA1 had the mildest impairment, only reporting fatigue when fixating
for prolonged times.
Eye motor function of subjects PC2 and PC3 were most severly affected.
Patient PC2 was only able to look up and down, and patient PC3 only retained
partial movement of the right eye, while the left eye was permanently closed.

\begin{table}[t]
  \centering
  \footnotesize
  % \begin{tabular}{lll}
%  \toprule
%  \textbf{ID} & \textbf{Oculomotor impairment} & \textbf{Vision (LogMAR)} \\ \midrule
%  PA1  & fixation fatigue & 0.0 \\
%  PB1  & impaired pursuit, fixation fatigue, fixation discomfort, tremor & 0.0 \\
%  PB2  & fixation fatigue, fixation discomfort, tremor & 0.6 \\
%  PB4  & impaired pursuit, fixation fatigue, fixation discomfort, tremor & 0.2 \\
%  PC2  & \makecell[l]{partial ophthalmoplegia (up-down preserved), \\ fixation
%  fatigue, fixation discomfort, tremor} & \makecell[l]{0.0, diplopia corrected \\ with prism glass)} \\
%  PC3  & \makecell[l]{right ophthalmoplegia, left partial ophthalmoplegia, \\ fixation fatigue, fixation discomfort, tremor} & 0.7, right eye closed \\
%  PC4  & deviation of the left eye & 0.6 \\
%  \bottomrule
%\end{tabular}
\footnotesize
\let\oldarraystretch\arraystretch
\renewcommand{\arraystretch}{2}
\newcommand{\skill}{\cellcolor{lightgray}}
\newcommand{\noskill}{\cellcolor{accent1}\textcolor{muteblack}{\BigCross}}
\newcommand{\snoskill}{\cellcolor{accent2}\textcolor{muteblack}{\BigDiamondshape}}
\begin{tabular}{r|ccccccc}
                          & PA1      & PB1      & PB2       & PB4      & PC2       & PC3       & PC4 \\ \hline
  Visual fixation         & \noskill & \noskill & \noskill  & \noskill & \noskill  & \noskill  & \noskill \\
  Eyelid function         & \skill   & \skill   & \skill    & \skill   & \noskill  & \noskill  & \skill \\
  Ocular motility         & \skill   & \noskill & \skill    & \noskill & \snoskill & \snoskill & \noskill\\
  Binocular vision        & \skill   & \skill   & \skill    & \skill   & \noskill  & \snoskill & \snoskill \\
  Field of vision         & \skill   & \skill   & \skill    & \skill   & \skill    & \noskill  & \noskill \\
  Visual perception       & \skill   & \skill   & \skill    & \skill   & \skill    & \skill    & \skill\\
  Involuntary movement    & \skill   & \noskill & \noskill  & \noskill & \noskill  & \noskill  & \skill \\ \hline
  Visual acuity (logMAR)  & 0.0      & 0.0      & 0.6       & 0.2      & 0.0       & 0.7  & 0.6\\
\end{tabular}

  \caption[Vision and eye motor impairment for included patients.]{%
  Vision and eye motor impairment for included patients.
  Eye motor impairment was assessed with a combination of self-reported issues
  by the subject and the NeuroEye~\cite{Hassan2022} test.
  LogMAR: lower is better.}
  \label{tab:patients/eye}
\end{table}

Eye tracking throughout the experimental session was performed using the Tobii~\todo{model, make}
portable eye tracker.
\todo{describe analysis during experiment}

\subsection{\Ac{bci} stimulation}

The \ac{bci} stimulation procedure was based on the
Hex-o-Spell~\cite{Treder2010} implementation presented
by~\textcite{VanDenKerchove2024}.
Similar to this study, the task consists of counting the flashes of a cued
target among all flashing targets.
We refer to section~\ref{sec:covert-align/stimulation} for the exact details.

To adapt the interface for the patient setting, the number of blocks was
decreased to 6 per \ac{vsa} setting.
An additional \emph{free \ac{vsa}} condition was introduced.
Here, the patient was instructed to perform the task as they deemed most
comfortable.
The cued \emph{split attention} setting proposed
by~\textcite{VanDenKerchove2024} is not studied here, since we are interested
in natural \ac{vsa} operation settings for gaze-impaired patients.
If the patient was not fully paralyzed, they were instructed not to move the head.
\Ac{isi} was increased to\todo{number} to decrease task difficulty.
The experiment alsy started with a training block in each condition, where the
patient was instructed with feedback on their performance to make sure they
understood and were able to perform the task.

\fourpanefig{%
  \figpane{\includegraphics[width=\textwidth]{figures/patients/PD01a-obfuscated.jpg}}{A
  patient with the stimulation and recording setup.}{fig:patients/photos/side}
}{%
  \figpane{\includegraphics[width=\textwidth]{figures/patients/PD01b-obfuscated.jpg}}{subcaption}{fig:patients/photos/front}
}{%
  the interface
}{%
  4
}{%
  caption
}{fig:patients/photos}

Mention change in ISI
Mention less rounds

\subsection{Data collection \& preprocessing}

\todo{check checklist}
\subsection{BCI Decoding}

\section{Outcomes}

\subsection{Eye tracking analysis}

PC2 reported that he was unable to fixate the lower two targets, and th

\fullpagefig{%
  \includegraphics[width=\textwidth]{figures/patients/fig_gaze.png}
}{%
  Distribution of the recorded gaze position during the experimental session in the three \ac{vsa}
  conditions.
  Subjects PB2 and PC4 preferred covert \ac{bci} operation, with PB2 resting gaze
  near the middle of the screen, and PC4 near the bottom.
}{%
  fig:patients/gaze
}


\subsection{Decoding performance}

\begin{figure}[t]
  %% Creator: Matplotlib, PGF backend
%%
%% To include the figure in your LaTeX document, write
%%   \input{<filename>.pgf}
%%
%% Make sure the required packages are loaded in your preamble
%%   \usepackage{pgf}
%%
%% Also ensure that all the required font packages are loaded; for instance,
%% the lmodern package is sometimes necessary when using math font.
%%   \usepackage{lmodern}
%%
%% Figures using additional raster images can only be included by \input if
%% they are in the same directory as the main LaTeX file. For loading figures
%% from other directories you can use the `import` package
%%   \usepackage{import}
%%
%% and then include the figures with
%%   \import{<path to file>}{<filename>.pgf}
%%
%% Matplotlib used the following preamble
%%   \def\mathdefault#1{#1}
%%   \everymath=\expandafter{\the\everymath\displaystyle}
%%   
%%   \ifdefined\pdftexversion\else  % non-pdftex case.
%%     \usepackage{fontspec}
%%   \fi
%%   \makeatletter\@ifpackageloaded{underscore}{}{\usepackage[strings]{underscore}}\makeatother
%%
\begingroup%
\makeatletter%
\begin{pgfpicture}%
\pgfpathrectangle{\pgfpointorigin}{\pgfqpoint{6.585459in}{2.630850in}}%
\pgfusepath{use as bounding box, clip}%
\begin{pgfscope}%
\pgfsetbuttcap%
\pgfsetmiterjoin%
\pgfsetlinewidth{0.000000pt}%
\definecolor{currentstroke}{rgb}{0.000000,0.000000,0.000000}%
\pgfsetstrokecolor{currentstroke}%
\pgfsetstrokeopacity{0.000000}%
\pgfsetdash{}{0pt}%
\pgfpathmoveto{\pgfqpoint{0.000000in}{0.000000in}}%
\pgfpathlineto{\pgfqpoint{6.585459in}{0.000000in}}%
\pgfpathlineto{\pgfqpoint{6.585459in}{2.630850in}}%
\pgfpathlineto{\pgfqpoint{0.000000in}{2.630850in}}%
\pgfpathlineto{\pgfqpoint{0.000000in}{0.000000in}}%
\pgfpathclose%
\pgfusepath{}%
\end{pgfscope}%
\begin{pgfscope}%
\pgfsetbuttcap%
\pgfsetmiterjoin%
\pgfsetlinewidth{0.000000pt}%
\definecolor{currentstroke}{rgb}{0.000000,0.000000,0.000000}%
\pgfsetstrokecolor{currentstroke}%
\pgfsetstrokeopacity{0.000000}%
\pgfsetdash{}{0pt}%
\pgfpathmoveto{\pgfqpoint{0.379436in}{0.778861in}}%
\pgfpathlineto{\pgfqpoint{2.404777in}{0.778861in}}%
\pgfpathlineto{\pgfqpoint{2.404777in}{2.460711in}}%
\pgfpathlineto{\pgfqpoint{0.379436in}{2.460711in}}%
\pgfpathlineto{\pgfqpoint{0.379436in}{0.778861in}}%
\pgfpathclose%
\pgfusepath{}%
\end{pgfscope}%
\begin{pgfscope}%
\pgfpathrectangle{\pgfqpoint{0.379436in}{0.778861in}}{\pgfqpoint{2.025341in}{1.681851in}}%
\pgfusepath{clip}%
\pgfsetbuttcap%
\pgfsetmiterjoin%
\definecolor{currentfill}{rgb}{0.842157,0.553922,0.200980}%
\pgfsetfillcolor{currentfill}%
\pgfsetlinewidth{0.000000pt}%
\definecolor{currentstroke}{rgb}{0.000000,0.000000,0.000000}%
\pgfsetstrokecolor{currentstroke}%
\pgfsetstrokeopacity{0.000000}%
\pgfsetdash{}{0pt}%
\pgfpathmoveto{\pgfqpoint{0.404752in}{0.778861in}}%
\pgfpathlineto{\pgfqpoint{0.472264in}{0.778861in}}%
\pgfpathlineto{\pgfqpoint{0.472264in}{1.906614in}}%
\pgfpathlineto{\pgfqpoint{0.404752in}{1.906614in}}%
\pgfpathlineto{\pgfqpoint{0.404752in}{0.778861in}}%
\pgfpathclose%
\pgfusepath{fill}%
\end{pgfscope}%
\begin{pgfscope}%
\pgfpathrectangle{\pgfqpoint{0.379436in}{0.778861in}}{\pgfqpoint{2.025341in}{1.681851in}}%
\pgfusepath{clip}%
\pgfsetbuttcap%
\pgfsetmiterjoin%
\definecolor{currentfill}{rgb}{0.842157,0.553922,0.200980}%
\pgfsetfillcolor{currentfill}%
\pgfsetlinewidth{0.000000pt}%
\definecolor{currentstroke}{rgb}{0.000000,0.000000,0.000000}%
\pgfsetstrokecolor{currentstroke}%
\pgfsetstrokeopacity{0.000000}%
\pgfsetdash{}{0pt}%
\pgfpathmoveto{\pgfqpoint{0.657920in}{0.778861in}}%
\pgfpathlineto{\pgfqpoint{0.725431in}{0.778861in}}%
\pgfpathlineto{\pgfqpoint{0.725431in}{2.306133in}}%
\pgfpathlineto{\pgfqpoint{0.657920in}{2.306133in}}%
\pgfpathlineto{\pgfqpoint{0.657920in}{0.778861in}}%
\pgfpathclose%
\pgfusepath{fill}%
\end{pgfscope}%
\begin{pgfscope}%
\pgfpathrectangle{\pgfqpoint{0.379436in}{0.778861in}}{\pgfqpoint{2.025341in}{1.681851in}}%
\pgfusepath{clip}%
\pgfsetbuttcap%
\pgfsetmiterjoin%
\definecolor{currentfill}{rgb}{0.842157,0.553922,0.200980}%
\pgfsetfillcolor{currentfill}%
\pgfsetlinewidth{0.000000pt}%
\definecolor{currentstroke}{rgb}{0.000000,0.000000,0.000000}%
\pgfsetstrokecolor{currentstroke}%
\pgfsetstrokeopacity{0.000000}%
\pgfsetdash{}{0pt}%
\pgfpathmoveto{\pgfqpoint{0.911088in}{0.778861in}}%
\pgfpathlineto{\pgfqpoint{0.978599in}{0.778861in}}%
\pgfpathlineto{\pgfqpoint{0.978599in}{2.178356in}}%
\pgfpathlineto{\pgfqpoint{0.911088in}{2.178356in}}%
\pgfpathlineto{\pgfqpoint{0.911088in}{0.778861in}}%
\pgfpathclose%
\pgfusepath{fill}%
\end{pgfscope}%
\begin{pgfscope}%
\pgfpathrectangle{\pgfqpoint{0.379436in}{0.778861in}}{\pgfqpoint{2.025341in}{1.681851in}}%
\pgfusepath{clip}%
\pgfsetbuttcap%
\pgfsetmiterjoin%
\definecolor{currentfill}{rgb}{0.842157,0.553922,0.200980}%
\pgfsetfillcolor{currentfill}%
\pgfsetlinewidth{0.000000pt}%
\definecolor{currentstroke}{rgb}{0.000000,0.000000,0.000000}%
\pgfsetstrokecolor{currentstroke}%
\pgfsetstrokeopacity{0.000000}%
\pgfsetdash{}{0pt}%
\pgfpathmoveto{\pgfqpoint{1.164255in}{0.778861in}}%
\pgfpathlineto{\pgfqpoint{1.231767in}{0.778861in}}%
\pgfpathlineto{\pgfqpoint{1.231767in}{2.177982in}}%
\pgfpathlineto{\pgfqpoint{1.164255in}{2.177982in}}%
\pgfpathlineto{\pgfqpoint{1.164255in}{0.778861in}}%
\pgfpathclose%
\pgfusepath{fill}%
\end{pgfscope}%
\begin{pgfscope}%
\pgfpathrectangle{\pgfqpoint{0.379436in}{0.778861in}}{\pgfqpoint{2.025341in}{1.681851in}}%
\pgfusepath{clip}%
\pgfsetbuttcap%
\pgfsetmiterjoin%
\definecolor{currentfill}{rgb}{0.842157,0.553922,0.200980}%
\pgfsetfillcolor{currentfill}%
\pgfsetlinewidth{0.000000pt}%
\definecolor{currentstroke}{rgb}{0.000000,0.000000,0.000000}%
\pgfsetstrokecolor{currentstroke}%
\pgfsetstrokeopacity{0.000000}%
\pgfsetdash{}{0pt}%
\pgfpathmoveto{\pgfqpoint{1.417423in}{0.778861in}}%
\pgfpathlineto{\pgfqpoint{1.484934in}{0.778861in}}%
\pgfpathlineto{\pgfqpoint{1.484934in}{1.973583in}}%
\pgfpathlineto{\pgfqpoint{1.417423in}{1.973583in}}%
\pgfpathlineto{\pgfqpoint{1.417423in}{0.778861in}}%
\pgfpathclose%
\pgfusepath{fill}%
\end{pgfscope}%
\begin{pgfscope}%
\pgfpathrectangle{\pgfqpoint{0.379436in}{0.778861in}}{\pgfqpoint{2.025341in}{1.681851in}}%
\pgfusepath{clip}%
\pgfsetbuttcap%
\pgfsetmiterjoin%
\definecolor{currentfill}{rgb}{0.842157,0.553922,0.200980}%
\pgfsetfillcolor{currentfill}%
\pgfsetlinewidth{0.000000pt}%
\definecolor{currentstroke}{rgb}{0.000000,0.000000,0.000000}%
\pgfsetstrokecolor{currentstroke}%
\pgfsetstrokeopacity{0.000000}%
\pgfsetdash{}{0pt}%
\pgfpathmoveto{\pgfqpoint{1.670591in}{0.778861in}}%
\pgfpathlineto{\pgfqpoint{1.738102in}{0.778861in}}%
\pgfpathlineto{\pgfqpoint{1.738102in}{1.964477in}}%
\pgfpathlineto{\pgfqpoint{1.670591in}{1.964477in}}%
\pgfpathlineto{\pgfqpoint{1.670591in}{0.778861in}}%
\pgfpathclose%
\pgfusepath{fill}%
\end{pgfscope}%
\begin{pgfscope}%
\pgfpathrectangle{\pgfqpoint{0.379436in}{0.778861in}}{\pgfqpoint{2.025341in}{1.681851in}}%
\pgfusepath{clip}%
\pgfsetbuttcap%
\pgfsetmiterjoin%
\definecolor{currentfill}{rgb}{0.842157,0.553922,0.200980}%
\pgfsetfillcolor{currentfill}%
\pgfsetlinewidth{0.000000pt}%
\definecolor{currentstroke}{rgb}{0.000000,0.000000,0.000000}%
\pgfsetstrokecolor{currentstroke}%
\pgfsetstrokeopacity{0.000000}%
\pgfsetdash{}{0pt}%
\pgfpathmoveto{\pgfqpoint{1.923758in}{0.778861in}}%
\pgfpathlineto{\pgfqpoint{1.991270in}{0.778861in}}%
\pgfpathlineto{\pgfqpoint{1.991270in}{1.890869in}}%
\pgfpathlineto{\pgfqpoint{1.923758in}{1.890869in}}%
\pgfpathlineto{\pgfqpoint{1.923758in}{0.778861in}}%
\pgfpathclose%
\pgfusepath{fill}%
\end{pgfscope}%
\begin{pgfscope}%
\pgfpathrectangle{\pgfqpoint{0.379436in}{0.778861in}}{\pgfqpoint{2.025341in}{1.681851in}}%
\pgfusepath{clip}%
\pgfsetbuttcap%
\pgfsetmiterjoin%
\definecolor{currentfill}{rgb}{0.842157,0.553922,0.200980}%
\pgfsetfillcolor{currentfill}%
\pgfsetlinewidth{0.000000pt}%
\definecolor{currentstroke}{rgb}{0.000000,0.000000,0.000000}%
\pgfsetstrokecolor{currentstroke}%
\pgfsetstrokeopacity{0.000000}%
\pgfsetdash{}{0pt}%
\pgfpathmoveto{\pgfqpoint{2.176926in}{0.778861in}}%
\pgfpathlineto{\pgfqpoint{2.244437in}{0.778861in}}%
\pgfpathlineto{\pgfqpoint{2.244437in}{2.056859in}}%
\pgfpathlineto{\pgfqpoint{2.176926in}{2.056859in}}%
\pgfpathlineto{\pgfqpoint{2.176926in}{0.778861in}}%
\pgfpathclose%
\pgfusepath{fill}%
\end{pgfscope}%
\begin{pgfscope}%
\pgfpathrectangle{\pgfqpoint{0.379436in}{0.778861in}}{\pgfqpoint{2.025341in}{1.681851in}}%
\pgfusepath{clip}%
\pgfsetbuttcap%
\pgfsetmiterjoin%
\definecolor{currentfill}{rgb}{0.858824,0.314706,0.223529}%
\pgfsetfillcolor{currentfill}%
\pgfsetlinewidth{0.000000pt}%
\definecolor{currentstroke}{rgb}{0.000000,0.000000,0.000000}%
\pgfsetstrokecolor{currentstroke}%
\pgfsetstrokeopacity{0.000000}%
\pgfsetdash{}{0pt}%
\pgfpathmoveto{\pgfqpoint{0.472264in}{0.778861in}}%
\pgfpathlineto{\pgfqpoint{0.539775in}{0.778861in}}%
\pgfpathlineto{\pgfqpoint{0.539775in}{1.982233in}}%
\pgfpathlineto{\pgfqpoint{0.472264in}{1.982233in}}%
\pgfpathlineto{\pgfqpoint{0.472264in}{0.778861in}}%
\pgfpathclose%
\pgfusepath{fill}%
\end{pgfscope}%
\begin{pgfscope}%
\pgfpathrectangle{\pgfqpoint{0.379436in}{0.778861in}}{\pgfqpoint{2.025341in}{1.681851in}}%
\pgfusepath{clip}%
\pgfsetbuttcap%
\pgfsetmiterjoin%
\definecolor{currentfill}{rgb}{0.858824,0.314706,0.223529}%
\pgfsetfillcolor{currentfill}%
\pgfsetlinewidth{0.000000pt}%
\definecolor{currentstroke}{rgb}{0.000000,0.000000,0.000000}%
\pgfsetstrokecolor{currentstroke}%
\pgfsetstrokeopacity{0.000000}%
\pgfsetdash{}{0pt}%
\pgfpathmoveto{\pgfqpoint{0.725431in}{0.778861in}}%
\pgfpathlineto{\pgfqpoint{0.792943in}{0.778861in}}%
\pgfpathlineto{\pgfqpoint{0.792943in}{2.283928in}}%
\pgfpathlineto{\pgfqpoint{0.725431in}{2.283928in}}%
\pgfpathlineto{\pgfqpoint{0.725431in}{0.778861in}}%
\pgfpathclose%
\pgfusepath{fill}%
\end{pgfscope}%
\begin{pgfscope}%
\pgfpathrectangle{\pgfqpoint{0.379436in}{0.778861in}}{\pgfqpoint{2.025341in}{1.681851in}}%
\pgfusepath{clip}%
\pgfsetbuttcap%
\pgfsetmiterjoin%
\definecolor{currentfill}{rgb}{0.858824,0.314706,0.223529}%
\pgfsetfillcolor{currentfill}%
\pgfsetlinewidth{0.000000pt}%
\definecolor{currentstroke}{rgb}{0.000000,0.000000,0.000000}%
\pgfsetstrokecolor{currentstroke}%
\pgfsetstrokeopacity{0.000000}%
\pgfsetdash{}{0pt}%
\pgfpathmoveto{\pgfqpoint{0.978599in}{0.778861in}}%
\pgfpathlineto{\pgfqpoint{1.046110in}{0.778861in}}%
\pgfpathlineto{\pgfqpoint{1.046110in}{2.170722in}}%
\pgfpathlineto{\pgfqpoint{0.978599in}{2.170722in}}%
\pgfpathlineto{\pgfqpoint{0.978599in}{0.778861in}}%
\pgfpathclose%
\pgfusepath{fill}%
\end{pgfscope}%
\begin{pgfscope}%
\pgfpathrectangle{\pgfqpoint{0.379436in}{0.778861in}}{\pgfqpoint{2.025341in}{1.681851in}}%
\pgfusepath{clip}%
\pgfsetbuttcap%
\pgfsetmiterjoin%
\definecolor{currentfill}{rgb}{0.858824,0.314706,0.223529}%
\pgfsetfillcolor{currentfill}%
\pgfsetlinewidth{0.000000pt}%
\definecolor{currentstroke}{rgb}{0.000000,0.000000,0.000000}%
\pgfsetstrokecolor{currentstroke}%
\pgfsetstrokeopacity{0.000000}%
\pgfsetdash{}{0pt}%
\pgfpathmoveto{\pgfqpoint{1.231767in}{0.778861in}}%
\pgfpathlineto{\pgfqpoint{1.299278in}{0.778861in}}%
\pgfpathlineto{\pgfqpoint{1.299278in}{2.133317in}}%
\pgfpathlineto{\pgfqpoint{1.231767in}{2.133317in}}%
\pgfpathlineto{\pgfqpoint{1.231767in}{0.778861in}}%
\pgfpathclose%
\pgfusepath{fill}%
\end{pgfscope}%
\begin{pgfscope}%
\pgfpathrectangle{\pgfqpoint{0.379436in}{0.778861in}}{\pgfqpoint{2.025341in}{1.681851in}}%
\pgfusepath{clip}%
\pgfsetbuttcap%
\pgfsetmiterjoin%
\definecolor{currentfill}{rgb}{0.858824,0.314706,0.223529}%
\pgfsetfillcolor{currentfill}%
\pgfsetlinewidth{0.000000pt}%
\definecolor{currentstroke}{rgb}{0.000000,0.000000,0.000000}%
\pgfsetstrokecolor{currentstroke}%
\pgfsetstrokeopacity{0.000000}%
\pgfsetdash{}{0pt}%
\pgfpathmoveto{\pgfqpoint{1.484934in}{0.778861in}}%
\pgfpathlineto{\pgfqpoint{1.552446in}{0.778861in}}%
\pgfpathlineto{\pgfqpoint{1.552446in}{1.935470in}}%
\pgfpathlineto{\pgfqpoint{1.484934in}{1.935470in}}%
\pgfpathlineto{\pgfqpoint{1.484934in}{0.778861in}}%
\pgfpathclose%
\pgfusepath{fill}%
\end{pgfscope}%
\begin{pgfscope}%
\pgfpathrectangle{\pgfqpoint{0.379436in}{0.778861in}}{\pgfqpoint{2.025341in}{1.681851in}}%
\pgfusepath{clip}%
\pgfsetbuttcap%
\pgfsetmiterjoin%
\definecolor{currentfill}{rgb}{0.858824,0.314706,0.223529}%
\pgfsetfillcolor{currentfill}%
\pgfsetlinewidth{0.000000pt}%
\definecolor{currentstroke}{rgb}{0.000000,0.000000,0.000000}%
\pgfsetstrokecolor{currentstroke}%
\pgfsetstrokeopacity{0.000000}%
\pgfsetdash{}{0pt}%
\pgfpathmoveto{\pgfqpoint{1.738102in}{0.778861in}}%
\pgfpathlineto{\pgfqpoint{1.805613in}{0.778861in}}%
\pgfpathlineto{\pgfqpoint{1.805613in}{1.961384in}}%
\pgfpathlineto{\pgfqpoint{1.738102in}{1.961384in}}%
\pgfpathlineto{\pgfqpoint{1.738102in}{0.778861in}}%
\pgfpathclose%
\pgfusepath{fill}%
\end{pgfscope}%
\begin{pgfscope}%
\pgfpathrectangle{\pgfqpoint{0.379436in}{0.778861in}}{\pgfqpoint{2.025341in}{1.681851in}}%
\pgfusepath{clip}%
\pgfsetbuttcap%
\pgfsetmiterjoin%
\definecolor{currentfill}{rgb}{0.858824,0.314706,0.223529}%
\pgfsetfillcolor{currentfill}%
\pgfsetlinewidth{0.000000pt}%
\definecolor{currentstroke}{rgb}{0.000000,0.000000,0.000000}%
\pgfsetstrokecolor{currentstroke}%
\pgfsetstrokeopacity{0.000000}%
\pgfsetdash{}{0pt}%
\pgfpathmoveto{\pgfqpoint{1.991270in}{0.778861in}}%
\pgfpathlineto{\pgfqpoint{2.058781in}{0.778861in}}%
\pgfpathlineto{\pgfqpoint{2.058781in}{1.882998in}}%
\pgfpathlineto{\pgfqpoint{1.991270in}{1.882998in}}%
\pgfpathlineto{\pgfqpoint{1.991270in}{0.778861in}}%
\pgfpathclose%
\pgfusepath{fill}%
\end{pgfscope}%
\begin{pgfscope}%
\pgfpathrectangle{\pgfqpoint{0.379436in}{0.778861in}}{\pgfqpoint{2.025341in}{1.681851in}}%
\pgfusepath{clip}%
\pgfsetbuttcap%
\pgfsetmiterjoin%
\definecolor{currentfill}{rgb}{0.858824,0.314706,0.223529}%
\pgfsetfillcolor{currentfill}%
\pgfsetlinewidth{0.000000pt}%
\definecolor{currentstroke}{rgb}{0.000000,0.000000,0.000000}%
\pgfsetstrokecolor{currentstroke}%
\pgfsetstrokeopacity{0.000000}%
\pgfsetdash{}{0pt}%
\pgfpathmoveto{\pgfqpoint{2.244437in}{0.778861in}}%
\pgfpathlineto{\pgfqpoint{2.311949in}{0.778861in}}%
\pgfpathlineto{\pgfqpoint{2.311949in}{2.050008in}}%
\pgfpathlineto{\pgfqpoint{2.244437in}{2.050008in}}%
\pgfpathlineto{\pgfqpoint{2.244437in}{0.778861in}}%
\pgfpathclose%
\pgfusepath{fill}%
\end{pgfscope}%
\begin{pgfscope}%
\pgfpathrectangle{\pgfqpoint{0.379436in}{0.778861in}}{\pgfqpoint{2.025341in}{1.681851in}}%
\pgfusepath{clip}%
\pgfsetbuttcap%
\pgfsetmiterjoin%
\definecolor{currentfill}{rgb}{0.464706,0.320588,0.573529}%
\pgfsetfillcolor{currentfill}%
\pgfsetlinewidth{0.000000pt}%
\definecolor{currentstroke}{rgb}{0.000000,0.000000,0.000000}%
\pgfsetstrokecolor{currentstroke}%
\pgfsetstrokeopacity{0.000000}%
\pgfsetdash{}{0pt}%
\pgfpathmoveto{\pgfqpoint{0.539775in}{0.778861in}}%
\pgfpathlineto{\pgfqpoint{0.607287in}{0.778861in}}%
\pgfpathlineto{\pgfqpoint{0.607287in}{1.920767in}}%
\pgfpathlineto{\pgfqpoint{0.539775in}{1.920767in}}%
\pgfpathlineto{\pgfqpoint{0.539775in}{0.778861in}}%
\pgfpathclose%
\pgfusepath{fill}%
\end{pgfscope}%
\begin{pgfscope}%
\pgfpathrectangle{\pgfqpoint{0.379436in}{0.778861in}}{\pgfqpoint{2.025341in}{1.681851in}}%
\pgfusepath{clip}%
\pgfsetbuttcap%
\pgfsetmiterjoin%
\definecolor{currentfill}{rgb}{0.464706,0.320588,0.573529}%
\pgfsetfillcolor{currentfill}%
\pgfsetlinewidth{0.000000pt}%
\definecolor{currentstroke}{rgb}{0.000000,0.000000,0.000000}%
\pgfsetstrokecolor{currentstroke}%
\pgfsetstrokeopacity{0.000000}%
\pgfsetdash{}{0pt}%
\pgfpathmoveto{\pgfqpoint{0.792943in}{0.778861in}}%
\pgfpathlineto{\pgfqpoint{0.860454in}{0.778861in}}%
\pgfpathlineto{\pgfqpoint{0.860454in}{2.285304in}}%
\pgfpathlineto{\pgfqpoint{0.792943in}{2.285304in}}%
\pgfpathlineto{\pgfqpoint{0.792943in}{0.778861in}}%
\pgfpathclose%
\pgfusepath{fill}%
\end{pgfscope}%
\begin{pgfscope}%
\pgfpathrectangle{\pgfqpoint{0.379436in}{0.778861in}}{\pgfqpoint{2.025341in}{1.681851in}}%
\pgfusepath{clip}%
\pgfsetbuttcap%
\pgfsetmiterjoin%
\definecolor{currentfill}{rgb}{0.464706,0.320588,0.573529}%
\pgfsetfillcolor{currentfill}%
\pgfsetlinewidth{0.000000pt}%
\definecolor{currentstroke}{rgb}{0.000000,0.000000,0.000000}%
\pgfsetstrokecolor{currentstroke}%
\pgfsetstrokeopacity{0.000000}%
\pgfsetdash{}{0pt}%
\pgfpathmoveto{\pgfqpoint{1.046110in}{0.778861in}}%
\pgfpathlineto{\pgfqpoint{1.113622in}{0.778861in}}%
\pgfpathlineto{\pgfqpoint{1.113622in}{2.136860in}}%
\pgfpathlineto{\pgfqpoint{1.046110in}{2.136860in}}%
\pgfpathlineto{\pgfqpoint{1.046110in}{0.778861in}}%
\pgfpathclose%
\pgfusepath{fill}%
\end{pgfscope}%
\begin{pgfscope}%
\pgfpathrectangle{\pgfqpoint{0.379436in}{0.778861in}}{\pgfqpoint{2.025341in}{1.681851in}}%
\pgfusepath{clip}%
\pgfsetbuttcap%
\pgfsetmiterjoin%
\definecolor{currentfill}{rgb}{0.464706,0.320588,0.573529}%
\pgfsetfillcolor{currentfill}%
\pgfsetlinewidth{0.000000pt}%
\definecolor{currentstroke}{rgb}{0.000000,0.000000,0.000000}%
\pgfsetstrokecolor{currentstroke}%
\pgfsetstrokeopacity{0.000000}%
\pgfsetdash{}{0pt}%
\pgfpathmoveto{\pgfqpoint{1.299278in}{0.778861in}}%
\pgfpathlineto{\pgfqpoint{1.366790in}{0.778861in}}%
\pgfpathlineto{\pgfqpoint{1.366790in}{2.171050in}}%
\pgfpathlineto{\pgfqpoint{1.299278in}{2.171050in}}%
\pgfpathlineto{\pgfqpoint{1.299278in}{0.778861in}}%
\pgfpathclose%
\pgfusepath{fill}%
\end{pgfscope}%
\begin{pgfscope}%
\pgfpathrectangle{\pgfqpoint{0.379436in}{0.778861in}}{\pgfqpoint{2.025341in}{1.681851in}}%
\pgfusepath{clip}%
\pgfsetbuttcap%
\pgfsetmiterjoin%
\definecolor{currentfill}{rgb}{0.464706,0.320588,0.573529}%
\pgfsetfillcolor{currentfill}%
\pgfsetlinewidth{0.000000pt}%
\definecolor{currentstroke}{rgb}{0.000000,0.000000,0.000000}%
\pgfsetstrokecolor{currentstroke}%
\pgfsetstrokeopacity{0.000000}%
\pgfsetdash{}{0pt}%
\pgfpathmoveto{\pgfqpoint{1.552446in}{0.778861in}}%
\pgfpathlineto{\pgfqpoint{1.619957in}{0.778861in}}%
\pgfpathlineto{\pgfqpoint{1.619957in}{1.977682in}}%
\pgfpathlineto{\pgfqpoint{1.552446in}{1.977682in}}%
\pgfpathlineto{\pgfqpoint{1.552446in}{0.778861in}}%
\pgfpathclose%
\pgfusepath{fill}%
\end{pgfscope}%
\begin{pgfscope}%
\pgfpathrectangle{\pgfqpoint{0.379436in}{0.778861in}}{\pgfqpoint{2.025341in}{1.681851in}}%
\pgfusepath{clip}%
\pgfsetbuttcap%
\pgfsetmiterjoin%
\definecolor{currentfill}{rgb}{0.464706,0.320588,0.573529}%
\pgfsetfillcolor{currentfill}%
\pgfsetlinewidth{0.000000pt}%
\definecolor{currentstroke}{rgb}{0.000000,0.000000,0.000000}%
\pgfsetstrokecolor{currentstroke}%
\pgfsetstrokeopacity{0.000000}%
\pgfsetdash{}{0pt}%
\pgfpathmoveto{\pgfqpoint{1.805613in}{0.778861in}}%
\pgfpathlineto{\pgfqpoint{1.873125in}{0.778861in}}%
\pgfpathlineto{\pgfqpoint{1.873125in}{1.878189in}}%
\pgfpathlineto{\pgfqpoint{1.805613in}{1.878189in}}%
\pgfpathlineto{\pgfqpoint{1.805613in}{0.778861in}}%
\pgfpathclose%
\pgfusepath{fill}%
\end{pgfscope}%
\begin{pgfscope}%
\pgfpathrectangle{\pgfqpoint{0.379436in}{0.778861in}}{\pgfqpoint{2.025341in}{1.681851in}}%
\pgfusepath{clip}%
\pgfsetbuttcap%
\pgfsetmiterjoin%
\definecolor{currentfill}{rgb}{0.464706,0.320588,0.573529}%
\pgfsetfillcolor{currentfill}%
\pgfsetlinewidth{0.000000pt}%
\definecolor{currentstroke}{rgb}{0.000000,0.000000,0.000000}%
\pgfsetstrokecolor{currentstroke}%
\pgfsetstrokeopacity{0.000000}%
\pgfsetdash{}{0pt}%
\pgfpathmoveto{\pgfqpoint{2.058781in}{0.778861in}}%
\pgfpathlineto{\pgfqpoint{2.126292in}{0.778861in}}%
\pgfpathlineto{\pgfqpoint{2.126292in}{1.821868in}}%
\pgfpathlineto{\pgfqpoint{2.058781in}{1.821868in}}%
\pgfpathlineto{\pgfqpoint{2.058781in}{0.778861in}}%
\pgfpathclose%
\pgfusepath{fill}%
\end{pgfscope}%
\begin{pgfscope}%
\pgfpathrectangle{\pgfqpoint{0.379436in}{0.778861in}}{\pgfqpoint{2.025341in}{1.681851in}}%
\pgfusepath{clip}%
\pgfsetbuttcap%
\pgfsetmiterjoin%
\definecolor{currentfill}{rgb}{0.464706,0.320588,0.573529}%
\pgfsetfillcolor{currentfill}%
\pgfsetlinewidth{0.000000pt}%
\definecolor{currentstroke}{rgb}{0.000000,0.000000,0.000000}%
\pgfsetstrokecolor{currentstroke}%
\pgfsetstrokeopacity{0.000000}%
\pgfsetdash{}{0pt}%
\pgfpathmoveto{\pgfqpoint{2.311949in}{0.778861in}}%
\pgfpathlineto{\pgfqpoint{2.379460in}{0.778861in}}%
\pgfpathlineto{\pgfqpoint{2.379460in}{2.027389in}}%
\pgfpathlineto{\pgfqpoint{2.311949in}{2.027389in}}%
\pgfpathlineto{\pgfqpoint{2.311949in}{0.778861in}}%
\pgfpathclose%
\pgfusepath{fill}%
\end{pgfscope}%
\begin{pgfscope}%
\pgfsetbuttcap%
\pgfsetroundjoin%
\definecolor{currentfill}{rgb}{0.552941,0.501961,0.478431}%
\pgfsetfillcolor{currentfill}%
\pgfsetlinewidth{0.803000pt}%
\definecolor{currentstroke}{rgb}{0.552941,0.501961,0.478431}%
\pgfsetstrokecolor{currentstroke}%
\pgfsetdash{}{0pt}%
\pgfsys@defobject{currentmarker}{\pgfqpoint{0.000000in}{0.000000in}}{\pgfqpoint{0.000000in}{0.083333in}}{%
\pgfpathmoveto{\pgfqpoint{0.000000in}{0.000000in}}%
\pgfpathlineto{\pgfqpoint{0.000000in}{0.083333in}}%
\pgfusepath{stroke,fill}%
}%
\begin{pgfscope}%
\pgfsys@transformshift{0.506020in}{0.778861in}%
\pgfsys@useobject{currentmarker}{}%
\end{pgfscope}%
\end{pgfscope}%
\begin{pgfscope}%
\definecolor{textcolor}{rgb}{0.552941,0.501961,0.478431}%
\pgfsetstrokecolor{textcolor}%
\pgfsetfillcolor{textcolor}%
\pgftext[x=0.537270in, y=0.444468in, left, base,rotate=90.000000]{\color{textcolor}{\sffamily\fontsize{9.000000}{10.800000}\selectfont\catcode`\^=\active\def^{\ifmmode\sp\else\^{}\fi}\catcode`\%=\active\def%{\%}PA01}}%
\end{pgfscope}%
\begin{pgfscope}%
\pgfsetbuttcap%
\pgfsetroundjoin%
\definecolor{currentfill}{rgb}{0.552941,0.501961,0.478431}%
\pgfsetfillcolor{currentfill}%
\pgfsetlinewidth{0.803000pt}%
\definecolor{currentstroke}{rgb}{0.552941,0.501961,0.478431}%
\pgfsetstrokecolor{currentstroke}%
\pgfsetdash{}{0pt}%
\pgfsys@defobject{currentmarker}{\pgfqpoint{0.000000in}{0.000000in}}{\pgfqpoint{0.000000in}{0.083333in}}{%
\pgfpathmoveto{\pgfqpoint{0.000000in}{0.000000in}}%
\pgfpathlineto{\pgfqpoint{0.000000in}{0.083333in}}%
\pgfusepath{stroke,fill}%
}%
\begin{pgfscope}%
\pgfsys@transformshift{0.759187in}{0.778861in}%
\pgfsys@useobject{currentmarker}{}%
\end{pgfscope}%
\end{pgfscope}%
\begin{pgfscope}%
\definecolor{textcolor}{rgb}{0.552941,0.501961,0.478431}%
\pgfsetstrokecolor{textcolor}%
\pgfsetfillcolor{textcolor}%
\pgftext[x=0.790437in, y=0.433859in, left, base,rotate=90.000000]{\color{textcolor}{\sffamily\fontsize{9.000000}{10.800000}\selectfont\catcode`\^=\active\def^{\ifmmode\sp\else\^{}\fi}\catcode`\%=\active\def%{\%}PB01}}%
\end{pgfscope}%
\begin{pgfscope}%
\pgfsetbuttcap%
\pgfsetroundjoin%
\definecolor{currentfill}{rgb}{0.552941,0.501961,0.478431}%
\pgfsetfillcolor{currentfill}%
\pgfsetlinewidth{0.803000pt}%
\definecolor{currentstroke}{rgb}{0.552941,0.501961,0.478431}%
\pgfsetstrokecolor{currentstroke}%
\pgfsetdash{}{0pt}%
\pgfsys@defobject{currentmarker}{\pgfqpoint{0.000000in}{0.000000in}}{\pgfqpoint{0.000000in}{0.083333in}}{%
\pgfpathmoveto{\pgfqpoint{0.000000in}{0.000000in}}%
\pgfpathlineto{\pgfqpoint{0.000000in}{0.083333in}}%
\pgfusepath{stroke,fill}%
}%
\begin{pgfscope}%
\pgfsys@transformshift{1.012355in}{0.778861in}%
\pgfsys@useobject{currentmarker}{}%
\end{pgfscope}%
\end{pgfscope}%
\begin{pgfscope}%
\definecolor{textcolor}{rgb}{0.552941,0.501961,0.478431}%
\pgfsetstrokecolor{textcolor}%
\pgfsetfillcolor{textcolor}%
\pgftext[x=1.043605in, y=0.433859in, left, base,rotate=90.000000]{\color{textcolor}{\sffamily\fontsize{9.000000}{10.800000}\selectfont\catcode`\^=\active\def^{\ifmmode\sp\else\^{}\fi}\catcode`\%=\active\def%{\%}PB02}}%
\end{pgfscope}%
\begin{pgfscope}%
\pgfsetbuttcap%
\pgfsetroundjoin%
\definecolor{currentfill}{rgb}{0.552941,0.501961,0.478431}%
\pgfsetfillcolor{currentfill}%
\pgfsetlinewidth{0.803000pt}%
\definecolor{currentstroke}{rgb}{0.552941,0.501961,0.478431}%
\pgfsetstrokecolor{currentstroke}%
\pgfsetdash{}{0pt}%
\pgfsys@defobject{currentmarker}{\pgfqpoint{0.000000in}{0.000000in}}{\pgfqpoint{0.000000in}{0.083333in}}{%
\pgfpathmoveto{\pgfqpoint{0.000000in}{0.000000in}}%
\pgfpathlineto{\pgfqpoint{0.000000in}{0.083333in}}%
\pgfusepath{stroke,fill}%
}%
\begin{pgfscope}%
\pgfsys@transformshift{1.265522in}{0.778861in}%
\pgfsys@useobject{currentmarker}{}%
\end{pgfscope}%
\end{pgfscope}%
\begin{pgfscope}%
\definecolor{textcolor}{rgb}{0.552941,0.501961,0.478431}%
\pgfsetstrokecolor{textcolor}%
\pgfsetfillcolor{textcolor}%
\pgftext[x=1.296772in, y=0.433859in, left, base,rotate=90.000000]{\color{textcolor}{\sffamily\fontsize{9.000000}{10.800000}\selectfont\catcode`\^=\active\def^{\ifmmode\sp\else\^{}\fi}\catcode`\%=\active\def%{\%}PB04}}%
\end{pgfscope}%
\begin{pgfscope}%
\pgfsetbuttcap%
\pgfsetroundjoin%
\definecolor{currentfill}{rgb}{0.552941,0.501961,0.478431}%
\pgfsetfillcolor{currentfill}%
\pgfsetlinewidth{0.803000pt}%
\definecolor{currentstroke}{rgb}{0.552941,0.501961,0.478431}%
\pgfsetstrokecolor{currentstroke}%
\pgfsetdash{}{0pt}%
\pgfsys@defobject{currentmarker}{\pgfqpoint{0.000000in}{0.000000in}}{\pgfqpoint{0.000000in}{0.083333in}}{%
\pgfpathmoveto{\pgfqpoint{0.000000in}{0.000000in}}%
\pgfpathlineto{\pgfqpoint{0.000000in}{0.083333in}}%
\pgfusepath{stroke,fill}%
}%
\begin{pgfscope}%
\pgfsys@transformshift{1.518690in}{0.778861in}%
\pgfsys@useobject{currentmarker}{}%
\end{pgfscope}%
\end{pgfscope}%
\begin{pgfscope}%
\definecolor{textcolor}{rgb}{0.552941,0.501961,0.478431}%
\pgfsetstrokecolor{textcolor}%
\pgfsetfillcolor{textcolor}%
\pgftext[x=1.549940in, y=0.437524in, left, base,rotate=90.000000]{\color{textcolor}{\sffamily\fontsize{9.000000}{10.800000}\selectfont\catcode`\^=\active\def^{\ifmmode\sp\else\^{}\fi}\catcode`\%=\active\def%{\%}PC02}}%
\end{pgfscope}%
\begin{pgfscope}%
\pgfsetbuttcap%
\pgfsetroundjoin%
\definecolor{currentfill}{rgb}{0.552941,0.501961,0.478431}%
\pgfsetfillcolor{currentfill}%
\pgfsetlinewidth{0.803000pt}%
\definecolor{currentstroke}{rgb}{0.552941,0.501961,0.478431}%
\pgfsetstrokecolor{currentstroke}%
\pgfsetdash{}{0pt}%
\pgfsys@defobject{currentmarker}{\pgfqpoint{0.000000in}{0.000000in}}{\pgfqpoint{0.000000in}{0.083333in}}{%
\pgfpathmoveto{\pgfqpoint{0.000000in}{0.000000in}}%
\pgfpathlineto{\pgfqpoint{0.000000in}{0.083333in}}%
\pgfusepath{stroke,fill}%
}%
\begin{pgfscope}%
\pgfsys@transformshift{1.771858in}{0.778861in}%
\pgfsys@useobject{currentmarker}{}%
\end{pgfscope}%
\end{pgfscope}%
\begin{pgfscope}%
\definecolor{textcolor}{rgb}{0.552941,0.501961,0.478431}%
\pgfsetstrokecolor{textcolor}%
\pgfsetfillcolor{textcolor}%
\pgftext[x=1.803108in, y=0.437524in, left, base,rotate=90.000000]{\color{textcolor}{\sffamily\fontsize{9.000000}{10.800000}\selectfont\catcode`\^=\active\def^{\ifmmode\sp\else\^{}\fi}\catcode`\%=\active\def%{\%}PC03}}%
\end{pgfscope}%
\begin{pgfscope}%
\pgfsetbuttcap%
\pgfsetroundjoin%
\definecolor{currentfill}{rgb}{0.552941,0.501961,0.478431}%
\pgfsetfillcolor{currentfill}%
\pgfsetlinewidth{0.803000pt}%
\definecolor{currentstroke}{rgb}{0.552941,0.501961,0.478431}%
\pgfsetstrokecolor{currentstroke}%
\pgfsetdash{}{0pt}%
\pgfsys@defobject{currentmarker}{\pgfqpoint{0.000000in}{0.000000in}}{\pgfqpoint{0.000000in}{0.083333in}}{%
\pgfpathmoveto{\pgfqpoint{0.000000in}{0.000000in}}%
\pgfpathlineto{\pgfqpoint{0.000000in}{0.083333in}}%
\pgfusepath{stroke,fill}%
}%
\begin{pgfscope}%
\pgfsys@transformshift{2.025025in}{0.778861in}%
\pgfsys@useobject{currentmarker}{}%
\end{pgfscope}%
\end{pgfscope}%
\begin{pgfscope}%
\definecolor{textcolor}{rgb}{0.552941,0.501961,0.478431}%
\pgfsetstrokecolor{textcolor}%
\pgfsetfillcolor{textcolor}%
\pgftext[x=2.056275in, y=0.437524in, left, base,rotate=90.000000]{\color{textcolor}{\sffamily\fontsize{9.000000}{10.800000}\selectfont\catcode`\^=\active\def^{\ifmmode\sp\else\^{}\fi}\catcode`\%=\active\def%{\%}PC04}}%
\end{pgfscope}%
\begin{pgfscope}%
\pgfsetbuttcap%
\pgfsetroundjoin%
\definecolor{currentfill}{rgb}{0.552941,0.501961,0.478431}%
\pgfsetfillcolor{currentfill}%
\pgfsetlinewidth{0.803000pt}%
\definecolor{currentstroke}{rgb}{0.552941,0.501961,0.478431}%
\pgfsetstrokecolor{currentstroke}%
\pgfsetdash{}{0pt}%
\pgfsys@defobject{currentmarker}{\pgfqpoint{0.000000in}{0.000000in}}{\pgfqpoint{0.000000in}{0.083333in}}{%
\pgfpathmoveto{\pgfqpoint{0.000000in}{0.000000in}}%
\pgfpathlineto{\pgfqpoint{0.000000in}{0.083333in}}%
\pgfusepath{stroke,fill}%
}%
\begin{pgfscope}%
\pgfsys@transformshift{2.278193in}{0.778861in}%
\pgfsys@useobject{currentmarker}{}%
\end{pgfscope}%
\end{pgfscope}%
\begin{pgfscope}%
\definecolor{textcolor}{rgb}{0.552941,0.501961,0.478431}%
\pgfsetstrokecolor{textcolor}%
\pgfsetfillcolor{textcolor}%
\pgftext[x=2.309443in, y=0.509380in, left, base,rotate=90.000000]{\color{textcolor}{\sffamily\fontsize{9.000000}{10.800000}\selectfont\catcode`\^=\active\def^{\ifmmode\sp\else\^{}\fi}\catcode`\%=\active\def%{\%}avg.}}%
\end{pgfscope}%
\begin{pgfscope}%
\definecolor{textcolor}{rgb}{0.552941,0.501961,0.478431}%
\pgfsetstrokecolor{textcolor}%
\pgfsetfillcolor{textcolor}%
\pgftext[x=1.392106in,y=0.378303in,,top]{\color{textcolor}{\sffamily\fontsize{9.000000}{10.800000}\selectfont\catcode`\^=\active\def^{\ifmmode\sp\else\^{}\fi}\catcode`\%=\active\def%{\%}patient}}%
\end{pgfscope}%
\begin{pgfscope}%
\pgfsetbuttcap%
\pgfsetroundjoin%
\definecolor{currentfill}{rgb}{0.552941,0.501961,0.478431}%
\pgfsetfillcolor{currentfill}%
\pgfsetlinewidth{0.803000pt}%
\definecolor{currentstroke}{rgb}{0.552941,0.501961,0.478431}%
\pgfsetstrokecolor{currentstroke}%
\pgfsetdash{}{0pt}%
\pgfsys@defobject{currentmarker}{\pgfqpoint{0.000000in}{0.000000in}}{\pgfqpoint{0.083333in}{0.000000in}}{%
\pgfpathmoveto{\pgfqpoint{0.000000in}{0.000000in}}%
\pgfpathlineto{\pgfqpoint{0.083333in}{0.000000in}}%
\pgfusepath{stroke,fill}%
}%
\begin{pgfscope}%
\pgfsys@transformshift{0.379436in}{0.778861in}%
\pgfsys@useobject{currentmarker}{}%
\end{pgfscope}%
\end{pgfscope}%
\begin{pgfscope}%
\definecolor{textcolor}{rgb}{0.552941,0.501961,0.478431}%
\pgfsetstrokecolor{textcolor}%
\pgfsetfillcolor{textcolor}%
\pgftext[x=0.166667in, y=0.735458in, left, base]{\color{textcolor}{\sffamily\fontsize{9.000000}{10.800000}\selectfont\catcode`\^=\active\def^{\ifmmode\sp\else\^{}\fi}\catcode`\%=\active\def%{\%}$\mathdefault{0.0}$}}%
\end{pgfscope}%
\begin{pgfscope}%
\pgfsetbuttcap%
\pgfsetroundjoin%
\definecolor{currentfill}{rgb}{0.552941,0.501961,0.478431}%
\pgfsetfillcolor{currentfill}%
\pgfsetlinewidth{0.803000pt}%
\definecolor{currentstroke}{rgb}{0.552941,0.501961,0.478431}%
\pgfsetstrokecolor{currentstroke}%
\pgfsetdash{}{0pt}%
\pgfsys@defobject{currentmarker}{\pgfqpoint{0.000000in}{0.000000in}}{\pgfqpoint{0.083333in}{0.000000in}}{%
\pgfpathmoveto{\pgfqpoint{0.000000in}{0.000000in}}%
\pgfpathlineto{\pgfqpoint{0.083333in}{0.000000in}}%
\pgfusepath{stroke,fill}%
}%
\begin{pgfscope}%
\pgfsys@transformshift{0.379436in}{1.115231in}%
\pgfsys@useobject{currentmarker}{}%
\end{pgfscope}%
\end{pgfscope}%
\begin{pgfscope}%
\definecolor{textcolor}{rgb}{0.552941,0.501961,0.478431}%
\pgfsetstrokecolor{textcolor}%
\pgfsetfillcolor{textcolor}%
\pgftext[x=0.166667in, y=1.071828in, left, base]{\color{textcolor}{\sffamily\fontsize{9.000000}{10.800000}\selectfont\catcode`\^=\active\def^{\ifmmode\sp\else\^{}\fi}\catcode`\%=\active\def%{\%}$\mathdefault{0.2}$}}%
\end{pgfscope}%
\begin{pgfscope}%
\pgfsetbuttcap%
\pgfsetroundjoin%
\definecolor{currentfill}{rgb}{0.552941,0.501961,0.478431}%
\pgfsetfillcolor{currentfill}%
\pgfsetlinewidth{0.803000pt}%
\definecolor{currentstroke}{rgb}{0.552941,0.501961,0.478431}%
\pgfsetstrokecolor{currentstroke}%
\pgfsetdash{}{0pt}%
\pgfsys@defobject{currentmarker}{\pgfqpoint{0.000000in}{0.000000in}}{\pgfqpoint{0.083333in}{0.000000in}}{%
\pgfpathmoveto{\pgfqpoint{0.000000in}{0.000000in}}%
\pgfpathlineto{\pgfqpoint{0.083333in}{0.000000in}}%
\pgfusepath{stroke,fill}%
}%
\begin{pgfscope}%
\pgfsys@transformshift{0.379436in}{1.451601in}%
\pgfsys@useobject{currentmarker}{}%
\end{pgfscope}%
\end{pgfscope}%
\begin{pgfscope}%
\definecolor{textcolor}{rgb}{0.552941,0.501961,0.478431}%
\pgfsetstrokecolor{textcolor}%
\pgfsetfillcolor{textcolor}%
\pgftext[x=0.166667in, y=1.408198in, left, base]{\color{textcolor}{\sffamily\fontsize{9.000000}{10.800000}\selectfont\catcode`\^=\active\def^{\ifmmode\sp\else\^{}\fi}\catcode`\%=\active\def%{\%}$\mathdefault{0.4}$}}%
\end{pgfscope}%
\begin{pgfscope}%
\pgfsetbuttcap%
\pgfsetroundjoin%
\definecolor{currentfill}{rgb}{0.552941,0.501961,0.478431}%
\pgfsetfillcolor{currentfill}%
\pgfsetlinewidth{0.803000pt}%
\definecolor{currentstroke}{rgb}{0.552941,0.501961,0.478431}%
\pgfsetstrokecolor{currentstroke}%
\pgfsetdash{}{0pt}%
\pgfsys@defobject{currentmarker}{\pgfqpoint{0.000000in}{0.000000in}}{\pgfqpoint{0.083333in}{0.000000in}}{%
\pgfpathmoveto{\pgfqpoint{0.000000in}{0.000000in}}%
\pgfpathlineto{\pgfqpoint{0.083333in}{0.000000in}}%
\pgfusepath{stroke,fill}%
}%
\begin{pgfscope}%
\pgfsys@transformshift{0.379436in}{1.787971in}%
\pgfsys@useobject{currentmarker}{}%
\end{pgfscope}%
\end{pgfscope}%
\begin{pgfscope}%
\definecolor{textcolor}{rgb}{0.552941,0.501961,0.478431}%
\pgfsetstrokecolor{textcolor}%
\pgfsetfillcolor{textcolor}%
\pgftext[x=0.166667in, y=1.744568in, left, base]{\color{textcolor}{\sffamily\fontsize{9.000000}{10.800000}\selectfont\catcode`\^=\active\def^{\ifmmode\sp\else\^{}\fi}\catcode`\%=\active\def%{\%}$\mathdefault{0.6}$}}%
\end{pgfscope}%
\begin{pgfscope}%
\pgfsetbuttcap%
\pgfsetroundjoin%
\definecolor{currentfill}{rgb}{0.552941,0.501961,0.478431}%
\pgfsetfillcolor{currentfill}%
\pgfsetlinewidth{0.803000pt}%
\definecolor{currentstroke}{rgb}{0.552941,0.501961,0.478431}%
\pgfsetstrokecolor{currentstroke}%
\pgfsetdash{}{0pt}%
\pgfsys@defobject{currentmarker}{\pgfqpoint{0.000000in}{0.000000in}}{\pgfqpoint{0.083333in}{0.000000in}}{%
\pgfpathmoveto{\pgfqpoint{0.000000in}{0.000000in}}%
\pgfpathlineto{\pgfqpoint{0.083333in}{0.000000in}}%
\pgfusepath{stroke,fill}%
}%
\begin{pgfscope}%
\pgfsys@transformshift{0.379436in}{2.124341in}%
\pgfsys@useobject{currentmarker}{}%
\end{pgfscope}%
\end{pgfscope}%
\begin{pgfscope}%
\definecolor{textcolor}{rgb}{0.552941,0.501961,0.478431}%
\pgfsetstrokecolor{textcolor}%
\pgfsetfillcolor{textcolor}%
\pgftext[x=0.166667in, y=2.080939in, left, base]{\color{textcolor}{\sffamily\fontsize{9.000000}{10.800000}\selectfont\catcode`\^=\active\def^{\ifmmode\sp\else\^{}\fi}\catcode`\%=\active\def%{\%}$\mathdefault{0.8}$}}%
\end{pgfscope}%
\begin{pgfscope}%
\pgfsetbuttcap%
\pgfsetroundjoin%
\definecolor{currentfill}{rgb}{0.552941,0.501961,0.478431}%
\pgfsetfillcolor{currentfill}%
\pgfsetlinewidth{0.803000pt}%
\definecolor{currentstroke}{rgb}{0.552941,0.501961,0.478431}%
\pgfsetstrokecolor{currentstroke}%
\pgfsetdash{}{0pt}%
\pgfsys@defobject{currentmarker}{\pgfqpoint{0.000000in}{0.000000in}}{\pgfqpoint{0.083333in}{0.000000in}}{%
\pgfpathmoveto{\pgfqpoint{0.000000in}{0.000000in}}%
\pgfpathlineto{\pgfqpoint{0.083333in}{0.000000in}}%
\pgfusepath{stroke,fill}%
}%
\begin{pgfscope}%
\pgfsys@transformshift{0.379436in}{2.460711in}%
\pgfsys@useobject{currentmarker}{}%
\end{pgfscope}%
\end{pgfscope}%
\begin{pgfscope}%
\definecolor{textcolor}{rgb}{0.552941,0.501961,0.478431}%
\pgfsetstrokecolor{textcolor}%
\pgfsetfillcolor{textcolor}%
\pgftext[x=0.166667in, y=2.417309in, left, base]{\color{textcolor}{\sffamily\fontsize{9.000000}{10.800000}\selectfont\catcode`\^=\active\def^{\ifmmode\sp\else\^{}\fi}\catcode`\%=\active\def%{\%}$\mathdefault{1.0}$}}%
\end{pgfscope}%
\begin{pgfscope}%
\definecolor{textcolor}{rgb}{0.552941,0.501961,0.478431}%
\pgfsetstrokecolor{textcolor}%
\pgfsetfillcolor{textcolor}%
\pgftext[x=0.111111in,y=1.619786in,,bottom,rotate=90.000000]{\color{textcolor}{\sffamily\fontsize{9.000000}{10.800000}\selectfont\catcode`\^=\active\def^{\ifmmode\sp\else\^{}\fi}\catcode`\%=\active\def%{\%}ROC-AUC}}%
\end{pgfscope}%
\begin{pgfscope}%
\pgfpathrectangle{\pgfqpoint{0.379436in}{0.778861in}}{\pgfqpoint{2.025341in}{1.681851in}}%
\pgfusepath{clip}%
\pgfsetrectcap%
\pgfsetroundjoin%
\pgfsetlinewidth{2.258437pt}%
\definecolor{currentstroke}{rgb}{0.260000,0.260000,0.260000}%
\pgfsetstrokecolor{currentstroke}%
\pgfsetdash{}{0pt}%
\pgfpathmoveto{\pgfqpoint{0.438508in}{1.788799in}}%
\pgfpathlineto{\pgfqpoint{0.438508in}{2.029139in}}%
\pgfusepath{stroke}%
\end{pgfscope}%
\begin{pgfscope}%
\pgfpathrectangle{\pgfqpoint{0.379436in}{0.778861in}}{\pgfqpoint{2.025341in}{1.681851in}}%
\pgfusepath{clip}%
\pgfsetrectcap%
\pgfsetroundjoin%
\pgfsetlinewidth{2.258437pt}%
\definecolor{currentstroke}{rgb}{0.260000,0.260000,0.260000}%
\pgfsetstrokecolor{currentstroke}%
\pgfsetdash{}{0pt}%
\pgfpathmoveto{\pgfqpoint{0.691676in}{2.224751in}}%
\pgfpathlineto{\pgfqpoint{0.691676in}{2.380542in}}%
\pgfusepath{stroke}%
\end{pgfscope}%
\begin{pgfscope}%
\pgfpathrectangle{\pgfqpoint{0.379436in}{0.778861in}}{\pgfqpoint{2.025341in}{1.681851in}}%
\pgfusepath{clip}%
\pgfsetrectcap%
\pgfsetroundjoin%
\pgfsetlinewidth{2.258437pt}%
\definecolor{currentstroke}{rgb}{0.260000,0.260000,0.260000}%
\pgfsetstrokecolor{currentstroke}%
\pgfsetdash{}{0pt}%
\pgfpathmoveto{\pgfqpoint{0.944843in}{2.094334in}}%
\pgfpathlineto{\pgfqpoint{0.944843in}{2.262001in}}%
\pgfusepath{stroke}%
\end{pgfscope}%
\begin{pgfscope}%
\pgfpathrectangle{\pgfqpoint{0.379436in}{0.778861in}}{\pgfqpoint{2.025341in}{1.681851in}}%
\pgfusepath{clip}%
\pgfsetrectcap%
\pgfsetroundjoin%
\pgfsetlinewidth{2.258437pt}%
\definecolor{currentstroke}{rgb}{0.260000,0.260000,0.260000}%
\pgfsetstrokecolor{currentstroke}%
\pgfsetdash{}{0pt}%
\pgfpathmoveto{\pgfqpoint{1.198011in}{2.139579in}}%
\pgfpathlineto{\pgfqpoint{1.198011in}{2.214382in}}%
\pgfusepath{stroke}%
\end{pgfscope}%
\begin{pgfscope}%
\pgfpathrectangle{\pgfqpoint{0.379436in}{0.778861in}}{\pgfqpoint{2.025341in}{1.681851in}}%
\pgfusepath{clip}%
\pgfsetrectcap%
\pgfsetroundjoin%
\pgfsetlinewidth{2.258437pt}%
\definecolor{currentstroke}{rgb}{0.260000,0.260000,0.260000}%
\pgfsetstrokecolor{currentstroke}%
\pgfsetdash{}{0pt}%
\pgfpathmoveto{\pgfqpoint{1.451179in}{1.848959in}}%
\pgfpathlineto{\pgfqpoint{1.451179in}{2.082260in}}%
\pgfusepath{stroke}%
\end{pgfscope}%
\begin{pgfscope}%
\pgfpathrectangle{\pgfqpoint{0.379436in}{0.778861in}}{\pgfqpoint{2.025341in}{1.681851in}}%
\pgfusepath{clip}%
\pgfsetrectcap%
\pgfsetroundjoin%
\pgfsetlinewidth{2.258437pt}%
\definecolor{currentstroke}{rgb}{0.260000,0.260000,0.260000}%
\pgfsetstrokecolor{currentstroke}%
\pgfsetdash{}{0pt}%
\pgfpathmoveto{\pgfqpoint{1.704346in}{1.892100in}}%
\pgfpathlineto{\pgfqpoint{1.704346in}{2.046662in}}%
\pgfusepath{stroke}%
\end{pgfscope}%
\begin{pgfscope}%
\pgfpathrectangle{\pgfqpoint{0.379436in}{0.778861in}}{\pgfqpoint{2.025341in}{1.681851in}}%
\pgfusepath{clip}%
\pgfsetrectcap%
\pgfsetroundjoin%
\pgfsetlinewidth{2.258437pt}%
\definecolor{currentstroke}{rgb}{0.260000,0.260000,0.260000}%
\pgfsetstrokecolor{currentstroke}%
\pgfsetdash{}{0pt}%
\pgfpathmoveto{\pgfqpoint{1.957514in}{1.776149in}}%
\pgfpathlineto{\pgfqpoint{1.957514in}{1.993194in}}%
\pgfusepath{stroke}%
\end{pgfscope}%
\begin{pgfscope}%
\pgfpathrectangle{\pgfqpoint{0.379436in}{0.778861in}}{\pgfqpoint{2.025341in}{1.681851in}}%
\pgfusepath{clip}%
\pgfsetrectcap%
\pgfsetroundjoin%
\pgfsetlinewidth{2.258437pt}%
\definecolor{currentstroke}{rgb}{0.260000,0.260000,0.260000}%
\pgfsetstrokecolor{currentstroke}%
\pgfsetdash{}{0pt}%
\pgfpathmoveto{\pgfqpoint{2.210682in}{2.009845in}}%
\pgfpathlineto{\pgfqpoint{2.210682in}{2.103358in}}%
\pgfusepath{stroke}%
\end{pgfscope}%
\begin{pgfscope}%
\pgfpathrectangle{\pgfqpoint{0.379436in}{0.778861in}}{\pgfqpoint{2.025341in}{1.681851in}}%
\pgfusepath{clip}%
\pgfsetrectcap%
\pgfsetroundjoin%
\pgfsetlinewidth{2.258437pt}%
\definecolor{currentstroke}{rgb}{0.260000,0.260000,0.260000}%
\pgfsetstrokecolor{currentstroke}%
\pgfsetdash{}{0pt}%
\pgfpathmoveto{\pgfqpoint{0.506020in}{1.855519in}}%
\pgfpathlineto{\pgfqpoint{0.506020in}{2.106219in}}%
\pgfusepath{stroke}%
\end{pgfscope}%
\begin{pgfscope}%
\pgfpathrectangle{\pgfqpoint{0.379436in}{0.778861in}}{\pgfqpoint{2.025341in}{1.681851in}}%
\pgfusepath{clip}%
\pgfsetrectcap%
\pgfsetroundjoin%
\pgfsetlinewidth{2.258437pt}%
\definecolor{currentstroke}{rgb}{0.260000,0.260000,0.260000}%
\pgfsetstrokecolor{currentstroke}%
\pgfsetdash{}{0pt}%
\pgfpathmoveto{\pgfqpoint{0.759187in}{2.213660in}}%
\pgfpathlineto{\pgfqpoint{0.759187in}{2.349212in}}%
\pgfusepath{stroke}%
\end{pgfscope}%
\begin{pgfscope}%
\pgfpathrectangle{\pgfqpoint{0.379436in}{0.778861in}}{\pgfqpoint{2.025341in}{1.681851in}}%
\pgfusepath{clip}%
\pgfsetrectcap%
\pgfsetroundjoin%
\pgfsetlinewidth{2.258437pt}%
\definecolor{currentstroke}{rgb}{0.260000,0.260000,0.260000}%
\pgfsetstrokecolor{currentstroke}%
\pgfsetdash{}{0pt}%
\pgfpathmoveto{\pgfqpoint{1.012355in}{2.072428in}}%
\pgfpathlineto{\pgfqpoint{1.012355in}{2.255954in}}%
\pgfusepath{stroke}%
\end{pgfscope}%
\begin{pgfscope}%
\pgfpathrectangle{\pgfqpoint{0.379436in}{0.778861in}}{\pgfqpoint{2.025341in}{1.681851in}}%
\pgfusepath{clip}%
\pgfsetrectcap%
\pgfsetroundjoin%
\pgfsetlinewidth{2.258437pt}%
\definecolor{currentstroke}{rgb}{0.260000,0.260000,0.260000}%
\pgfsetstrokecolor{currentstroke}%
\pgfsetdash{}{0pt}%
\pgfpathmoveto{\pgfqpoint{1.265522in}{2.066844in}}%
\pgfpathlineto{\pgfqpoint{1.265522in}{2.193048in}}%
\pgfusepath{stroke}%
\end{pgfscope}%
\begin{pgfscope}%
\pgfpathrectangle{\pgfqpoint{0.379436in}{0.778861in}}{\pgfqpoint{2.025341in}{1.681851in}}%
\pgfusepath{clip}%
\pgfsetrectcap%
\pgfsetroundjoin%
\pgfsetlinewidth{2.258437pt}%
\definecolor{currentstroke}{rgb}{0.260000,0.260000,0.260000}%
\pgfsetstrokecolor{currentstroke}%
\pgfsetdash{}{0pt}%
\pgfpathmoveto{\pgfqpoint{1.518690in}{1.787262in}}%
\pgfpathlineto{\pgfqpoint{1.518690in}{2.044964in}}%
\pgfusepath{stroke}%
\end{pgfscope}%
\begin{pgfscope}%
\pgfpathrectangle{\pgfqpoint{0.379436in}{0.778861in}}{\pgfqpoint{2.025341in}{1.681851in}}%
\pgfusepath{clip}%
\pgfsetrectcap%
\pgfsetroundjoin%
\pgfsetlinewidth{2.258437pt}%
\definecolor{currentstroke}{rgb}{0.260000,0.260000,0.260000}%
\pgfsetstrokecolor{currentstroke}%
\pgfsetdash{}{0pt}%
\pgfpathmoveto{\pgfqpoint{1.771858in}{1.880057in}}%
\pgfpathlineto{\pgfqpoint{1.771858in}{2.062938in}}%
\pgfusepath{stroke}%
\end{pgfscope}%
\begin{pgfscope}%
\pgfpathrectangle{\pgfqpoint{0.379436in}{0.778861in}}{\pgfqpoint{2.025341in}{1.681851in}}%
\pgfusepath{clip}%
\pgfsetrectcap%
\pgfsetroundjoin%
\pgfsetlinewidth{2.258437pt}%
\definecolor{currentstroke}{rgb}{0.260000,0.260000,0.260000}%
\pgfsetstrokecolor{currentstroke}%
\pgfsetdash{}{0pt}%
\pgfpathmoveto{\pgfqpoint{2.025025in}{1.788708in}}%
\pgfpathlineto{\pgfqpoint{2.025025in}{1.985145in}}%
\pgfusepath{stroke}%
\end{pgfscope}%
\begin{pgfscope}%
\pgfpathrectangle{\pgfqpoint{0.379436in}{0.778861in}}{\pgfqpoint{2.025341in}{1.681851in}}%
\pgfusepath{clip}%
\pgfsetrectcap%
\pgfsetroundjoin%
\pgfsetlinewidth{2.258437pt}%
\definecolor{currentstroke}{rgb}{0.260000,0.260000,0.260000}%
\pgfsetstrokecolor{currentstroke}%
\pgfsetdash{}{0pt}%
\pgfpathmoveto{\pgfqpoint{2.278193in}{1.999497in}}%
\pgfpathlineto{\pgfqpoint{2.278193in}{2.098688in}}%
\pgfusepath{stroke}%
\end{pgfscope}%
\begin{pgfscope}%
\pgfpathrectangle{\pgfqpoint{0.379436in}{0.778861in}}{\pgfqpoint{2.025341in}{1.681851in}}%
\pgfusepath{clip}%
\pgfsetrectcap%
\pgfsetroundjoin%
\pgfsetlinewidth{2.258437pt}%
\definecolor{currentstroke}{rgb}{0.260000,0.260000,0.260000}%
\pgfsetstrokecolor{currentstroke}%
\pgfsetdash{}{0pt}%
\pgfpathmoveto{\pgfqpoint{0.573531in}{1.847615in}}%
\pgfpathlineto{\pgfqpoint{0.573531in}{1.993107in}}%
\pgfusepath{stroke}%
\end{pgfscope}%
\begin{pgfscope}%
\pgfpathrectangle{\pgfqpoint{0.379436in}{0.778861in}}{\pgfqpoint{2.025341in}{1.681851in}}%
\pgfusepath{clip}%
\pgfsetrectcap%
\pgfsetroundjoin%
\pgfsetlinewidth{2.258437pt}%
\definecolor{currentstroke}{rgb}{0.260000,0.260000,0.260000}%
\pgfsetstrokecolor{currentstroke}%
\pgfsetdash{}{0pt}%
\pgfpathmoveto{\pgfqpoint{0.826699in}{2.217741in}}%
\pgfpathlineto{\pgfqpoint{0.826699in}{2.346020in}}%
\pgfusepath{stroke}%
\end{pgfscope}%
\begin{pgfscope}%
\pgfpathrectangle{\pgfqpoint{0.379436in}{0.778861in}}{\pgfqpoint{2.025341in}{1.681851in}}%
\pgfusepath{clip}%
\pgfsetrectcap%
\pgfsetroundjoin%
\pgfsetlinewidth{2.258437pt}%
\definecolor{currentstroke}{rgb}{0.260000,0.260000,0.260000}%
\pgfsetstrokecolor{currentstroke}%
\pgfsetdash{}{0pt}%
\pgfpathmoveto{\pgfqpoint{1.079866in}{2.030903in}}%
\pgfpathlineto{\pgfqpoint{1.079866in}{2.226678in}}%
\pgfusepath{stroke}%
\end{pgfscope}%
\begin{pgfscope}%
\pgfpathrectangle{\pgfqpoint{0.379436in}{0.778861in}}{\pgfqpoint{2.025341in}{1.681851in}}%
\pgfusepath{clip}%
\pgfsetrectcap%
\pgfsetroundjoin%
\pgfsetlinewidth{2.258437pt}%
\definecolor{currentstroke}{rgb}{0.260000,0.260000,0.260000}%
\pgfsetstrokecolor{currentstroke}%
\pgfsetdash{}{0pt}%
\pgfpathmoveto{\pgfqpoint{1.333034in}{2.089972in}}%
\pgfpathlineto{\pgfqpoint{1.333034in}{2.230854in}}%
\pgfusepath{stroke}%
\end{pgfscope}%
\begin{pgfscope}%
\pgfpathrectangle{\pgfqpoint{0.379436in}{0.778861in}}{\pgfqpoint{2.025341in}{1.681851in}}%
\pgfusepath{clip}%
\pgfsetrectcap%
\pgfsetroundjoin%
\pgfsetlinewidth{2.258437pt}%
\definecolor{currentstroke}{rgb}{0.260000,0.260000,0.260000}%
\pgfsetstrokecolor{currentstroke}%
\pgfsetdash{}{0pt}%
\pgfpathmoveto{\pgfqpoint{1.586201in}{1.846090in}}%
\pgfpathlineto{\pgfqpoint{1.586201in}{2.106130in}}%
\pgfusepath{stroke}%
\end{pgfscope}%
\begin{pgfscope}%
\pgfpathrectangle{\pgfqpoint{0.379436in}{0.778861in}}{\pgfqpoint{2.025341in}{1.681851in}}%
\pgfusepath{clip}%
\pgfsetrectcap%
\pgfsetroundjoin%
\pgfsetlinewidth{2.258437pt}%
\definecolor{currentstroke}{rgb}{0.260000,0.260000,0.260000}%
\pgfsetstrokecolor{currentstroke}%
\pgfsetdash{}{0pt}%
\pgfpathmoveto{\pgfqpoint{1.839369in}{1.749817in}}%
\pgfpathlineto{\pgfqpoint{1.839369in}{1.991253in}}%
\pgfusepath{stroke}%
\end{pgfscope}%
\begin{pgfscope}%
\pgfpathrectangle{\pgfqpoint{0.379436in}{0.778861in}}{\pgfqpoint{2.025341in}{1.681851in}}%
\pgfusepath{clip}%
\pgfsetrectcap%
\pgfsetroundjoin%
\pgfsetlinewidth{2.258437pt}%
\definecolor{currentstroke}{rgb}{0.260000,0.260000,0.260000}%
\pgfsetstrokecolor{currentstroke}%
\pgfsetdash{}{0pt}%
\pgfpathmoveto{\pgfqpoint{2.092537in}{1.703751in}}%
\pgfpathlineto{\pgfqpoint{2.092537in}{1.916821in}}%
\pgfusepath{stroke}%
\end{pgfscope}%
\begin{pgfscope}%
\pgfpathrectangle{\pgfqpoint{0.379436in}{0.778861in}}{\pgfqpoint{2.025341in}{1.681851in}}%
\pgfusepath{clip}%
\pgfsetrectcap%
\pgfsetroundjoin%
\pgfsetlinewidth{2.258437pt}%
\definecolor{currentstroke}{rgb}{0.260000,0.260000,0.260000}%
\pgfsetstrokecolor{currentstroke}%
\pgfsetdash{}{0pt}%
\pgfpathmoveto{\pgfqpoint{2.345704in}{1.975249in}}%
\pgfpathlineto{\pgfqpoint{2.345704in}{2.077297in}}%
\pgfusepath{stroke}%
\end{pgfscope}%
\begin{pgfscope}%
\pgfpathrectangle{\pgfqpoint{0.379436in}{0.778861in}}{\pgfqpoint{2.025341in}{1.681851in}}%
\pgfusepath{clip}%
\pgfsetbuttcap%
\pgfsetroundjoin%
\pgfsetlinewidth{1.003750pt}%
\definecolor{currentstroke}{rgb}{0.552941,0.501961,0.478431}%
\pgfsetstrokecolor{currentstroke}%
\pgfsetdash{{3.700000pt}{1.600000pt}}{0.000000pt}%
\pgfpathmoveto{\pgfqpoint{0.379436in}{1.619786in}}%
\pgfpathlineto{\pgfqpoint{2.404777in}{1.619786in}}%
\pgfusepath{stroke}%
\end{pgfscope}%
\begin{pgfscope}%
\pgfsetrectcap%
\pgfsetmiterjoin%
\pgfsetlinewidth{0.803000pt}%
\definecolor{currentstroke}{rgb}{0.552941,0.501961,0.478431}%
\pgfsetstrokecolor{currentstroke}%
\pgfsetdash{}{0pt}%
\pgfpathmoveto{\pgfqpoint{0.379436in}{0.778861in}}%
\pgfpathlineto{\pgfqpoint{0.379436in}{2.460711in}}%
\pgfusepath{stroke}%
\end{pgfscope}%
\begin{pgfscope}%
\pgfsetrectcap%
\pgfsetmiterjoin%
\pgfsetlinewidth{0.803000pt}%
\definecolor{currentstroke}{rgb}{0.552941,0.501961,0.478431}%
\pgfsetstrokecolor{currentstroke}%
\pgfsetdash{}{0pt}%
\pgfpathmoveto{\pgfqpoint{0.379436in}{0.778861in}}%
\pgfpathlineto{\pgfqpoint{2.404777in}{0.778861in}}%
\pgfusepath{stroke}%
\end{pgfscope}%
\begin{pgfscope}%
\definecolor{textcolor}{rgb}{0.552941,0.501961,0.478431}%
\pgfsetstrokecolor{textcolor}%
\pgfsetfillcolor{textcolor}%
\pgftext[x=0.379436in,y=2.544045in,left,base]{\color{textcolor}{\sffamily\fontsize{9.000000}{10.800000}\selectfont\catcode`\^=\active\def^{\ifmmode\sp\else\^{}\fi}\catcode`\%=\active\def%{\%}overt VSA}}%
\end{pgfscope}%
\begin{pgfscope}%
\pgfsetbuttcap%
\pgfsetmiterjoin%
\pgfsetlinewidth{0.000000pt}%
\definecolor{currentstroke}{rgb}{0.000000,0.000000,0.000000}%
\pgfsetstrokecolor{currentstroke}%
\pgfsetstrokeopacity{0.000000}%
\pgfsetdash{}{0pt}%
\pgfpathmoveto{\pgfqpoint{2.469777in}{0.778861in}}%
\pgfpathlineto{\pgfqpoint{4.495118in}{0.778861in}}%
\pgfpathlineto{\pgfqpoint{4.495118in}{2.460711in}}%
\pgfpathlineto{\pgfqpoint{2.469777in}{2.460711in}}%
\pgfpathlineto{\pgfqpoint{2.469777in}{0.778861in}}%
\pgfpathclose%
\pgfusepath{}%
\end{pgfscope}%
\begin{pgfscope}%
\pgfpathrectangle{\pgfqpoint{2.469777in}{0.778861in}}{\pgfqpoint{2.025341in}{1.681851in}}%
\pgfusepath{clip}%
\pgfsetbuttcap%
\pgfsetmiterjoin%
\definecolor{currentfill}{rgb}{0.842157,0.553922,0.200980}%
\pgfsetfillcolor{currentfill}%
\pgfsetlinewidth{0.000000pt}%
\definecolor{currentstroke}{rgb}{0.000000,0.000000,0.000000}%
\pgfsetstrokecolor{currentstroke}%
\pgfsetstrokeopacity{0.000000}%
\pgfsetdash{}{0pt}%
\pgfpathmoveto{\pgfqpoint{2.495094in}{0.778861in}}%
\pgfpathlineto{\pgfqpoint{2.562605in}{0.778861in}}%
\pgfpathlineto{\pgfqpoint{2.562605in}{1.789774in}}%
\pgfpathlineto{\pgfqpoint{2.495094in}{1.789774in}}%
\pgfpathlineto{\pgfqpoint{2.495094in}{0.778861in}}%
\pgfpathclose%
\pgfusepath{fill}%
\end{pgfscope}%
\begin{pgfscope}%
\pgfpathrectangle{\pgfqpoint{2.469777in}{0.778861in}}{\pgfqpoint{2.025341in}{1.681851in}}%
\pgfusepath{clip}%
\pgfsetbuttcap%
\pgfsetmiterjoin%
\definecolor{currentfill}{rgb}{0.842157,0.553922,0.200980}%
\pgfsetfillcolor{currentfill}%
\pgfsetlinewidth{0.000000pt}%
\definecolor{currentstroke}{rgb}{0.000000,0.000000,0.000000}%
\pgfsetstrokecolor{currentstroke}%
\pgfsetstrokeopacity{0.000000}%
\pgfsetdash{}{0pt}%
\pgfpathmoveto{\pgfqpoint{2.748261in}{0.778861in}}%
\pgfpathlineto{\pgfqpoint{2.815773in}{0.778861in}}%
\pgfpathlineto{\pgfqpoint{2.815773in}{1.955947in}}%
\pgfpathlineto{\pgfqpoint{2.748261in}{1.955947in}}%
\pgfpathlineto{\pgfqpoint{2.748261in}{0.778861in}}%
\pgfpathclose%
\pgfusepath{fill}%
\end{pgfscope}%
\begin{pgfscope}%
\pgfpathrectangle{\pgfqpoint{2.469777in}{0.778861in}}{\pgfqpoint{2.025341in}{1.681851in}}%
\pgfusepath{clip}%
\pgfsetbuttcap%
\pgfsetmiterjoin%
\definecolor{currentfill}{rgb}{0.842157,0.553922,0.200980}%
\pgfsetfillcolor{currentfill}%
\pgfsetlinewidth{0.000000pt}%
\definecolor{currentstroke}{rgb}{0.000000,0.000000,0.000000}%
\pgfsetstrokecolor{currentstroke}%
\pgfsetstrokeopacity{0.000000}%
\pgfsetdash{}{0pt}%
\pgfpathmoveto{\pgfqpoint{3.001429in}{0.778861in}}%
\pgfpathlineto{\pgfqpoint{3.068940in}{0.778861in}}%
\pgfpathlineto{\pgfqpoint{3.068940in}{1.818951in}}%
\pgfpathlineto{\pgfqpoint{3.001429in}{1.818951in}}%
\pgfpathlineto{\pgfqpoint{3.001429in}{0.778861in}}%
\pgfpathclose%
\pgfusepath{fill}%
\end{pgfscope}%
\begin{pgfscope}%
\pgfpathrectangle{\pgfqpoint{2.469777in}{0.778861in}}{\pgfqpoint{2.025341in}{1.681851in}}%
\pgfusepath{clip}%
\pgfsetbuttcap%
\pgfsetmiterjoin%
\definecolor{currentfill}{rgb}{0.842157,0.553922,0.200980}%
\pgfsetfillcolor{currentfill}%
\pgfsetlinewidth{0.000000pt}%
\definecolor{currentstroke}{rgb}{0.000000,0.000000,0.000000}%
\pgfsetstrokecolor{currentstroke}%
\pgfsetstrokeopacity{0.000000}%
\pgfsetdash{}{0pt}%
\pgfpathmoveto{\pgfqpoint{3.254597in}{0.778861in}}%
\pgfpathlineto{\pgfqpoint{3.322108in}{0.778861in}}%
\pgfpathlineto{\pgfqpoint{3.322108in}{1.650777in}}%
\pgfpathlineto{\pgfqpoint{3.254597in}{1.650777in}}%
\pgfpathlineto{\pgfqpoint{3.254597in}{0.778861in}}%
\pgfpathclose%
\pgfusepath{fill}%
\end{pgfscope}%
\begin{pgfscope}%
\pgfpathrectangle{\pgfqpoint{2.469777in}{0.778861in}}{\pgfqpoint{2.025341in}{1.681851in}}%
\pgfusepath{clip}%
\pgfsetbuttcap%
\pgfsetmiterjoin%
\definecolor{currentfill}{rgb}{0.842157,0.553922,0.200980}%
\pgfsetfillcolor{currentfill}%
\pgfsetlinewidth{0.000000pt}%
\definecolor{currentstroke}{rgb}{0.000000,0.000000,0.000000}%
\pgfsetstrokecolor{currentstroke}%
\pgfsetstrokeopacity{0.000000}%
\pgfsetdash{}{0pt}%
\pgfpathmoveto{\pgfqpoint{3.507764in}{0.778861in}}%
\pgfpathlineto{\pgfqpoint{3.575276in}{0.778861in}}%
\pgfpathlineto{\pgfqpoint{3.575276in}{1.753191in}}%
\pgfpathlineto{\pgfqpoint{3.507764in}{1.753191in}}%
\pgfpathlineto{\pgfqpoint{3.507764in}{0.778861in}}%
\pgfpathclose%
\pgfusepath{fill}%
\end{pgfscope}%
\begin{pgfscope}%
\pgfpathrectangle{\pgfqpoint{2.469777in}{0.778861in}}{\pgfqpoint{2.025341in}{1.681851in}}%
\pgfusepath{clip}%
\pgfsetbuttcap%
\pgfsetmiterjoin%
\definecolor{currentfill}{rgb}{0.842157,0.553922,0.200980}%
\pgfsetfillcolor{currentfill}%
\pgfsetlinewidth{0.000000pt}%
\definecolor{currentstroke}{rgb}{0.000000,0.000000,0.000000}%
\pgfsetstrokecolor{currentstroke}%
\pgfsetstrokeopacity{0.000000}%
\pgfsetdash{}{0pt}%
\pgfpathmoveto{\pgfqpoint{3.760932in}{0.778861in}}%
\pgfpathlineto{\pgfqpoint{3.828443in}{0.778861in}}%
\pgfpathlineto{\pgfqpoint{3.828443in}{1.614086in}}%
\pgfpathlineto{\pgfqpoint{3.760932in}{1.614086in}}%
\pgfpathlineto{\pgfqpoint{3.760932in}{0.778861in}}%
\pgfpathclose%
\pgfusepath{fill}%
\end{pgfscope}%
\begin{pgfscope}%
\pgfpathrectangle{\pgfqpoint{2.469777in}{0.778861in}}{\pgfqpoint{2.025341in}{1.681851in}}%
\pgfusepath{clip}%
\pgfsetbuttcap%
\pgfsetmiterjoin%
\definecolor{currentfill}{rgb}{0.842157,0.553922,0.200980}%
\pgfsetfillcolor{currentfill}%
\pgfsetlinewidth{0.000000pt}%
\definecolor{currentstroke}{rgb}{0.000000,0.000000,0.000000}%
\pgfsetstrokecolor{currentstroke}%
\pgfsetstrokeopacity{0.000000}%
\pgfsetdash{}{0pt}%
\pgfpathmoveto{\pgfqpoint{4.014100in}{0.778861in}}%
\pgfpathlineto{\pgfqpoint{4.081611in}{0.778861in}}%
\pgfpathlineto{\pgfqpoint{4.081611in}{1.820960in}}%
\pgfpathlineto{\pgfqpoint{4.014100in}{1.820960in}}%
\pgfpathlineto{\pgfqpoint{4.014100in}{0.778861in}}%
\pgfpathclose%
\pgfusepath{fill}%
\end{pgfscope}%
\begin{pgfscope}%
\pgfpathrectangle{\pgfqpoint{2.469777in}{0.778861in}}{\pgfqpoint{2.025341in}{1.681851in}}%
\pgfusepath{clip}%
\pgfsetbuttcap%
\pgfsetmiterjoin%
\definecolor{currentfill}{rgb}{0.842157,0.553922,0.200980}%
\pgfsetfillcolor{currentfill}%
\pgfsetlinewidth{0.000000pt}%
\definecolor{currentstroke}{rgb}{0.000000,0.000000,0.000000}%
\pgfsetstrokecolor{currentstroke}%
\pgfsetstrokeopacity{0.000000}%
\pgfsetdash{}{0pt}%
\pgfpathmoveto{\pgfqpoint{4.267267in}{0.778861in}}%
\pgfpathlineto{\pgfqpoint{4.334779in}{0.778861in}}%
\pgfpathlineto{\pgfqpoint{4.334779in}{1.771955in}}%
\pgfpathlineto{\pgfqpoint{4.267267in}{1.771955in}}%
\pgfpathlineto{\pgfqpoint{4.267267in}{0.778861in}}%
\pgfpathclose%
\pgfusepath{fill}%
\end{pgfscope}%
\begin{pgfscope}%
\pgfpathrectangle{\pgfqpoint{2.469777in}{0.778861in}}{\pgfqpoint{2.025341in}{1.681851in}}%
\pgfusepath{clip}%
\pgfsetbuttcap%
\pgfsetmiterjoin%
\definecolor{currentfill}{rgb}{0.858824,0.314706,0.223529}%
\pgfsetfillcolor{currentfill}%
\pgfsetlinewidth{0.000000pt}%
\definecolor{currentstroke}{rgb}{0.000000,0.000000,0.000000}%
\pgfsetstrokecolor{currentstroke}%
\pgfsetstrokeopacity{0.000000}%
\pgfsetdash{}{0pt}%
\pgfpathmoveto{\pgfqpoint{2.562605in}{0.778861in}}%
\pgfpathlineto{\pgfqpoint{2.630116in}{0.778861in}}%
\pgfpathlineto{\pgfqpoint{2.630116in}{1.814512in}}%
\pgfpathlineto{\pgfqpoint{2.562605in}{1.814512in}}%
\pgfpathlineto{\pgfqpoint{2.562605in}{0.778861in}}%
\pgfpathclose%
\pgfusepath{fill}%
\end{pgfscope}%
\begin{pgfscope}%
\pgfpathrectangle{\pgfqpoint{2.469777in}{0.778861in}}{\pgfqpoint{2.025341in}{1.681851in}}%
\pgfusepath{clip}%
\pgfsetbuttcap%
\pgfsetmiterjoin%
\definecolor{currentfill}{rgb}{0.858824,0.314706,0.223529}%
\pgfsetfillcolor{currentfill}%
\pgfsetlinewidth{0.000000pt}%
\definecolor{currentstroke}{rgb}{0.000000,0.000000,0.000000}%
\pgfsetstrokecolor{currentstroke}%
\pgfsetstrokeopacity{0.000000}%
\pgfsetdash{}{0pt}%
\pgfpathmoveto{\pgfqpoint{2.815773in}{0.778861in}}%
\pgfpathlineto{\pgfqpoint{2.883284in}{0.778861in}}%
\pgfpathlineto{\pgfqpoint{2.883284in}{2.036698in}}%
\pgfpathlineto{\pgfqpoint{2.815773in}{2.036698in}}%
\pgfpathlineto{\pgfqpoint{2.815773in}{0.778861in}}%
\pgfpathclose%
\pgfusepath{fill}%
\end{pgfscope}%
\begin{pgfscope}%
\pgfpathrectangle{\pgfqpoint{2.469777in}{0.778861in}}{\pgfqpoint{2.025341in}{1.681851in}}%
\pgfusepath{clip}%
\pgfsetbuttcap%
\pgfsetmiterjoin%
\definecolor{currentfill}{rgb}{0.858824,0.314706,0.223529}%
\pgfsetfillcolor{currentfill}%
\pgfsetlinewidth{0.000000pt}%
\definecolor{currentstroke}{rgb}{0.000000,0.000000,0.000000}%
\pgfsetstrokecolor{currentstroke}%
\pgfsetstrokeopacity{0.000000}%
\pgfsetdash{}{0pt}%
\pgfpathmoveto{\pgfqpoint{3.068940in}{0.778861in}}%
\pgfpathlineto{\pgfqpoint{3.136452in}{0.778861in}}%
\pgfpathlineto{\pgfqpoint{3.136452in}{1.961118in}}%
\pgfpathlineto{\pgfqpoint{3.068940in}{1.961118in}}%
\pgfpathlineto{\pgfqpoint{3.068940in}{0.778861in}}%
\pgfpathclose%
\pgfusepath{fill}%
\end{pgfscope}%
\begin{pgfscope}%
\pgfpathrectangle{\pgfqpoint{2.469777in}{0.778861in}}{\pgfqpoint{2.025341in}{1.681851in}}%
\pgfusepath{clip}%
\pgfsetbuttcap%
\pgfsetmiterjoin%
\definecolor{currentfill}{rgb}{0.858824,0.314706,0.223529}%
\pgfsetfillcolor{currentfill}%
\pgfsetlinewidth{0.000000pt}%
\definecolor{currentstroke}{rgb}{0.000000,0.000000,0.000000}%
\pgfsetstrokecolor{currentstroke}%
\pgfsetstrokeopacity{0.000000}%
\pgfsetdash{}{0pt}%
\pgfpathmoveto{\pgfqpoint{3.322108in}{0.778861in}}%
\pgfpathlineto{\pgfqpoint{3.389619in}{0.778861in}}%
\pgfpathlineto{\pgfqpoint{3.389619in}{1.592037in}}%
\pgfpathlineto{\pgfqpoint{3.322108in}{1.592037in}}%
\pgfpathlineto{\pgfqpoint{3.322108in}{0.778861in}}%
\pgfpathclose%
\pgfusepath{fill}%
\end{pgfscope}%
\begin{pgfscope}%
\pgfpathrectangle{\pgfqpoint{2.469777in}{0.778861in}}{\pgfqpoint{2.025341in}{1.681851in}}%
\pgfusepath{clip}%
\pgfsetbuttcap%
\pgfsetmiterjoin%
\definecolor{currentfill}{rgb}{0.858824,0.314706,0.223529}%
\pgfsetfillcolor{currentfill}%
\pgfsetlinewidth{0.000000pt}%
\definecolor{currentstroke}{rgb}{0.000000,0.000000,0.000000}%
\pgfsetstrokecolor{currentstroke}%
\pgfsetstrokeopacity{0.000000}%
\pgfsetdash{}{0pt}%
\pgfpathmoveto{\pgfqpoint{3.575276in}{0.778861in}}%
\pgfpathlineto{\pgfqpoint{3.642787in}{0.778861in}}%
\pgfpathlineto{\pgfqpoint{3.642787in}{1.832961in}}%
\pgfpathlineto{\pgfqpoint{3.575276in}{1.832961in}}%
\pgfpathlineto{\pgfqpoint{3.575276in}{0.778861in}}%
\pgfpathclose%
\pgfusepath{fill}%
\end{pgfscope}%
\begin{pgfscope}%
\pgfpathrectangle{\pgfqpoint{2.469777in}{0.778861in}}{\pgfqpoint{2.025341in}{1.681851in}}%
\pgfusepath{clip}%
\pgfsetbuttcap%
\pgfsetmiterjoin%
\definecolor{currentfill}{rgb}{0.858824,0.314706,0.223529}%
\pgfsetfillcolor{currentfill}%
\pgfsetlinewidth{0.000000pt}%
\definecolor{currentstroke}{rgb}{0.000000,0.000000,0.000000}%
\pgfsetstrokecolor{currentstroke}%
\pgfsetstrokeopacity{0.000000}%
\pgfsetdash{}{0pt}%
\pgfpathmoveto{\pgfqpoint{3.828443in}{0.778861in}}%
\pgfpathlineto{\pgfqpoint{3.895955in}{0.778861in}}%
\pgfpathlineto{\pgfqpoint{3.895955in}{1.678622in}}%
\pgfpathlineto{\pgfqpoint{3.828443in}{1.678622in}}%
\pgfpathlineto{\pgfqpoint{3.828443in}{0.778861in}}%
\pgfpathclose%
\pgfusepath{fill}%
\end{pgfscope}%
\begin{pgfscope}%
\pgfpathrectangle{\pgfqpoint{2.469777in}{0.778861in}}{\pgfqpoint{2.025341in}{1.681851in}}%
\pgfusepath{clip}%
\pgfsetbuttcap%
\pgfsetmiterjoin%
\definecolor{currentfill}{rgb}{0.858824,0.314706,0.223529}%
\pgfsetfillcolor{currentfill}%
\pgfsetlinewidth{0.000000pt}%
\definecolor{currentstroke}{rgb}{0.000000,0.000000,0.000000}%
\pgfsetstrokecolor{currentstroke}%
\pgfsetstrokeopacity{0.000000}%
\pgfsetdash{}{0pt}%
\pgfpathmoveto{\pgfqpoint{4.081611in}{0.778861in}}%
\pgfpathlineto{\pgfqpoint{4.149122in}{0.778861in}}%
\pgfpathlineto{\pgfqpoint{4.149122in}{1.893121in}}%
\pgfpathlineto{\pgfqpoint{4.081611in}{1.893121in}}%
\pgfpathlineto{\pgfqpoint{4.081611in}{0.778861in}}%
\pgfpathclose%
\pgfusepath{fill}%
\end{pgfscope}%
\begin{pgfscope}%
\pgfpathrectangle{\pgfqpoint{2.469777in}{0.778861in}}{\pgfqpoint{2.025341in}{1.681851in}}%
\pgfusepath{clip}%
\pgfsetbuttcap%
\pgfsetmiterjoin%
\definecolor{currentfill}{rgb}{0.858824,0.314706,0.223529}%
\pgfsetfillcolor{currentfill}%
\pgfsetlinewidth{0.000000pt}%
\definecolor{currentstroke}{rgb}{0.000000,0.000000,0.000000}%
\pgfsetstrokecolor{currentstroke}%
\pgfsetstrokeopacity{0.000000}%
\pgfsetdash{}{0pt}%
\pgfpathmoveto{\pgfqpoint{4.334779in}{0.778861in}}%
\pgfpathlineto{\pgfqpoint{4.402290in}{0.778861in}}%
\pgfpathlineto{\pgfqpoint{4.402290in}{1.829867in}}%
\pgfpathlineto{\pgfqpoint{4.334779in}{1.829867in}}%
\pgfpathlineto{\pgfqpoint{4.334779in}{0.778861in}}%
\pgfpathclose%
\pgfusepath{fill}%
\end{pgfscope}%
\begin{pgfscope}%
\pgfpathrectangle{\pgfqpoint{2.469777in}{0.778861in}}{\pgfqpoint{2.025341in}{1.681851in}}%
\pgfusepath{clip}%
\pgfsetbuttcap%
\pgfsetmiterjoin%
\definecolor{currentfill}{rgb}{0.464706,0.320588,0.573529}%
\pgfsetfillcolor{currentfill}%
\pgfsetlinewidth{0.000000pt}%
\definecolor{currentstroke}{rgb}{0.000000,0.000000,0.000000}%
\pgfsetstrokecolor{currentstroke}%
\pgfsetstrokeopacity{0.000000}%
\pgfsetdash{}{0pt}%
\pgfpathmoveto{\pgfqpoint{2.630116in}{0.778861in}}%
\pgfpathlineto{\pgfqpoint{2.697628in}{0.778861in}}%
\pgfpathlineto{\pgfqpoint{2.697628in}{1.827927in}}%
\pgfpathlineto{\pgfqpoint{2.630116in}{1.827927in}}%
\pgfpathlineto{\pgfqpoint{2.630116in}{0.778861in}}%
\pgfpathclose%
\pgfusepath{fill}%
\end{pgfscope}%
\begin{pgfscope}%
\pgfpathrectangle{\pgfqpoint{2.469777in}{0.778861in}}{\pgfqpoint{2.025341in}{1.681851in}}%
\pgfusepath{clip}%
\pgfsetbuttcap%
\pgfsetmiterjoin%
\definecolor{currentfill}{rgb}{0.464706,0.320588,0.573529}%
\pgfsetfillcolor{currentfill}%
\pgfsetlinewidth{0.000000pt}%
\definecolor{currentstroke}{rgb}{0.000000,0.000000,0.000000}%
\pgfsetstrokecolor{currentstroke}%
\pgfsetstrokeopacity{0.000000}%
\pgfsetdash{}{0pt}%
\pgfpathmoveto{\pgfqpoint{2.883284in}{0.778861in}}%
\pgfpathlineto{\pgfqpoint{2.950795in}{0.778861in}}%
\pgfpathlineto{\pgfqpoint{2.950795in}{1.972674in}}%
\pgfpathlineto{\pgfqpoint{2.883284in}{1.972674in}}%
\pgfpathlineto{\pgfqpoint{2.883284in}{0.778861in}}%
\pgfpathclose%
\pgfusepath{fill}%
\end{pgfscope}%
\begin{pgfscope}%
\pgfpathrectangle{\pgfqpoint{2.469777in}{0.778861in}}{\pgfqpoint{2.025341in}{1.681851in}}%
\pgfusepath{clip}%
\pgfsetbuttcap%
\pgfsetmiterjoin%
\definecolor{currentfill}{rgb}{0.464706,0.320588,0.573529}%
\pgfsetfillcolor{currentfill}%
\pgfsetlinewidth{0.000000pt}%
\definecolor{currentstroke}{rgb}{0.000000,0.000000,0.000000}%
\pgfsetstrokecolor{currentstroke}%
\pgfsetstrokeopacity{0.000000}%
\pgfsetdash{}{0pt}%
\pgfpathmoveto{\pgfqpoint{3.136452in}{0.778861in}}%
\pgfpathlineto{\pgfqpoint{3.203963in}{0.778861in}}%
\pgfpathlineto{\pgfqpoint{3.203963in}{1.891346in}}%
\pgfpathlineto{\pgfqpoint{3.136452in}{1.891346in}}%
\pgfpathlineto{\pgfqpoint{3.136452in}{0.778861in}}%
\pgfpathclose%
\pgfusepath{fill}%
\end{pgfscope}%
\begin{pgfscope}%
\pgfpathrectangle{\pgfqpoint{2.469777in}{0.778861in}}{\pgfqpoint{2.025341in}{1.681851in}}%
\pgfusepath{clip}%
\pgfsetbuttcap%
\pgfsetmiterjoin%
\definecolor{currentfill}{rgb}{0.464706,0.320588,0.573529}%
\pgfsetfillcolor{currentfill}%
\pgfsetlinewidth{0.000000pt}%
\definecolor{currentstroke}{rgb}{0.000000,0.000000,0.000000}%
\pgfsetstrokecolor{currentstroke}%
\pgfsetstrokeopacity{0.000000}%
\pgfsetdash{}{0pt}%
\pgfpathmoveto{\pgfqpoint{3.389619in}{0.778861in}}%
\pgfpathlineto{\pgfqpoint{3.457131in}{0.778861in}}%
\pgfpathlineto{\pgfqpoint{3.457131in}{1.570358in}}%
\pgfpathlineto{\pgfqpoint{3.389619in}{1.570358in}}%
\pgfpathlineto{\pgfqpoint{3.389619in}{0.778861in}}%
\pgfpathclose%
\pgfusepath{fill}%
\end{pgfscope}%
\begin{pgfscope}%
\pgfpathrectangle{\pgfqpoint{2.469777in}{0.778861in}}{\pgfqpoint{2.025341in}{1.681851in}}%
\pgfusepath{clip}%
\pgfsetbuttcap%
\pgfsetmiterjoin%
\definecolor{currentfill}{rgb}{0.464706,0.320588,0.573529}%
\pgfsetfillcolor{currentfill}%
\pgfsetlinewidth{0.000000pt}%
\definecolor{currentstroke}{rgb}{0.000000,0.000000,0.000000}%
\pgfsetstrokecolor{currentstroke}%
\pgfsetstrokeopacity{0.000000}%
\pgfsetdash{}{0pt}%
\pgfpathmoveto{\pgfqpoint{3.642787in}{0.778861in}}%
\pgfpathlineto{\pgfqpoint{3.710298in}{0.778861in}}%
\pgfpathlineto{\pgfqpoint{3.710298in}{1.770823in}}%
\pgfpathlineto{\pgfqpoint{3.642787in}{1.770823in}}%
\pgfpathlineto{\pgfqpoint{3.642787in}{0.778861in}}%
\pgfpathclose%
\pgfusepath{fill}%
\end{pgfscope}%
\begin{pgfscope}%
\pgfpathrectangle{\pgfqpoint{2.469777in}{0.778861in}}{\pgfqpoint{2.025341in}{1.681851in}}%
\pgfusepath{clip}%
\pgfsetbuttcap%
\pgfsetmiterjoin%
\definecolor{currentfill}{rgb}{0.464706,0.320588,0.573529}%
\pgfsetfillcolor{currentfill}%
\pgfsetlinewidth{0.000000pt}%
\definecolor{currentstroke}{rgb}{0.000000,0.000000,0.000000}%
\pgfsetstrokecolor{currentstroke}%
\pgfsetstrokeopacity{0.000000}%
\pgfsetdash{}{0pt}%
\pgfpathmoveto{\pgfqpoint{3.895955in}{0.778861in}}%
\pgfpathlineto{\pgfqpoint{3.963466in}{0.778861in}}%
\pgfpathlineto{\pgfqpoint{3.963466in}{1.681100in}}%
\pgfpathlineto{\pgfqpoint{3.895955in}{1.681100in}}%
\pgfpathlineto{\pgfqpoint{3.895955in}{0.778861in}}%
\pgfpathclose%
\pgfusepath{fill}%
\end{pgfscope}%
\begin{pgfscope}%
\pgfpathrectangle{\pgfqpoint{2.469777in}{0.778861in}}{\pgfqpoint{2.025341in}{1.681851in}}%
\pgfusepath{clip}%
\pgfsetbuttcap%
\pgfsetmiterjoin%
\definecolor{currentfill}{rgb}{0.464706,0.320588,0.573529}%
\pgfsetfillcolor{currentfill}%
\pgfsetlinewidth{0.000000pt}%
\definecolor{currentstroke}{rgb}{0.000000,0.000000,0.000000}%
\pgfsetstrokecolor{currentstroke}%
\pgfsetstrokeopacity{0.000000}%
\pgfsetdash{}{0pt}%
\pgfpathmoveto{\pgfqpoint{4.149122in}{0.778861in}}%
\pgfpathlineto{\pgfqpoint{4.216634in}{0.778861in}}%
\pgfpathlineto{\pgfqpoint{4.216634in}{1.733756in}}%
\pgfpathlineto{\pgfqpoint{4.149122in}{1.733756in}}%
\pgfpathlineto{\pgfqpoint{4.149122in}{0.778861in}}%
\pgfpathclose%
\pgfusepath{fill}%
\end{pgfscope}%
\begin{pgfscope}%
\pgfpathrectangle{\pgfqpoint{2.469777in}{0.778861in}}{\pgfqpoint{2.025341in}{1.681851in}}%
\pgfusepath{clip}%
\pgfsetbuttcap%
\pgfsetmiterjoin%
\definecolor{currentfill}{rgb}{0.464706,0.320588,0.573529}%
\pgfsetfillcolor{currentfill}%
\pgfsetlinewidth{0.000000pt}%
\definecolor{currentstroke}{rgb}{0.000000,0.000000,0.000000}%
\pgfsetstrokecolor{currentstroke}%
\pgfsetstrokeopacity{0.000000}%
\pgfsetdash{}{0pt}%
\pgfpathmoveto{\pgfqpoint{4.402290in}{0.778861in}}%
\pgfpathlineto{\pgfqpoint{4.469801in}{0.778861in}}%
\pgfpathlineto{\pgfqpoint{4.469801in}{1.778284in}}%
\pgfpathlineto{\pgfqpoint{4.402290in}{1.778284in}}%
\pgfpathlineto{\pgfqpoint{4.402290in}{0.778861in}}%
\pgfpathclose%
\pgfusepath{fill}%
\end{pgfscope}%
\begin{pgfscope}%
\pgfsetbuttcap%
\pgfsetroundjoin%
\definecolor{currentfill}{rgb}{0.552941,0.501961,0.478431}%
\pgfsetfillcolor{currentfill}%
\pgfsetlinewidth{0.803000pt}%
\definecolor{currentstroke}{rgb}{0.552941,0.501961,0.478431}%
\pgfsetstrokecolor{currentstroke}%
\pgfsetdash{}{0pt}%
\pgfsys@defobject{currentmarker}{\pgfqpoint{0.000000in}{0.000000in}}{\pgfqpoint{0.000000in}{0.083333in}}{%
\pgfpathmoveto{\pgfqpoint{0.000000in}{0.000000in}}%
\pgfpathlineto{\pgfqpoint{0.000000in}{0.083333in}}%
\pgfusepath{stroke,fill}%
}%
\begin{pgfscope}%
\pgfsys@transformshift{2.596361in}{0.778861in}%
\pgfsys@useobject{currentmarker}{}%
\end{pgfscope}%
\end{pgfscope}%
\begin{pgfscope}%
\definecolor{textcolor}{rgb}{0.552941,0.501961,0.478431}%
\pgfsetstrokecolor{textcolor}%
\pgfsetfillcolor{textcolor}%
\pgftext[x=2.627611in, y=0.444468in, left, base,rotate=90.000000]{\color{textcolor}{\sffamily\fontsize{9.000000}{10.800000}\selectfont\catcode`\^=\active\def^{\ifmmode\sp\else\^{}\fi}\catcode`\%=\active\def%{\%}PA01}}%
\end{pgfscope}%
\begin{pgfscope}%
\pgfsetbuttcap%
\pgfsetroundjoin%
\definecolor{currentfill}{rgb}{0.552941,0.501961,0.478431}%
\pgfsetfillcolor{currentfill}%
\pgfsetlinewidth{0.803000pt}%
\definecolor{currentstroke}{rgb}{0.552941,0.501961,0.478431}%
\pgfsetstrokecolor{currentstroke}%
\pgfsetdash{}{0pt}%
\pgfsys@defobject{currentmarker}{\pgfqpoint{0.000000in}{0.000000in}}{\pgfqpoint{0.000000in}{0.083333in}}{%
\pgfpathmoveto{\pgfqpoint{0.000000in}{0.000000in}}%
\pgfpathlineto{\pgfqpoint{0.000000in}{0.083333in}}%
\pgfusepath{stroke,fill}%
}%
\begin{pgfscope}%
\pgfsys@transformshift{2.849528in}{0.778861in}%
\pgfsys@useobject{currentmarker}{}%
\end{pgfscope}%
\end{pgfscope}%
\begin{pgfscope}%
\definecolor{textcolor}{rgb}{0.552941,0.501961,0.478431}%
\pgfsetstrokecolor{textcolor}%
\pgfsetfillcolor{textcolor}%
\pgftext[x=2.880778in, y=0.433859in, left, base,rotate=90.000000]{\color{textcolor}{\sffamily\fontsize{9.000000}{10.800000}\selectfont\catcode`\^=\active\def^{\ifmmode\sp\else\^{}\fi}\catcode`\%=\active\def%{\%}PB01}}%
\end{pgfscope}%
\begin{pgfscope}%
\pgfsetbuttcap%
\pgfsetroundjoin%
\definecolor{currentfill}{rgb}{0.552941,0.501961,0.478431}%
\pgfsetfillcolor{currentfill}%
\pgfsetlinewidth{0.803000pt}%
\definecolor{currentstroke}{rgb}{0.552941,0.501961,0.478431}%
\pgfsetstrokecolor{currentstroke}%
\pgfsetdash{}{0pt}%
\pgfsys@defobject{currentmarker}{\pgfqpoint{0.000000in}{0.000000in}}{\pgfqpoint{0.000000in}{0.083333in}}{%
\pgfpathmoveto{\pgfqpoint{0.000000in}{0.000000in}}%
\pgfpathlineto{\pgfqpoint{0.000000in}{0.083333in}}%
\pgfusepath{stroke,fill}%
}%
\begin{pgfscope}%
\pgfsys@transformshift{3.102696in}{0.778861in}%
\pgfsys@useobject{currentmarker}{}%
\end{pgfscope}%
\end{pgfscope}%
\begin{pgfscope}%
\definecolor{textcolor}{rgb}{0.552941,0.501961,0.478431}%
\pgfsetstrokecolor{textcolor}%
\pgfsetfillcolor{textcolor}%
\pgftext[x=3.133946in, y=0.433859in, left, base,rotate=90.000000]{\color{textcolor}{\sffamily\fontsize{9.000000}{10.800000}\selectfont\catcode`\^=\active\def^{\ifmmode\sp\else\^{}\fi}\catcode`\%=\active\def%{\%}PB02}}%
\end{pgfscope}%
\begin{pgfscope}%
\pgfsetbuttcap%
\pgfsetroundjoin%
\definecolor{currentfill}{rgb}{0.552941,0.501961,0.478431}%
\pgfsetfillcolor{currentfill}%
\pgfsetlinewidth{0.803000pt}%
\definecolor{currentstroke}{rgb}{0.552941,0.501961,0.478431}%
\pgfsetstrokecolor{currentstroke}%
\pgfsetdash{}{0pt}%
\pgfsys@defobject{currentmarker}{\pgfqpoint{0.000000in}{0.000000in}}{\pgfqpoint{0.000000in}{0.083333in}}{%
\pgfpathmoveto{\pgfqpoint{0.000000in}{0.000000in}}%
\pgfpathlineto{\pgfqpoint{0.000000in}{0.083333in}}%
\pgfusepath{stroke,fill}%
}%
\begin{pgfscope}%
\pgfsys@transformshift{3.355864in}{0.778861in}%
\pgfsys@useobject{currentmarker}{}%
\end{pgfscope}%
\end{pgfscope}%
\begin{pgfscope}%
\definecolor{textcolor}{rgb}{0.552941,0.501961,0.478431}%
\pgfsetstrokecolor{textcolor}%
\pgfsetfillcolor{textcolor}%
\pgftext[x=3.387114in, y=0.433859in, left, base,rotate=90.000000]{\color{textcolor}{\sffamily\fontsize{9.000000}{10.800000}\selectfont\catcode`\^=\active\def^{\ifmmode\sp\else\^{}\fi}\catcode`\%=\active\def%{\%}PB04}}%
\end{pgfscope}%
\begin{pgfscope}%
\pgfsetbuttcap%
\pgfsetroundjoin%
\definecolor{currentfill}{rgb}{0.552941,0.501961,0.478431}%
\pgfsetfillcolor{currentfill}%
\pgfsetlinewidth{0.803000pt}%
\definecolor{currentstroke}{rgb}{0.552941,0.501961,0.478431}%
\pgfsetstrokecolor{currentstroke}%
\pgfsetdash{}{0pt}%
\pgfsys@defobject{currentmarker}{\pgfqpoint{0.000000in}{0.000000in}}{\pgfqpoint{0.000000in}{0.083333in}}{%
\pgfpathmoveto{\pgfqpoint{0.000000in}{0.000000in}}%
\pgfpathlineto{\pgfqpoint{0.000000in}{0.083333in}}%
\pgfusepath{stroke,fill}%
}%
\begin{pgfscope}%
\pgfsys@transformshift{3.609031in}{0.778861in}%
\pgfsys@useobject{currentmarker}{}%
\end{pgfscope}%
\end{pgfscope}%
\begin{pgfscope}%
\definecolor{textcolor}{rgb}{0.552941,0.501961,0.478431}%
\pgfsetstrokecolor{textcolor}%
\pgfsetfillcolor{textcolor}%
\pgftext[x=3.640281in, y=0.437524in, left, base,rotate=90.000000]{\color{textcolor}{\sffamily\fontsize{9.000000}{10.800000}\selectfont\catcode`\^=\active\def^{\ifmmode\sp\else\^{}\fi}\catcode`\%=\active\def%{\%}PC02}}%
\end{pgfscope}%
\begin{pgfscope}%
\pgfsetbuttcap%
\pgfsetroundjoin%
\definecolor{currentfill}{rgb}{0.552941,0.501961,0.478431}%
\pgfsetfillcolor{currentfill}%
\pgfsetlinewidth{0.803000pt}%
\definecolor{currentstroke}{rgb}{0.552941,0.501961,0.478431}%
\pgfsetstrokecolor{currentstroke}%
\pgfsetdash{}{0pt}%
\pgfsys@defobject{currentmarker}{\pgfqpoint{0.000000in}{0.000000in}}{\pgfqpoint{0.000000in}{0.083333in}}{%
\pgfpathmoveto{\pgfqpoint{0.000000in}{0.000000in}}%
\pgfpathlineto{\pgfqpoint{0.000000in}{0.083333in}}%
\pgfusepath{stroke,fill}%
}%
\begin{pgfscope}%
\pgfsys@transformshift{3.862199in}{0.778861in}%
\pgfsys@useobject{currentmarker}{}%
\end{pgfscope}%
\end{pgfscope}%
\begin{pgfscope}%
\definecolor{textcolor}{rgb}{0.552941,0.501961,0.478431}%
\pgfsetstrokecolor{textcolor}%
\pgfsetfillcolor{textcolor}%
\pgftext[x=3.893449in, y=0.437524in, left, base,rotate=90.000000]{\color{textcolor}{\sffamily\fontsize{9.000000}{10.800000}\selectfont\catcode`\^=\active\def^{\ifmmode\sp\else\^{}\fi}\catcode`\%=\active\def%{\%}PC03}}%
\end{pgfscope}%
\begin{pgfscope}%
\pgfsetbuttcap%
\pgfsetroundjoin%
\definecolor{currentfill}{rgb}{0.552941,0.501961,0.478431}%
\pgfsetfillcolor{currentfill}%
\pgfsetlinewidth{0.803000pt}%
\definecolor{currentstroke}{rgb}{0.552941,0.501961,0.478431}%
\pgfsetstrokecolor{currentstroke}%
\pgfsetdash{}{0pt}%
\pgfsys@defobject{currentmarker}{\pgfqpoint{0.000000in}{0.000000in}}{\pgfqpoint{0.000000in}{0.083333in}}{%
\pgfpathmoveto{\pgfqpoint{0.000000in}{0.000000in}}%
\pgfpathlineto{\pgfqpoint{0.000000in}{0.083333in}}%
\pgfusepath{stroke,fill}%
}%
\begin{pgfscope}%
\pgfsys@transformshift{4.115367in}{0.778861in}%
\pgfsys@useobject{currentmarker}{}%
\end{pgfscope}%
\end{pgfscope}%
\begin{pgfscope}%
\definecolor{textcolor}{rgb}{0.552941,0.501961,0.478431}%
\pgfsetstrokecolor{textcolor}%
\pgfsetfillcolor{textcolor}%
\pgftext[x=4.146617in, y=0.437524in, left, base,rotate=90.000000]{\color{textcolor}{\sffamily\fontsize{9.000000}{10.800000}\selectfont\catcode`\^=\active\def^{\ifmmode\sp\else\^{}\fi}\catcode`\%=\active\def%{\%}PC04}}%
\end{pgfscope}%
\begin{pgfscope}%
\pgfsetbuttcap%
\pgfsetroundjoin%
\definecolor{currentfill}{rgb}{0.552941,0.501961,0.478431}%
\pgfsetfillcolor{currentfill}%
\pgfsetlinewidth{0.803000pt}%
\definecolor{currentstroke}{rgb}{0.552941,0.501961,0.478431}%
\pgfsetstrokecolor{currentstroke}%
\pgfsetdash{}{0pt}%
\pgfsys@defobject{currentmarker}{\pgfqpoint{0.000000in}{0.000000in}}{\pgfqpoint{0.000000in}{0.083333in}}{%
\pgfpathmoveto{\pgfqpoint{0.000000in}{0.000000in}}%
\pgfpathlineto{\pgfqpoint{0.000000in}{0.083333in}}%
\pgfusepath{stroke,fill}%
}%
\begin{pgfscope}%
\pgfsys@transformshift{4.368534in}{0.778861in}%
\pgfsys@useobject{currentmarker}{}%
\end{pgfscope}%
\end{pgfscope}%
\begin{pgfscope}%
\definecolor{textcolor}{rgb}{0.552941,0.501961,0.478431}%
\pgfsetstrokecolor{textcolor}%
\pgfsetfillcolor{textcolor}%
\pgftext[x=4.399784in, y=0.509380in, left, base,rotate=90.000000]{\color{textcolor}{\sffamily\fontsize{9.000000}{10.800000}\selectfont\catcode`\^=\active\def^{\ifmmode\sp\else\^{}\fi}\catcode`\%=\active\def%{\%}avg.}}%
\end{pgfscope}%
\begin{pgfscope}%
\definecolor{textcolor}{rgb}{0.552941,0.501961,0.478431}%
\pgfsetstrokecolor{textcolor}%
\pgfsetfillcolor{textcolor}%
\pgftext[x=3.482447in,y=0.378303in,,top]{\color{textcolor}{\sffamily\fontsize{9.000000}{10.800000}\selectfont\catcode`\^=\active\def^{\ifmmode\sp\else\^{}\fi}\catcode`\%=\active\def%{\%}patient}}%
\end{pgfscope}%
\begin{pgfscope}%
\pgfsetbuttcap%
\pgfsetroundjoin%
\definecolor{currentfill}{rgb}{0.552941,0.501961,0.478431}%
\pgfsetfillcolor{currentfill}%
\pgfsetlinewidth{0.803000pt}%
\definecolor{currentstroke}{rgb}{0.552941,0.501961,0.478431}%
\pgfsetstrokecolor{currentstroke}%
\pgfsetdash{}{0pt}%
\pgfsys@defobject{currentmarker}{\pgfqpoint{0.000000in}{0.000000in}}{\pgfqpoint{0.083333in}{0.000000in}}{%
\pgfpathmoveto{\pgfqpoint{0.000000in}{0.000000in}}%
\pgfpathlineto{\pgfqpoint{0.083333in}{0.000000in}}%
\pgfusepath{stroke,fill}%
}%
\begin{pgfscope}%
\pgfsys@transformshift{2.469777in}{0.778861in}%
\pgfsys@useobject{currentmarker}{}%
\end{pgfscope}%
\end{pgfscope}%
\begin{pgfscope}%
\pgfsetbuttcap%
\pgfsetroundjoin%
\definecolor{currentfill}{rgb}{0.552941,0.501961,0.478431}%
\pgfsetfillcolor{currentfill}%
\pgfsetlinewidth{0.803000pt}%
\definecolor{currentstroke}{rgb}{0.552941,0.501961,0.478431}%
\pgfsetstrokecolor{currentstroke}%
\pgfsetdash{}{0pt}%
\pgfsys@defobject{currentmarker}{\pgfqpoint{0.000000in}{0.000000in}}{\pgfqpoint{0.083333in}{0.000000in}}{%
\pgfpathmoveto{\pgfqpoint{0.000000in}{0.000000in}}%
\pgfpathlineto{\pgfqpoint{0.083333in}{0.000000in}}%
\pgfusepath{stroke,fill}%
}%
\begin{pgfscope}%
\pgfsys@transformshift{2.469777in}{1.115231in}%
\pgfsys@useobject{currentmarker}{}%
\end{pgfscope}%
\end{pgfscope}%
\begin{pgfscope}%
\pgfsetbuttcap%
\pgfsetroundjoin%
\definecolor{currentfill}{rgb}{0.552941,0.501961,0.478431}%
\pgfsetfillcolor{currentfill}%
\pgfsetlinewidth{0.803000pt}%
\definecolor{currentstroke}{rgb}{0.552941,0.501961,0.478431}%
\pgfsetstrokecolor{currentstroke}%
\pgfsetdash{}{0pt}%
\pgfsys@defobject{currentmarker}{\pgfqpoint{0.000000in}{0.000000in}}{\pgfqpoint{0.083333in}{0.000000in}}{%
\pgfpathmoveto{\pgfqpoint{0.000000in}{0.000000in}}%
\pgfpathlineto{\pgfqpoint{0.083333in}{0.000000in}}%
\pgfusepath{stroke,fill}%
}%
\begin{pgfscope}%
\pgfsys@transformshift{2.469777in}{1.451601in}%
\pgfsys@useobject{currentmarker}{}%
\end{pgfscope}%
\end{pgfscope}%
\begin{pgfscope}%
\pgfsetbuttcap%
\pgfsetroundjoin%
\definecolor{currentfill}{rgb}{0.552941,0.501961,0.478431}%
\pgfsetfillcolor{currentfill}%
\pgfsetlinewidth{0.803000pt}%
\definecolor{currentstroke}{rgb}{0.552941,0.501961,0.478431}%
\pgfsetstrokecolor{currentstroke}%
\pgfsetdash{}{0pt}%
\pgfsys@defobject{currentmarker}{\pgfqpoint{0.000000in}{0.000000in}}{\pgfqpoint{0.083333in}{0.000000in}}{%
\pgfpathmoveto{\pgfqpoint{0.000000in}{0.000000in}}%
\pgfpathlineto{\pgfqpoint{0.083333in}{0.000000in}}%
\pgfusepath{stroke,fill}%
}%
\begin{pgfscope}%
\pgfsys@transformshift{2.469777in}{1.787971in}%
\pgfsys@useobject{currentmarker}{}%
\end{pgfscope}%
\end{pgfscope}%
\begin{pgfscope}%
\pgfsetbuttcap%
\pgfsetroundjoin%
\definecolor{currentfill}{rgb}{0.552941,0.501961,0.478431}%
\pgfsetfillcolor{currentfill}%
\pgfsetlinewidth{0.803000pt}%
\definecolor{currentstroke}{rgb}{0.552941,0.501961,0.478431}%
\pgfsetstrokecolor{currentstroke}%
\pgfsetdash{}{0pt}%
\pgfsys@defobject{currentmarker}{\pgfqpoint{0.000000in}{0.000000in}}{\pgfqpoint{0.083333in}{0.000000in}}{%
\pgfpathmoveto{\pgfqpoint{0.000000in}{0.000000in}}%
\pgfpathlineto{\pgfqpoint{0.083333in}{0.000000in}}%
\pgfusepath{stroke,fill}%
}%
\begin{pgfscope}%
\pgfsys@transformshift{2.469777in}{2.124341in}%
\pgfsys@useobject{currentmarker}{}%
\end{pgfscope}%
\end{pgfscope}%
\begin{pgfscope}%
\pgfsetbuttcap%
\pgfsetroundjoin%
\definecolor{currentfill}{rgb}{0.552941,0.501961,0.478431}%
\pgfsetfillcolor{currentfill}%
\pgfsetlinewidth{0.803000pt}%
\definecolor{currentstroke}{rgb}{0.552941,0.501961,0.478431}%
\pgfsetstrokecolor{currentstroke}%
\pgfsetdash{}{0pt}%
\pgfsys@defobject{currentmarker}{\pgfqpoint{0.000000in}{0.000000in}}{\pgfqpoint{0.083333in}{0.000000in}}{%
\pgfpathmoveto{\pgfqpoint{0.000000in}{0.000000in}}%
\pgfpathlineto{\pgfqpoint{0.083333in}{0.000000in}}%
\pgfusepath{stroke,fill}%
}%
\begin{pgfscope}%
\pgfsys@transformshift{2.469777in}{2.460711in}%
\pgfsys@useobject{currentmarker}{}%
\end{pgfscope}%
\end{pgfscope}%
\begin{pgfscope}%
\pgfpathrectangle{\pgfqpoint{2.469777in}{0.778861in}}{\pgfqpoint{2.025341in}{1.681851in}}%
\pgfusepath{clip}%
\pgfsetrectcap%
\pgfsetroundjoin%
\pgfsetlinewidth{2.258437pt}%
\definecolor{currentstroke}{rgb}{0.260000,0.260000,0.260000}%
\pgfsetstrokecolor{currentstroke}%
\pgfsetdash{}{0pt}%
\pgfpathmoveto{\pgfqpoint{2.528849in}{1.680385in}}%
\pgfpathlineto{\pgfqpoint{2.528849in}{1.886833in}}%
\pgfusepath{stroke}%
\end{pgfscope}%
\begin{pgfscope}%
\pgfpathrectangle{\pgfqpoint{2.469777in}{0.778861in}}{\pgfqpoint{2.025341in}{1.681851in}}%
\pgfusepath{clip}%
\pgfsetrectcap%
\pgfsetroundjoin%
\pgfsetlinewidth{2.258437pt}%
\definecolor{currentstroke}{rgb}{0.260000,0.260000,0.260000}%
\pgfsetstrokecolor{currentstroke}%
\pgfsetdash{}{0pt}%
\pgfpathmoveto{\pgfqpoint{2.782017in}{1.848994in}}%
\pgfpathlineto{\pgfqpoint{2.782017in}{2.032310in}}%
\pgfusepath{stroke}%
\end{pgfscope}%
\begin{pgfscope}%
\pgfpathrectangle{\pgfqpoint{2.469777in}{0.778861in}}{\pgfqpoint{2.025341in}{1.681851in}}%
\pgfusepath{clip}%
\pgfsetrectcap%
\pgfsetroundjoin%
\pgfsetlinewidth{2.258437pt}%
\definecolor{currentstroke}{rgb}{0.260000,0.260000,0.260000}%
\pgfsetstrokecolor{currentstroke}%
\pgfsetdash{}{0pt}%
\pgfpathmoveto{\pgfqpoint{3.035185in}{1.712296in}}%
\pgfpathlineto{\pgfqpoint{3.035185in}{1.928044in}}%
\pgfusepath{stroke}%
\end{pgfscope}%
\begin{pgfscope}%
\pgfpathrectangle{\pgfqpoint{2.469777in}{0.778861in}}{\pgfqpoint{2.025341in}{1.681851in}}%
\pgfusepath{clip}%
\pgfsetrectcap%
\pgfsetroundjoin%
\pgfsetlinewidth{2.258437pt}%
\definecolor{currentstroke}{rgb}{0.260000,0.260000,0.260000}%
\pgfsetstrokecolor{currentstroke}%
\pgfsetdash{}{0pt}%
\pgfpathmoveto{\pgfqpoint{3.288352in}{1.560570in}}%
\pgfpathlineto{\pgfqpoint{3.288352in}{1.744861in}}%
\pgfusepath{stroke}%
\end{pgfscope}%
\begin{pgfscope}%
\pgfpathrectangle{\pgfqpoint{2.469777in}{0.778861in}}{\pgfqpoint{2.025341in}{1.681851in}}%
\pgfusepath{clip}%
\pgfsetrectcap%
\pgfsetroundjoin%
\pgfsetlinewidth{2.258437pt}%
\definecolor{currentstroke}{rgb}{0.260000,0.260000,0.260000}%
\pgfsetstrokecolor{currentstroke}%
\pgfsetdash{}{0pt}%
\pgfpathmoveto{\pgfqpoint{3.541520in}{1.662432in}}%
\pgfpathlineto{\pgfqpoint{3.541520in}{1.839502in}}%
\pgfusepath{stroke}%
\end{pgfscope}%
\begin{pgfscope}%
\pgfpathrectangle{\pgfqpoint{2.469777in}{0.778861in}}{\pgfqpoint{2.025341in}{1.681851in}}%
\pgfusepath{clip}%
\pgfsetrectcap%
\pgfsetroundjoin%
\pgfsetlinewidth{2.258437pt}%
\definecolor{currentstroke}{rgb}{0.260000,0.260000,0.260000}%
\pgfsetstrokecolor{currentstroke}%
\pgfsetdash{}{0pt}%
\pgfpathmoveto{\pgfqpoint{3.794688in}{1.490552in}}%
\pgfpathlineto{\pgfqpoint{3.794688in}{1.726170in}}%
\pgfusepath{stroke}%
\end{pgfscope}%
\begin{pgfscope}%
\pgfpathrectangle{\pgfqpoint{2.469777in}{0.778861in}}{\pgfqpoint{2.025341in}{1.681851in}}%
\pgfusepath{clip}%
\pgfsetrectcap%
\pgfsetroundjoin%
\pgfsetlinewidth{2.258437pt}%
\definecolor{currentstroke}{rgb}{0.260000,0.260000,0.260000}%
\pgfsetstrokecolor{currentstroke}%
\pgfsetdash{}{0pt}%
\pgfpathmoveto{\pgfqpoint{4.047855in}{1.751692in}}%
\pgfpathlineto{\pgfqpoint{4.047855in}{1.894040in}}%
\pgfusepath{stroke}%
\end{pgfscope}%
\begin{pgfscope}%
\pgfpathrectangle{\pgfqpoint{2.469777in}{0.778861in}}{\pgfqpoint{2.025341in}{1.681851in}}%
\pgfusepath{clip}%
\pgfsetrectcap%
\pgfsetroundjoin%
\pgfsetlinewidth{2.258437pt}%
\definecolor{currentstroke}{rgb}{0.260000,0.260000,0.260000}%
\pgfsetstrokecolor{currentstroke}%
\pgfsetdash{}{0pt}%
\pgfpathmoveto{\pgfqpoint{4.301023in}{1.728453in}}%
\pgfpathlineto{\pgfqpoint{4.301023in}{1.816247in}}%
\pgfusepath{stroke}%
\end{pgfscope}%
\begin{pgfscope}%
\pgfpathrectangle{\pgfqpoint{2.469777in}{0.778861in}}{\pgfqpoint{2.025341in}{1.681851in}}%
\pgfusepath{clip}%
\pgfsetrectcap%
\pgfsetroundjoin%
\pgfsetlinewidth{2.258437pt}%
\definecolor{currentstroke}{rgb}{0.260000,0.260000,0.260000}%
\pgfsetstrokecolor{currentstroke}%
\pgfsetdash{}{0pt}%
\pgfpathmoveto{\pgfqpoint{2.596361in}{1.708570in}}%
\pgfpathlineto{\pgfqpoint{2.596361in}{1.930512in}}%
\pgfusepath{stroke}%
\end{pgfscope}%
\begin{pgfscope}%
\pgfpathrectangle{\pgfqpoint{2.469777in}{0.778861in}}{\pgfqpoint{2.025341in}{1.681851in}}%
\pgfusepath{clip}%
\pgfsetrectcap%
\pgfsetroundjoin%
\pgfsetlinewidth{2.258437pt}%
\definecolor{currentstroke}{rgb}{0.260000,0.260000,0.260000}%
\pgfsetstrokecolor{currentstroke}%
\pgfsetdash{}{0pt}%
\pgfpathmoveto{\pgfqpoint{2.849528in}{1.944217in}}%
\pgfpathlineto{\pgfqpoint{2.849528in}{2.131063in}}%
\pgfusepath{stroke}%
\end{pgfscope}%
\begin{pgfscope}%
\pgfpathrectangle{\pgfqpoint{2.469777in}{0.778861in}}{\pgfqpoint{2.025341in}{1.681851in}}%
\pgfusepath{clip}%
\pgfsetrectcap%
\pgfsetroundjoin%
\pgfsetlinewidth{2.258437pt}%
\definecolor{currentstroke}{rgb}{0.260000,0.260000,0.260000}%
\pgfsetstrokecolor{currentstroke}%
\pgfsetdash{}{0pt}%
\pgfpathmoveto{\pgfqpoint{3.102696in}{1.871683in}}%
\pgfpathlineto{\pgfqpoint{3.102696in}{2.067090in}}%
\pgfusepath{stroke}%
\end{pgfscope}%
\begin{pgfscope}%
\pgfpathrectangle{\pgfqpoint{2.469777in}{0.778861in}}{\pgfqpoint{2.025341in}{1.681851in}}%
\pgfusepath{clip}%
\pgfsetrectcap%
\pgfsetroundjoin%
\pgfsetlinewidth{2.258437pt}%
\definecolor{currentstroke}{rgb}{0.260000,0.260000,0.260000}%
\pgfsetstrokecolor{currentstroke}%
\pgfsetdash{}{0pt}%
\pgfpathmoveto{\pgfqpoint{3.355864in}{1.488487in}}%
\pgfpathlineto{\pgfqpoint{3.355864in}{1.712538in}}%
\pgfusepath{stroke}%
\end{pgfscope}%
\begin{pgfscope}%
\pgfpathrectangle{\pgfqpoint{2.469777in}{0.778861in}}{\pgfqpoint{2.025341in}{1.681851in}}%
\pgfusepath{clip}%
\pgfsetrectcap%
\pgfsetroundjoin%
\pgfsetlinewidth{2.258437pt}%
\definecolor{currentstroke}{rgb}{0.260000,0.260000,0.260000}%
\pgfsetstrokecolor{currentstroke}%
\pgfsetdash{}{0pt}%
\pgfpathmoveto{\pgfqpoint{3.609031in}{1.744246in}}%
\pgfpathlineto{\pgfqpoint{3.609031in}{1.924697in}}%
\pgfusepath{stroke}%
\end{pgfscope}%
\begin{pgfscope}%
\pgfpathrectangle{\pgfqpoint{2.469777in}{0.778861in}}{\pgfqpoint{2.025341in}{1.681851in}}%
\pgfusepath{clip}%
\pgfsetrectcap%
\pgfsetroundjoin%
\pgfsetlinewidth{2.258437pt}%
\definecolor{currentstroke}{rgb}{0.260000,0.260000,0.260000}%
\pgfsetstrokecolor{currentstroke}%
\pgfsetdash{}{0pt}%
\pgfpathmoveto{\pgfqpoint{3.862199in}{1.543956in}}%
\pgfpathlineto{\pgfqpoint{3.862199in}{1.803264in}}%
\pgfusepath{stroke}%
\end{pgfscope}%
\begin{pgfscope}%
\pgfpathrectangle{\pgfqpoint{2.469777in}{0.778861in}}{\pgfqpoint{2.025341in}{1.681851in}}%
\pgfusepath{clip}%
\pgfsetrectcap%
\pgfsetroundjoin%
\pgfsetlinewidth{2.258437pt}%
\definecolor{currentstroke}{rgb}{0.260000,0.260000,0.260000}%
\pgfsetstrokecolor{currentstroke}%
\pgfsetdash{}{0pt}%
\pgfpathmoveto{\pgfqpoint{4.115367in}{1.774128in}}%
\pgfpathlineto{\pgfqpoint{4.115367in}{2.001647in}}%
\pgfusepath{stroke}%
\end{pgfscope}%
\begin{pgfscope}%
\pgfpathrectangle{\pgfqpoint{2.469777in}{0.778861in}}{\pgfqpoint{2.025341in}{1.681851in}}%
\pgfusepath{clip}%
\pgfsetrectcap%
\pgfsetroundjoin%
\pgfsetlinewidth{2.258437pt}%
\definecolor{currentstroke}{rgb}{0.260000,0.260000,0.260000}%
\pgfsetstrokecolor{currentstroke}%
\pgfsetdash{}{0pt}%
\pgfpathmoveto{\pgfqpoint{4.368534in}{1.775511in}}%
\pgfpathlineto{\pgfqpoint{4.368534in}{1.881612in}}%
\pgfusepath{stroke}%
\end{pgfscope}%
\begin{pgfscope}%
\pgfpathrectangle{\pgfqpoint{2.469777in}{0.778861in}}{\pgfqpoint{2.025341in}{1.681851in}}%
\pgfusepath{clip}%
\pgfsetrectcap%
\pgfsetroundjoin%
\pgfsetlinewidth{2.258437pt}%
\definecolor{currentstroke}{rgb}{0.260000,0.260000,0.260000}%
\pgfsetstrokecolor{currentstroke}%
\pgfsetdash{}{0pt}%
\pgfpathmoveto{\pgfqpoint{2.663872in}{1.724156in}}%
\pgfpathlineto{\pgfqpoint{2.663872in}{1.942243in}}%
\pgfusepath{stroke}%
\end{pgfscope}%
\begin{pgfscope}%
\pgfpathrectangle{\pgfqpoint{2.469777in}{0.778861in}}{\pgfqpoint{2.025341in}{1.681851in}}%
\pgfusepath{clip}%
\pgfsetrectcap%
\pgfsetroundjoin%
\pgfsetlinewidth{2.258437pt}%
\definecolor{currentstroke}{rgb}{0.260000,0.260000,0.260000}%
\pgfsetstrokecolor{currentstroke}%
\pgfsetdash{}{0pt}%
\pgfpathmoveto{\pgfqpoint{2.917040in}{1.825716in}}%
\pgfpathlineto{\pgfqpoint{2.917040in}{2.067768in}}%
\pgfusepath{stroke}%
\end{pgfscope}%
\begin{pgfscope}%
\pgfpathrectangle{\pgfqpoint{2.469777in}{0.778861in}}{\pgfqpoint{2.025341in}{1.681851in}}%
\pgfusepath{clip}%
\pgfsetrectcap%
\pgfsetroundjoin%
\pgfsetlinewidth{2.258437pt}%
\definecolor{currentstroke}{rgb}{0.260000,0.260000,0.260000}%
\pgfsetstrokecolor{currentstroke}%
\pgfsetdash{}{0pt}%
\pgfpathmoveto{\pgfqpoint{3.170207in}{1.780935in}}%
\pgfpathlineto{\pgfqpoint{3.170207in}{2.012954in}}%
\pgfusepath{stroke}%
\end{pgfscope}%
\begin{pgfscope}%
\pgfpathrectangle{\pgfqpoint{2.469777in}{0.778861in}}{\pgfqpoint{2.025341in}{1.681851in}}%
\pgfusepath{clip}%
\pgfsetrectcap%
\pgfsetroundjoin%
\pgfsetlinewidth{2.258437pt}%
\definecolor{currentstroke}{rgb}{0.260000,0.260000,0.260000}%
\pgfsetstrokecolor{currentstroke}%
\pgfsetdash{}{0pt}%
\pgfpathmoveto{\pgfqpoint{3.423375in}{1.472015in}}%
\pgfpathlineto{\pgfqpoint{3.423375in}{1.669575in}}%
\pgfusepath{stroke}%
\end{pgfscope}%
\begin{pgfscope}%
\pgfpathrectangle{\pgfqpoint{2.469777in}{0.778861in}}{\pgfqpoint{2.025341in}{1.681851in}}%
\pgfusepath{clip}%
\pgfsetrectcap%
\pgfsetroundjoin%
\pgfsetlinewidth{2.258437pt}%
\definecolor{currentstroke}{rgb}{0.260000,0.260000,0.260000}%
\pgfsetstrokecolor{currentstroke}%
\pgfsetdash{}{0pt}%
\pgfpathmoveto{\pgfqpoint{3.676543in}{1.665430in}}%
\pgfpathlineto{\pgfqpoint{3.676543in}{1.872311in}}%
\pgfusepath{stroke}%
\end{pgfscope}%
\begin{pgfscope}%
\pgfpathrectangle{\pgfqpoint{2.469777in}{0.778861in}}{\pgfqpoint{2.025341in}{1.681851in}}%
\pgfusepath{clip}%
\pgfsetrectcap%
\pgfsetroundjoin%
\pgfsetlinewidth{2.258437pt}%
\definecolor{currentstroke}{rgb}{0.260000,0.260000,0.260000}%
\pgfsetstrokecolor{currentstroke}%
\pgfsetdash{}{0pt}%
\pgfpathmoveto{\pgfqpoint{3.929710in}{1.503011in}}%
\pgfpathlineto{\pgfqpoint{3.929710in}{1.841672in}}%
\pgfusepath{stroke}%
\end{pgfscope}%
\begin{pgfscope}%
\pgfpathrectangle{\pgfqpoint{2.469777in}{0.778861in}}{\pgfqpoint{2.025341in}{1.681851in}}%
\pgfusepath{clip}%
\pgfsetrectcap%
\pgfsetroundjoin%
\pgfsetlinewidth{2.258437pt}%
\definecolor{currentstroke}{rgb}{0.260000,0.260000,0.260000}%
\pgfsetstrokecolor{currentstroke}%
\pgfsetdash{}{0pt}%
\pgfpathmoveto{\pgfqpoint{4.182878in}{1.646806in}}%
\pgfpathlineto{\pgfqpoint{4.182878in}{1.803122in}}%
\pgfusepath{stroke}%
\end{pgfscope}%
\begin{pgfscope}%
\pgfpathrectangle{\pgfqpoint{2.469777in}{0.778861in}}{\pgfqpoint{2.025341in}{1.681851in}}%
\pgfusepath{clip}%
\pgfsetrectcap%
\pgfsetroundjoin%
\pgfsetlinewidth{2.258437pt}%
\definecolor{currentstroke}{rgb}{0.260000,0.260000,0.260000}%
\pgfsetstrokecolor{currentstroke}%
\pgfsetdash{}{0pt}%
\pgfpathmoveto{\pgfqpoint{4.436046in}{1.727123in}}%
\pgfpathlineto{\pgfqpoint{4.436046in}{1.834010in}}%
\pgfusepath{stroke}%
\end{pgfscope}%
\begin{pgfscope}%
\pgfpathrectangle{\pgfqpoint{2.469777in}{0.778861in}}{\pgfqpoint{2.025341in}{1.681851in}}%
\pgfusepath{clip}%
\pgfsetbuttcap%
\pgfsetroundjoin%
\pgfsetlinewidth{1.003750pt}%
\definecolor{currentstroke}{rgb}{0.552941,0.501961,0.478431}%
\pgfsetstrokecolor{currentstroke}%
\pgfsetdash{{3.700000pt}{1.600000pt}}{0.000000pt}%
\pgfpathmoveto{\pgfqpoint{2.469777in}{1.619786in}}%
\pgfpathlineto{\pgfqpoint{4.495118in}{1.619786in}}%
\pgfusepath{stroke}%
\end{pgfscope}%
\begin{pgfscope}%
\pgfsetrectcap%
\pgfsetmiterjoin%
\pgfsetlinewidth{0.803000pt}%
\definecolor{currentstroke}{rgb}{0.552941,0.501961,0.478431}%
\pgfsetstrokecolor{currentstroke}%
\pgfsetdash{}{0pt}%
\pgfpathmoveto{\pgfqpoint{2.469777in}{0.778861in}}%
\pgfpathlineto{\pgfqpoint{2.469777in}{2.460711in}}%
\pgfusepath{stroke}%
\end{pgfscope}%
\begin{pgfscope}%
\pgfsetrectcap%
\pgfsetmiterjoin%
\pgfsetlinewidth{0.803000pt}%
\definecolor{currentstroke}{rgb}{0.552941,0.501961,0.478431}%
\pgfsetstrokecolor{currentstroke}%
\pgfsetdash{}{0pt}%
\pgfpathmoveto{\pgfqpoint{2.469777in}{0.778861in}}%
\pgfpathlineto{\pgfqpoint{4.495118in}{0.778861in}}%
\pgfusepath{stroke}%
\end{pgfscope}%
\begin{pgfscope}%
\definecolor{textcolor}{rgb}{0.552941,0.501961,0.478431}%
\pgfsetstrokecolor{textcolor}%
\pgfsetfillcolor{textcolor}%
\pgftext[x=2.469777in,y=2.544045in,left,base]{\color{textcolor}{\sffamily\fontsize{9.000000}{10.800000}\selectfont\catcode`\^=\active\def^{\ifmmode\sp\else\^{}\fi}\catcode`\%=\active\def%{\%}covert VSA}}%
\end{pgfscope}%
\begin{pgfscope}%
\pgfsetbuttcap%
\pgfsetmiterjoin%
\pgfsetlinewidth{0.000000pt}%
\definecolor{currentstroke}{rgb}{0.000000,0.000000,0.000000}%
\pgfsetstrokecolor{currentstroke}%
\pgfsetstrokeopacity{0.000000}%
\pgfsetdash{}{0pt}%
\pgfpathmoveto{\pgfqpoint{4.560118in}{0.778861in}}%
\pgfpathlineto{\pgfqpoint{6.585459in}{0.778861in}}%
\pgfpathlineto{\pgfqpoint{6.585459in}{2.460711in}}%
\pgfpathlineto{\pgfqpoint{4.560118in}{2.460711in}}%
\pgfpathlineto{\pgfqpoint{4.560118in}{0.778861in}}%
\pgfpathclose%
\pgfusepath{}%
\end{pgfscope}%
\begin{pgfscope}%
\pgfpathrectangle{\pgfqpoint{4.560118in}{0.778861in}}{\pgfqpoint{2.025341in}{1.681851in}}%
\pgfusepath{clip}%
\pgfsetbuttcap%
\pgfsetmiterjoin%
\definecolor{currentfill}{rgb}{0.842157,0.553922,0.200980}%
\pgfsetfillcolor{currentfill}%
\pgfsetlinewidth{0.000000pt}%
\definecolor{currentstroke}{rgb}{0.000000,0.000000,0.000000}%
\pgfsetstrokecolor{currentstroke}%
\pgfsetstrokeopacity{0.000000}%
\pgfsetdash{}{0pt}%
\pgfpathmoveto{\pgfqpoint{4.585435in}{0.778861in}}%
\pgfpathlineto{\pgfqpoint{4.652946in}{0.778861in}}%
\pgfpathlineto{\pgfqpoint{4.652946in}{1.951773in}}%
\pgfpathlineto{\pgfqpoint{4.585435in}{1.951773in}}%
\pgfpathlineto{\pgfqpoint{4.585435in}{0.778861in}}%
\pgfpathclose%
\pgfusepath{fill}%
\end{pgfscope}%
\begin{pgfscope}%
\pgfpathrectangle{\pgfqpoint{4.560118in}{0.778861in}}{\pgfqpoint{2.025341in}{1.681851in}}%
\pgfusepath{clip}%
\pgfsetbuttcap%
\pgfsetmiterjoin%
\definecolor{currentfill}{rgb}{0.842157,0.553922,0.200980}%
\pgfsetfillcolor{currentfill}%
\pgfsetlinewidth{0.000000pt}%
\definecolor{currentstroke}{rgb}{0.000000,0.000000,0.000000}%
\pgfsetstrokecolor{currentstroke}%
\pgfsetstrokeopacity{0.000000}%
\pgfsetdash{}{0pt}%
\pgfpathmoveto{\pgfqpoint{4.838602in}{0.778861in}}%
\pgfpathlineto{\pgfqpoint{4.906114in}{0.778861in}}%
\pgfpathlineto{\pgfqpoint{4.906114in}{2.328202in}}%
\pgfpathlineto{\pgfqpoint{4.838602in}{2.328202in}}%
\pgfpathlineto{\pgfqpoint{4.838602in}{0.778861in}}%
\pgfpathclose%
\pgfusepath{fill}%
\end{pgfscope}%
\begin{pgfscope}%
\pgfpathrectangle{\pgfqpoint{4.560118in}{0.778861in}}{\pgfqpoint{2.025341in}{1.681851in}}%
\pgfusepath{clip}%
\pgfsetbuttcap%
\pgfsetmiterjoin%
\definecolor{currentfill}{rgb}{0.842157,0.553922,0.200980}%
\pgfsetfillcolor{currentfill}%
\pgfsetlinewidth{0.000000pt}%
\definecolor{currentstroke}{rgb}{0.000000,0.000000,0.000000}%
\pgfsetstrokecolor{currentstroke}%
\pgfsetstrokeopacity{0.000000}%
\pgfsetdash{}{0pt}%
\pgfpathmoveto{\pgfqpoint{5.091770in}{0.778861in}}%
\pgfpathlineto{\pgfqpoint{5.159281in}{0.778861in}}%
\pgfpathlineto{\pgfqpoint{5.159281in}{2.104879in}}%
\pgfpathlineto{\pgfqpoint{5.091770in}{2.104879in}}%
\pgfpathlineto{\pgfqpoint{5.091770in}{0.778861in}}%
\pgfpathclose%
\pgfusepath{fill}%
\end{pgfscope}%
\begin{pgfscope}%
\pgfpathrectangle{\pgfqpoint{4.560118in}{0.778861in}}{\pgfqpoint{2.025341in}{1.681851in}}%
\pgfusepath{clip}%
\pgfsetbuttcap%
\pgfsetmiterjoin%
\definecolor{currentfill}{rgb}{0.842157,0.553922,0.200980}%
\pgfsetfillcolor{currentfill}%
\pgfsetlinewidth{0.000000pt}%
\definecolor{currentstroke}{rgb}{0.000000,0.000000,0.000000}%
\pgfsetstrokecolor{currentstroke}%
\pgfsetstrokeopacity{0.000000}%
\pgfsetdash{}{0pt}%
\pgfpathmoveto{\pgfqpoint{5.344938in}{0.778861in}}%
\pgfpathlineto{\pgfqpoint{5.412449in}{0.778861in}}%
\pgfpathlineto{\pgfqpoint{5.412449in}{2.223963in}}%
\pgfpathlineto{\pgfqpoint{5.344938in}{2.223963in}}%
\pgfpathlineto{\pgfqpoint{5.344938in}{0.778861in}}%
\pgfpathclose%
\pgfusepath{fill}%
\end{pgfscope}%
\begin{pgfscope}%
\pgfpathrectangle{\pgfqpoint{4.560118in}{0.778861in}}{\pgfqpoint{2.025341in}{1.681851in}}%
\pgfusepath{clip}%
\pgfsetbuttcap%
\pgfsetmiterjoin%
\definecolor{currentfill}{rgb}{0.842157,0.553922,0.200980}%
\pgfsetfillcolor{currentfill}%
\pgfsetlinewidth{0.000000pt}%
\definecolor{currentstroke}{rgb}{0.000000,0.000000,0.000000}%
\pgfsetstrokecolor{currentstroke}%
\pgfsetstrokeopacity{0.000000}%
\pgfsetdash{}{0pt}%
\pgfpathmoveto{\pgfqpoint{5.598105in}{0.778861in}}%
\pgfpathlineto{\pgfqpoint{5.665617in}{0.778861in}}%
\pgfpathlineto{\pgfqpoint{5.665617in}{1.997042in}}%
\pgfpathlineto{\pgfqpoint{5.598105in}{1.997042in}}%
\pgfpathlineto{\pgfqpoint{5.598105in}{0.778861in}}%
\pgfpathclose%
\pgfusepath{fill}%
\end{pgfscope}%
\begin{pgfscope}%
\pgfpathrectangle{\pgfqpoint{4.560118in}{0.778861in}}{\pgfqpoint{2.025341in}{1.681851in}}%
\pgfusepath{clip}%
\pgfsetbuttcap%
\pgfsetmiterjoin%
\definecolor{currentfill}{rgb}{0.842157,0.553922,0.200980}%
\pgfsetfillcolor{currentfill}%
\pgfsetlinewidth{0.000000pt}%
\definecolor{currentstroke}{rgb}{0.000000,0.000000,0.000000}%
\pgfsetstrokecolor{currentstroke}%
\pgfsetstrokeopacity{0.000000}%
\pgfsetdash{}{0pt}%
\pgfpathmoveto{\pgfqpoint{5.851273in}{0.778861in}}%
\pgfpathlineto{\pgfqpoint{5.918784in}{0.778861in}}%
\pgfpathlineto{\pgfqpoint{5.918784in}{1.851185in}}%
\pgfpathlineto{\pgfqpoint{5.851273in}{1.851185in}}%
\pgfpathlineto{\pgfqpoint{5.851273in}{0.778861in}}%
\pgfpathclose%
\pgfusepath{fill}%
\end{pgfscope}%
\begin{pgfscope}%
\pgfpathrectangle{\pgfqpoint{4.560118in}{0.778861in}}{\pgfqpoint{2.025341in}{1.681851in}}%
\pgfusepath{clip}%
\pgfsetbuttcap%
\pgfsetmiterjoin%
\definecolor{currentfill}{rgb}{0.842157,0.553922,0.200980}%
\pgfsetfillcolor{currentfill}%
\pgfsetlinewidth{0.000000pt}%
\definecolor{currentstroke}{rgb}{0.000000,0.000000,0.000000}%
\pgfsetstrokecolor{currentstroke}%
\pgfsetstrokeopacity{0.000000}%
\pgfsetdash{}{0pt}%
\pgfpathmoveto{\pgfqpoint{6.104441in}{0.778861in}}%
\pgfpathlineto{\pgfqpoint{6.171952in}{0.778861in}}%
\pgfpathlineto{\pgfqpoint{6.171952in}{1.738917in}}%
\pgfpathlineto{\pgfqpoint{6.104441in}{1.738917in}}%
\pgfpathlineto{\pgfqpoint{6.104441in}{0.778861in}}%
\pgfpathclose%
\pgfusepath{fill}%
\end{pgfscope}%
\begin{pgfscope}%
\pgfpathrectangle{\pgfqpoint{4.560118in}{0.778861in}}{\pgfqpoint{2.025341in}{1.681851in}}%
\pgfusepath{clip}%
\pgfsetbuttcap%
\pgfsetmiterjoin%
\definecolor{currentfill}{rgb}{0.842157,0.553922,0.200980}%
\pgfsetfillcolor{currentfill}%
\pgfsetlinewidth{0.000000pt}%
\definecolor{currentstroke}{rgb}{0.000000,0.000000,0.000000}%
\pgfsetstrokecolor{currentstroke}%
\pgfsetstrokeopacity{0.000000}%
\pgfsetdash{}{0pt}%
\pgfpathmoveto{\pgfqpoint{6.357608in}{0.778861in}}%
\pgfpathlineto{\pgfqpoint{6.425120in}{0.778861in}}%
\pgfpathlineto{\pgfqpoint{6.425120in}{2.027994in}}%
\pgfpathlineto{\pgfqpoint{6.357608in}{2.027994in}}%
\pgfpathlineto{\pgfqpoint{6.357608in}{0.778861in}}%
\pgfpathclose%
\pgfusepath{fill}%
\end{pgfscope}%
\begin{pgfscope}%
\pgfpathrectangle{\pgfqpoint{4.560118in}{0.778861in}}{\pgfqpoint{2.025341in}{1.681851in}}%
\pgfusepath{clip}%
\pgfsetbuttcap%
\pgfsetmiterjoin%
\definecolor{currentfill}{rgb}{0.858824,0.314706,0.223529}%
\pgfsetfillcolor{currentfill}%
\pgfsetlinewidth{0.000000pt}%
\definecolor{currentstroke}{rgb}{0.000000,0.000000,0.000000}%
\pgfsetstrokecolor{currentstroke}%
\pgfsetstrokeopacity{0.000000}%
\pgfsetdash{}{0pt}%
\pgfpathmoveto{\pgfqpoint{4.652946in}{0.778861in}}%
\pgfpathlineto{\pgfqpoint{4.720458in}{0.778861in}}%
\pgfpathlineto{\pgfqpoint{4.720458in}{1.987824in}}%
\pgfpathlineto{\pgfqpoint{4.652946in}{1.987824in}}%
\pgfpathlineto{\pgfqpoint{4.652946in}{0.778861in}}%
\pgfpathclose%
\pgfusepath{fill}%
\end{pgfscope}%
\begin{pgfscope}%
\pgfpathrectangle{\pgfqpoint{4.560118in}{0.778861in}}{\pgfqpoint{2.025341in}{1.681851in}}%
\pgfusepath{clip}%
\pgfsetbuttcap%
\pgfsetmiterjoin%
\definecolor{currentfill}{rgb}{0.858824,0.314706,0.223529}%
\pgfsetfillcolor{currentfill}%
\pgfsetlinewidth{0.000000pt}%
\definecolor{currentstroke}{rgb}{0.000000,0.000000,0.000000}%
\pgfsetstrokecolor{currentstroke}%
\pgfsetstrokeopacity{0.000000}%
\pgfsetdash{}{0pt}%
\pgfpathmoveto{\pgfqpoint{4.906114in}{0.778861in}}%
\pgfpathlineto{\pgfqpoint{4.973625in}{0.778861in}}%
\pgfpathlineto{\pgfqpoint{4.973625in}{2.302082in}}%
\pgfpathlineto{\pgfqpoint{4.906114in}{2.302082in}}%
\pgfpathlineto{\pgfqpoint{4.906114in}{0.778861in}}%
\pgfpathclose%
\pgfusepath{fill}%
\end{pgfscope}%
\begin{pgfscope}%
\pgfpathrectangle{\pgfqpoint{4.560118in}{0.778861in}}{\pgfqpoint{2.025341in}{1.681851in}}%
\pgfusepath{clip}%
\pgfsetbuttcap%
\pgfsetmiterjoin%
\definecolor{currentfill}{rgb}{0.858824,0.314706,0.223529}%
\pgfsetfillcolor{currentfill}%
\pgfsetlinewidth{0.000000pt}%
\definecolor{currentstroke}{rgb}{0.000000,0.000000,0.000000}%
\pgfsetstrokecolor{currentstroke}%
\pgfsetstrokeopacity{0.000000}%
\pgfsetdash{}{0pt}%
\pgfpathmoveto{\pgfqpoint{5.159281in}{0.778861in}}%
\pgfpathlineto{\pgfqpoint{5.226793in}{0.778861in}}%
\pgfpathlineto{\pgfqpoint{5.226793in}{2.105560in}}%
\pgfpathlineto{\pgfqpoint{5.159281in}{2.105560in}}%
\pgfpathlineto{\pgfqpoint{5.159281in}{0.778861in}}%
\pgfpathclose%
\pgfusepath{fill}%
\end{pgfscope}%
\begin{pgfscope}%
\pgfpathrectangle{\pgfqpoint{4.560118in}{0.778861in}}{\pgfqpoint{2.025341in}{1.681851in}}%
\pgfusepath{clip}%
\pgfsetbuttcap%
\pgfsetmiterjoin%
\definecolor{currentfill}{rgb}{0.858824,0.314706,0.223529}%
\pgfsetfillcolor{currentfill}%
\pgfsetlinewidth{0.000000pt}%
\definecolor{currentstroke}{rgb}{0.000000,0.000000,0.000000}%
\pgfsetstrokecolor{currentstroke}%
\pgfsetstrokeopacity{0.000000}%
\pgfsetdash{}{0pt}%
\pgfpathmoveto{\pgfqpoint{5.412449in}{0.778861in}}%
\pgfpathlineto{\pgfqpoint{5.479960in}{0.778861in}}%
\pgfpathlineto{\pgfqpoint{5.479960in}{2.231434in}}%
\pgfpathlineto{\pgfqpoint{5.412449in}{2.231434in}}%
\pgfpathlineto{\pgfqpoint{5.412449in}{0.778861in}}%
\pgfpathclose%
\pgfusepath{fill}%
\end{pgfscope}%
\begin{pgfscope}%
\pgfpathrectangle{\pgfqpoint{4.560118in}{0.778861in}}{\pgfqpoint{2.025341in}{1.681851in}}%
\pgfusepath{clip}%
\pgfsetbuttcap%
\pgfsetmiterjoin%
\definecolor{currentfill}{rgb}{0.858824,0.314706,0.223529}%
\pgfsetfillcolor{currentfill}%
\pgfsetlinewidth{0.000000pt}%
\definecolor{currentstroke}{rgb}{0.000000,0.000000,0.000000}%
\pgfsetstrokecolor{currentstroke}%
\pgfsetstrokeopacity{0.000000}%
\pgfsetdash{}{0pt}%
\pgfpathmoveto{\pgfqpoint{5.665617in}{0.778861in}}%
\pgfpathlineto{\pgfqpoint{5.733128in}{0.778861in}}%
\pgfpathlineto{\pgfqpoint{5.733128in}{2.015002in}}%
\pgfpathlineto{\pgfqpoint{5.665617in}{2.015002in}}%
\pgfpathlineto{\pgfqpoint{5.665617in}{0.778861in}}%
\pgfpathclose%
\pgfusepath{fill}%
\end{pgfscope}%
\begin{pgfscope}%
\pgfpathrectangle{\pgfqpoint{4.560118in}{0.778861in}}{\pgfqpoint{2.025341in}{1.681851in}}%
\pgfusepath{clip}%
\pgfsetbuttcap%
\pgfsetmiterjoin%
\definecolor{currentfill}{rgb}{0.858824,0.314706,0.223529}%
\pgfsetfillcolor{currentfill}%
\pgfsetlinewidth{0.000000pt}%
\definecolor{currentstroke}{rgb}{0.000000,0.000000,0.000000}%
\pgfsetstrokecolor{currentstroke}%
\pgfsetstrokeopacity{0.000000}%
\pgfsetdash{}{0pt}%
\pgfpathmoveto{\pgfqpoint{5.918784in}{0.778861in}}%
\pgfpathlineto{\pgfqpoint{5.986296in}{0.778861in}}%
\pgfpathlineto{\pgfqpoint{5.986296in}{1.823973in}}%
\pgfpathlineto{\pgfqpoint{5.918784in}{1.823973in}}%
\pgfpathlineto{\pgfqpoint{5.918784in}{0.778861in}}%
\pgfpathclose%
\pgfusepath{fill}%
\end{pgfscope}%
\begin{pgfscope}%
\pgfpathrectangle{\pgfqpoint{4.560118in}{0.778861in}}{\pgfqpoint{2.025341in}{1.681851in}}%
\pgfusepath{clip}%
\pgfsetbuttcap%
\pgfsetmiterjoin%
\definecolor{currentfill}{rgb}{0.858824,0.314706,0.223529}%
\pgfsetfillcolor{currentfill}%
\pgfsetlinewidth{0.000000pt}%
\definecolor{currentstroke}{rgb}{0.000000,0.000000,0.000000}%
\pgfsetstrokecolor{currentstroke}%
\pgfsetstrokeopacity{0.000000}%
\pgfsetdash{}{0pt}%
\pgfpathmoveto{\pgfqpoint{6.171952in}{0.778861in}}%
\pgfpathlineto{\pgfqpoint{6.239463in}{0.778861in}}%
\pgfpathlineto{\pgfqpoint{6.239463in}{1.715898in}}%
\pgfpathlineto{\pgfqpoint{6.171952in}{1.715898in}}%
\pgfpathlineto{\pgfqpoint{6.171952in}{0.778861in}}%
\pgfpathclose%
\pgfusepath{fill}%
\end{pgfscope}%
\begin{pgfscope}%
\pgfpathrectangle{\pgfqpoint{4.560118in}{0.778861in}}{\pgfqpoint{2.025341in}{1.681851in}}%
\pgfusepath{clip}%
\pgfsetbuttcap%
\pgfsetmiterjoin%
\definecolor{currentfill}{rgb}{0.858824,0.314706,0.223529}%
\pgfsetfillcolor{currentfill}%
\pgfsetlinewidth{0.000000pt}%
\definecolor{currentstroke}{rgb}{0.000000,0.000000,0.000000}%
\pgfsetstrokecolor{currentstroke}%
\pgfsetstrokeopacity{0.000000}%
\pgfsetdash{}{0pt}%
\pgfpathmoveto{\pgfqpoint{6.425120in}{0.778861in}}%
\pgfpathlineto{\pgfqpoint{6.492631in}{0.778861in}}%
\pgfpathlineto{\pgfqpoint{6.492631in}{2.025968in}}%
\pgfpathlineto{\pgfqpoint{6.425120in}{2.025968in}}%
\pgfpathlineto{\pgfqpoint{6.425120in}{0.778861in}}%
\pgfpathclose%
\pgfusepath{fill}%
\end{pgfscope}%
\begin{pgfscope}%
\pgfpathrectangle{\pgfqpoint{4.560118in}{0.778861in}}{\pgfqpoint{2.025341in}{1.681851in}}%
\pgfusepath{clip}%
\pgfsetbuttcap%
\pgfsetmiterjoin%
\definecolor{currentfill}{rgb}{0.464706,0.320588,0.573529}%
\pgfsetfillcolor{currentfill}%
\pgfsetlinewidth{0.000000pt}%
\definecolor{currentstroke}{rgb}{0.000000,0.000000,0.000000}%
\pgfsetstrokecolor{currentstroke}%
\pgfsetstrokeopacity{0.000000}%
\pgfsetdash{}{0pt}%
\pgfpathmoveto{\pgfqpoint{4.720458in}{0.778861in}}%
\pgfpathlineto{\pgfqpoint{4.787969in}{0.778861in}}%
\pgfpathlineto{\pgfqpoint{4.787969in}{1.895856in}}%
\pgfpathlineto{\pgfqpoint{4.720458in}{1.895856in}}%
\pgfpathlineto{\pgfqpoint{4.720458in}{0.778861in}}%
\pgfpathclose%
\pgfusepath{fill}%
\end{pgfscope}%
\begin{pgfscope}%
\pgfpathrectangle{\pgfqpoint{4.560118in}{0.778861in}}{\pgfqpoint{2.025341in}{1.681851in}}%
\pgfusepath{clip}%
\pgfsetbuttcap%
\pgfsetmiterjoin%
\definecolor{currentfill}{rgb}{0.464706,0.320588,0.573529}%
\pgfsetfillcolor{currentfill}%
\pgfsetlinewidth{0.000000pt}%
\definecolor{currentstroke}{rgb}{0.000000,0.000000,0.000000}%
\pgfsetstrokecolor{currentstroke}%
\pgfsetstrokeopacity{0.000000}%
\pgfsetdash{}{0pt}%
\pgfpathmoveto{\pgfqpoint{4.973625in}{0.778861in}}%
\pgfpathlineto{\pgfqpoint{5.041137in}{0.778861in}}%
\pgfpathlineto{\pgfqpoint{5.041137in}{2.276812in}}%
\pgfpathlineto{\pgfqpoint{4.973625in}{2.276812in}}%
\pgfpathlineto{\pgfqpoint{4.973625in}{0.778861in}}%
\pgfpathclose%
\pgfusepath{fill}%
\end{pgfscope}%
\begin{pgfscope}%
\pgfpathrectangle{\pgfqpoint{4.560118in}{0.778861in}}{\pgfqpoint{2.025341in}{1.681851in}}%
\pgfusepath{clip}%
\pgfsetbuttcap%
\pgfsetmiterjoin%
\definecolor{currentfill}{rgb}{0.464706,0.320588,0.573529}%
\pgfsetfillcolor{currentfill}%
\pgfsetlinewidth{0.000000pt}%
\definecolor{currentstroke}{rgb}{0.000000,0.000000,0.000000}%
\pgfsetstrokecolor{currentstroke}%
\pgfsetstrokeopacity{0.000000}%
\pgfsetdash{}{0pt}%
\pgfpathmoveto{\pgfqpoint{5.226793in}{0.778861in}}%
\pgfpathlineto{\pgfqpoint{5.294304in}{0.778861in}}%
\pgfpathlineto{\pgfqpoint{5.294304in}{2.090565in}}%
\pgfpathlineto{\pgfqpoint{5.226793in}{2.090565in}}%
\pgfpathlineto{\pgfqpoint{5.226793in}{0.778861in}}%
\pgfpathclose%
\pgfusepath{fill}%
\end{pgfscope}%
\begin{pgfscope}%
\pgfpathrectangle{\pgfqpoint{4.560118in}{0.778861in}}{\pgfqpoint{2.025341in}{1.681851in}}%
\pgfusepath{clip}%
\pgfsetbuttcap%
\pgfsetmiterjoin%
\definecolor{currentfill}{rgb}{0.464706,0.320588,0.573529}%
\pgfsetfillcolor{currentfill}%
\pgfsetlinewidth{0.000000pt}%
\definecolor{currentstroke}{rgb}{0.000000,0.000000,0.000000}%
\pgfsetstrokecolor{currentstroke}%
\pgfsetstrokeopacity{0.000000}%
\pgfsetdash{}{0pt}%
\pgfpathmoveto{\pgfqpoint{5.479960in}{0.778861in}}%
\pgfpathlineto{\pgfqpoint{5.547472in}{0.778861in}}%
\pgfpathlineto{\pgfqpoint{5.547472in}{2.161984in}}%
\pgfpathlineto{\pgfqpoint{5.479960in}{2.161984in}}%
\pgfpathlineto{\pgfqpoint{5.479960in}{0.778861in}}%
\pgfpathclose%
\pgfusepath{fill}%
\end{pgfscope}%
\begin{pgfscope}%
\pgfpathrectangle{\pgfqpoint{4.560118in}{0.778861in}}{\pgfqpoint{2.025341in}{1.681851in}}%
\pgfusepath{clip}%
\pgfsetbuttcap%
\pgfsetmiterjoin%
\definecolor{currentfill}{rgb}{0.464706,0.320588,0.573529}%
\pgfsetfillcolor{currentfill}%
\pgfsetlinewidth{0.000000pt}%
\definecolor{currentstroke}{rgb}{0.000000,0.000000,0.000000}%
\pgfsetstrokecolor{currentstroke}%
\pgfsetstrokeopacity{0.000000}%
\pgfsetdash{}{0pt}%
\pgfpathmoveto{\pgfqpoint{5.733128in}{0.778861in}}%
\pgfpathlineto{\pgfqpoint{5.800640in}{0.778861in}}%
\pgfpathlineto{\pgfqpoint{5.800640in}{1.983574in}}%
\pgfpathlineto{\pgfqpoint{5.733128in}{1.983574in}}%
\pgfpathlineto{\pgfqpoint{5.733128in}{0.778861in}}%
\pgfpathclose%
\pgfusepath{fill}%
\end{pgfscope}%
\begin{pgfscope}%
\pgfpathrectangle{\pgfqpoint{4.560118in}{0.778861in}}{\pgfqpoint{2.025341in}{1.681851in}}%
\pgfusepath{clip}%
\pgfsetbuttcap%
\pgfsetmiterjoin%
\definecolor{currentfill}{rgb}{0.464706,0.320588,0.573529}%
\pgfsetfillcolor{currentfill}%
\pgfsetlinewidth{0.000000pt}%
\definecolor{currentstroke}{rgb}{0.000000,0.000000,0.000000}%
\pgfsetstrokecolor{currentstroke}%
\pgfsetstrokeopacity{0.000000}%
\pgfsetdash{}{0pt}%
\pgfpathmoveto{\pgfqpoint{5.986296in}{0.778861in}}%
\pgfpathlineto{\pgfqpoint{6.053807in}{0.778861in}}%
\pgfpathlineto{\pgfqpoint{6.053807in}{1.824622in}}%
\pgfpathlineto{\pgfqpoint{5.986296in}{1.824622in}}%
\pgfpathlineto{\pgfqpoint{5.986296in}{0.778861in}}%
\pgfpathclose%
\pgfusepath{fill}%
\end{pgfscope}%
\begin{pgfscope}%
\pgfpathrectangle{\pgfqpoint{4.560118in}{0.778861in}}{\pgfqpoint{2.025341in}{1.681851in}}%
\pgfusepath{clip}%
\pgfsetbuttcap%
\pgfsetmiterjoin%
\definecolor{currentfill}{rgb}{0.464706,0.320588,0.573529}%
\pgfsetfillcolor{currentfill}%
\pgfsetlinewidth{0.000000pt}%
\definecolor{currentstroke}{rgb}{0.000000,0.000000,0.000000}%
\pgfsetstrokecolor{currentstroke}%
\pgfsetstrokeopacity{0.000000}%
\pgfsetdash{}{0pt}%
\pgfpathmoveto{\pgfqpoint{6.239463in}{0.778861in}}%
\pgfpathlineto{\pgfqpoint{6.306975in}{0.778861in}}%
\pgfpathlineto{\pgfqpoint{6.306975in}{1.691137in}}%
\pgfpathlineto{\pgfqpoint{6.239463in}{1.691137in}}%
\pgfpathlineto{\pgfqpoint{6.239463in}{0.778861in}}%
\pgfpathclose%
\pgfusepath{fill}%
\end{pgfscope}%
\begin{pgfscope}%
\pgfpathrectangle{\pgfqpoint{4.560118in}{0.778861in}}{\pgfqpoint{2.025341in}{1.681851in}}%
\pgfusepath{clip}%
\pgfsetbuttcap%
\pgfsetmiterjoin%
\definecolor{currentfill}{rgb}{0.464706,0.320588,0.573529}%
\pgfsetfillcolor{currentfill}%
\pgfsetlinewidth{0.000000pt}%
\definecolor{currentstroke}{rgb}{0.000000,0.000000,0.000000}%
\pgfsetstrokecolor{currentstroke}%
\pgfsetstrokeopacity{0.000000}%
\pgfsetdash{}{0pt}%
\pgfpathmoveto{\pgfqpoint{6.492631in}{0.778861in}}%
\pgfpathlineto{\pgfqpoint{6.560142in}{0.778861in}}%
\pgfpathlineto{\pgfqpoint{6.560142in}{1.987753in}}%
\pgfpathlineto{\pgfqpoint{6.492631in}{1.987753in}}%
\pgfpathlineto{\pgfqpoint{6.492631in}{0.778861in}}%
\pgfpathclose%
\pgfusepath{fill}%
\end{pgfscope}%
\begin{pgfscope}%
\pgfpathrectangle{\pgfqpoint{4.560118in}{0.778861in}}{\pgfqpoint{2.025341in}{1.681851in}}%
\pgfusepath{clip}%
\pgfsetbuttcap%
\pgfsetmiterjoin%
\definecolor{currentfill}{rgb}{0.842157,0.553922,0.200980}%
\pgfsetfillcolor{currentfill}%
\pgfsetlinewidth{0.000000pt}%
\definecolor{currentstroke}{rgb}{0.000000,0.000000,0.000000}%
\pgfsetstrokecolor{currentstroke}%
\pgfsetstrokeopacity{0.000000}%
\pgfsetdash{}{0pt}%
\pgfpathmoveto{\pgfqpoint{4.686702in}{0.778861in}}%
\pgfpathlineto{\pgfqpoint{4.686702in}{0.778861in}}%
\pgfpathlineto{\pgfqpoint{4.686702in}{0.778861in}}%
\pgfpathlineto{\pgfqpoint{4.686702in}{0.778861in}}%
\pgfpathlineto{\pgfqpoint{4.686702in}{0.778861in}}%
\pgfpathclose%
\pgfusepath{fill}%
\end{pgfscope}%
\begin{pgfscope}%
\pgfpathrectangle{\pgfqpoint{4.560118in}{0.778861in}}{\pgfqpoint{2.025341in}{1.681851in}}%
\pgfusepath{clip}%
\pgfsetbuttcap%
\pgfsetmiterjoin%
\definecolor{currentfill}{rgb}{0.858824,0.314706,0.223529}%
\pgfsetfillcolor{currentfill}%
\pgfsetlinewidth{0.000000pt}%
\definecolor{currentstroke}{rgb}{0.000000,0.000000,0.000000}%
\pgfsetstrokecolor{currentstroke}%
\pgfsetstrokeopacity{0.000000}%
\pgfsetdash{}{0pt}%
\pgfpathmoveto{\pgfqpoint{4.686702in}{0.778861in}}%
\pgfpathlineto{\pgfqpoint{4.686702in}{0.778861in}}%
\pgfpathlineto{\pgfqpoint{4.686702in}{0.778861in}}%
\pgfpathlineto{\pgfqpoint{4.686702in}{0.778861in}}%
\pgfpathlineto{\pgfqpoint{4.686702in}{0.778861in}}%
\pgfpathclose%
\pgfusepath{fill}%
\end{pgfscope}%
\begin{pgfscope}%
\pgfpathrectangle{\pgfqpoint{4.560118in}{0.778861in}}{\pgfqpoint{2.025341in}{1.681851in}}%
\pgfusepath{clip}%
\pgfsetbuttcap%
\pgfsetmiterjoin%
\definecolor{currentfill}{rgb}{0.464706,0.320588,0.573529}%
\pgfsetfillcolor{currentfill}%
\pgfsetlinewidth{0.000000pt}%
\definecolor{currentstroke}{rgb}{0.000000,0.000000,0.000000}%
\pgfsetstrokecolor{currentstroke}%
\pgfsetstrokeopacity{0.000000}%
\pgfsetdash{}{0pt}%
\pgfpathmoveto{\pgfqpoint{4.686702in}{0.778861in}}%
\pgfpathlineto{\pgfqpoint{4.686702in}{0.778861in}}%
\pgfpathlineto{\pgfqpoint{4.686702in}{0.778861in}}%
\pgfpathlineto{\pgfqpoint{4.686702in}{0.778861in}}%
\pgfpathlineto{\pgfqpoint{4.686702in}{0.778861in}}%
\pgfpathclose%
\pgfusepath{fill}%
\end{pgfscope}%
\begin{pgfscope}%
\pgfsetbuttcap%
\pgfsetroundjoin%
\definecolor{currentfill}{rgb}{0.552941,0.501961,0.478431}%
\pgfsetfillcolor{currentfill}%
\pgfsetlinewidth{0.803000pt}%
\definecolor{currentstroke}{rgb}{0.552941,0.501961,0.478431}%
\pgfsetstrokecolor{currentstroke}%
\pgfsetdash{}{0pt}%
\pgfsys@defobject{currentmarker}{\pgfqpoint{0.000000in}{0.000000in}}{\pgfqpoint{0.000000in}{0.083333in}}{%
\pgfpathmoveto{\pgfqpoint{0.000000in}{0.000000in}}%
\pgfpathlineto{\pgfqpoint{0.000000in}{0.083333in}}%
\pgfusepath{stroke,fill}%
}%
\begin{pgfscope}%
\pgfsys@transformshift{4.686702in}{0.778861in}%
\pgfsys@useobject{currentmarker}{}%
\end{pgfscope}%
\end{pgfscope}%
\begin{pgfscope}%
\definecolor{textcolor}{rgb}{0.552941,0.501961,0.478431}%
\pgfsetstrokecolor{textcolor}%
\pgfsetfillcolor{textcolor}%
\pgftext[x=4.717952in, y=0.444468in, left, base,rotate=90.000000]{\color{textcolor}{\sffamily\fontsize{9.000000}{10.800000}\selectfont\catcode`\^=\active\def^{\ifmmode\sp\else\^{}\fi}\catcode`\%=\active\def%{\%}PA01}}%
\end{pgfscope}%
\begin{pgfscope}%
\pgfsetbuttcap%
\pgfsetroundjoin%
\definecolor{currentfill}{rgb}{0.552941,0.501961,0.478431}%
\pgfsetfillcolor{currentfill}%
\pgfsetlinewidth{0.803000pt}%
\definecolor{currentstroke}{rgb}{0.552941,0.501961,0.478431}%
\pgfsetstrokecolor{currentstroke}%
\pgfsetdash{}{0pt}%
\pgfsys@defobject{currentmarker}{\pgfqpoint{0.000000in}{0.000000in}}{\pgfqpoint{0.000000in}{0.083333in}}{%
\pgfpathmoveto{\pgfqpoint{0.000000in}{0.000000in}}%
\pgfpathlineto{\pgfqpoint{0.000000in}{0.083333in}}%
\pgfusepath{stroke,fill}%
}%
\begin{pgfscope}%
\pgfsys@transformshift{4.939870in}{0.778861in}%
\pgfsys@useobject{currentmarker}{}%
\end{pgfscope}%
\end{pgfscope}%
\begin{pgfscope}%
\definecolor{textcolor}{rgb}{0.552941,0.501961,0.478431}%
\pgfsetstrokecolor{textcolor}%
\pgfsetfillcolor{textcolor}%
\pgftext[x=4.971120in, y=0.433859in, left, base,rotate=90.000000]{\color{textcolor}{\sffamily\fontsize{9.000000}{10.800000}\selectfont\catcode`\^=\active\def^{\ifmmode\sp\else\^{}\fi}\catcode`\%=\active\def%{\%}PB01}}%
\end{pgfscope}%
\begin{pgfscope}%
\pgfsetbuttcap%
\pgfsetroundjoin%
\definecolor{currentfill}{rgb}{0.552941,0.501961,0.478431}%
\pgfsetfillcolor{currentfill}%
\pgfsetlinewidth{0.803000pt}%
\definecolor{currentstroke}{rgb}{0.552941,0.501961,0.478431}%
\pgfsetstrokecolor{currentstroke}%
\pgfsetdash{}{0pt}%
\pgfsys@defobject{currentmarker}{\pgfqpoint{0.000000in}{0.000000in}}{\pgfqpoint{0.000000in}{0.083333in}}{%
\pgfpathmoveto{\pgfqpoint{0.000000in}{0.000000in}}%
\pgfpathlineto{\pgfqpoint{0.000000in}{0.083333in}}%
\pgfusepath{stroke,fill}%
}%
\begin{pgfscope}%
\pgfsys@transformshift{5.193037in}{0.778861in}%
\pgfsys@useobject{currentmarker}{}%
\end{pgfscope}%
\end{pgfscope}%
\begin{pgfscope}%
\definecolor{textcolor}{rgb}{0.552941,0.501961,0.478431}%
\pgfsetstrokecolor{textcolor}%
\pgfsetfillcolor{textcolor}%
\pgftext[x=5.224287in, y=0.433859in, left, base,rotate=90.000000]{\color{textcolor}{\sffamily\fontsize{9.000000}{10.800000}\selectfont\catcode`\^=\active\def^{\ifmmode\sp\else\^{}\fi}\catcode`\%=\active\def%{\%}PB02}}%
\end{pgfscope}%
\begin{pgfscope}%
\pgfsetbuttcap%
\pgfsetroundjoin%
\definecolor{currentfill}{rgb}{0.552941,0.501961,0.478431}%
\pgfsetfillcolor{currentfill}%
\pgfsetlinewidth{0.803000pt}%
\definecolor{currentstroke}{rgb}{0.552941,0.501961,0.478431}%
\pgfsetstrokecolor{currentstroke}%
\pgfsetdash{}{0pt}%
\pgfsys@defobject{currentmarker}{\pgfqpoint{0.000000in}{0.000000in}}{\pgfqpoint{0.000000in}{0.083333in}}{%
\pgfpathmoveto{\pgfqpoint{0.000000in}{0.000000in}}%
\pgfpathlineto{\pgfqpoint{0.000000in}{0.083333in}}%
\pgfusepath{stroke,fill}%
}%
\begin{pgfscope}%
\pgfsys@transformshift{5.446205in}{0.778861in}%
\pgfsys@useobject{currentmarker}{}%
\end{pgfscope}%
\end{pgfscope}%
\begin{pgfscope}%
\definecolor{textcolor}{rgb}{0.552941,0.501961,0.478431}%
\pgfsetstrokecolor{textcolor}%
\pgfsetfillcolor{textcolor}%
\pgftext[x=5.477455in, y=0.433859in, left, base,rotate=90.000000]{\color{textcolor}{\sffamily\fontsize{9.000000}{10.800000}\selectfont\catcode`\^=\active\def^{\ifmmode\sp\else\^{}\fi}\catcode`\%=\active\def%{\%}PB04}}%
\end{pgfscope}%
\begin{pgfscope}%
\pgfsetbuttcap%
\pgfsetroundjoin%
\definecolor{currentfill}{rgb}{0.552941,0.501961,0.478431}%
\pgfsetfillcolor{currentfill}%
\pgfsetlinewidth{0.803000pt}%
\definecolor{currentstroke}{rgb}{0.552941,0.501961,0.478431}%
\pgfsetstrokecolor{currentstroke}%
\pgfsetdash{}{0pt}%
\pgfsys@defobject{currentmarker}{\pgfqpoint{0.000000in}{0.000000in}}{\pgfqpoint{0.000000in}{0.083333in}}{%
\pgfpathmoveto{\pgfqpoint{0.000000in}{0.000000in}}%
\pgfpathlineto{\pgfqpoint{0.000000in}{0.083333in}}%
\pgfusepath{stroke,fill}%
}%
\begin{pgfscope}%
\pgfsys@transformshift{5.699372in}{0.778861in}%
\pgfsys@useobject{currentmarker}{}%
\end{pgfscope}%
\end{pgfscope}%
\begin{pgfscope}%
\definecolor{textcolor}{rgb}{0.552941,0.501961,0.478431}%
\pgfsetstrokecolor{textcolor}%
\pgfsetfillcolor{textcolor}%
\pgftext[x=5.730622in, y=0.437524in, left, base,rotate=90.000000]{\color{textcolor}{\sffamily\fontsize{9.000000}{10.800000}\selectfont\catcode`\^=\active\def^{\ifmmode\sp\else\^{}\fi}\catcode`\%=\active\def%{\%}PC02}}%
\end{pgfscope}%
\begin{pgfscope}%
\pgfsetbuttcap%
\pgfsetroundjoin%
\definecolor{currentfill}{rgb}{0.552941,0.501961,0.478431}%
\pgfsetfillcolor{currentfill}%
\pgfsetlinewidth{0.803000pt}%
\definecolor{currentstroke}{rgb}{0.552941,0.501961,0.478431}%
\pgfsetstrokecolor{currentstroke}%
\pgfsetdash{}{0pt}%
\pgfsys@defobject{currentmarker}{\pgfqpoint{0.000000in}{0.000000in}}{\pgfqpoint{0.000000in}{0.083333in}}{%
\pgfpathmoveto{\pgfqpoint{0.000000in}{0.000000in}}%
\pgfpathlineto{\pgfqpoint{0.000000in}{0.083333in}}%
\pgfusepath{stroke,fill}%
}%
\begin{pgfscope}%
\pgfsys@transformshift{5.952540in}{0.778861in}%
\pgfsys@useobject{currentmarker}{}%
\end{pgfscope}%
\end{pgfscope}%
\begin{pgfscope}%
\definecolor{textcolor}{rgb}{0.552941,0.501961,0.478431}%
\pgfsetstrokecolor{textcolor}%
\pgfsetfillcolor{textcolor}%
\pgftext[x=5.983790in, y=0.437524in, left, base,rotate=90.000000]{\color{textcolor}{\sffamily\fontsize{9.000000}{10.800000}\selectfont\catcode`\^=\active\def^{\ifmmode\sp\else\^{}\fi}\catcode`\%=\active\def%{\%}PC03}}%
\end{pgfscope}%
\begin{pgfscope}%
\pgfsetbuttcap%
\pgfsetroundjoin%
\definecolor{currentfill}{rgb}{0.552941,0.501961,0.478431}%
\pgfsetfillcolor{currentfill}%
\pgfsetlinewidth{0.803000pt}%
\definecolor{currentstroke}{rgb}{0.552941,0.501961,0.478431}%
\pgfsetstrokecolor{currentstroke}%
\pgfsetdash{}{0pt}%
\pgfsys@defobject{currentmarker}{\pgfqpoint{0.000000in}{0.000000in}}{\pgfqpoint{0.000000in}{0.083333in}}{%
\pgfpathmoveto{\pgfqpoint{0.000000in}{0.000000in}}%
\pgfpathlineto{\pgfqpoint{0.000000in}{0.083333in}}%
\pgfusepath{stroke,fill}%
}%
\begin{pgfscope}%
\pgfsys@transformshift{6.205708in}{0.778861in}%
\pgfsys@useobject{currentmarker}{}%
\end{pgfscope}%
\end{pgfscope}%
\begin{pgfscope}%
\definecolor{textcolor}{rgb}{0.552941,0.501961,0.478431}%
\pgfsetstrokecolor{textcolor}%
\pgfsetfillcolor{textcolor}%
\pgftext[x=6.236958in, y=0.437524in, left, base,rotate=90.000000]{\color{textcolor}{\sffamily\fontsize{9.000000}{10.800000}\selectfont\catcode`\^=\active\def^{\ifmmode\sp\else\^{}\fi}\catcode`\%=\active\def%{\%}PC04}}%
\end{pgfscope}%
\begin{pgfscope}%
\pgfsetbuttcap%
\pgfsetroundjoin%
\definecolor{currentfill}{rgb}{0.552941,0.501961,0.478431}%
\pgfsetfillcolor{currentfill}%
\pgfsetlinewidth{0.803000pt}%
\definecolor{currentstroke}{rgb}{0.552941,0.501961,0.478431}%
\pgfsetstrokecolor{currentstroke}%
\pgfsetdash{}{0pt}%
\pgfsys@defobject{currentmarker}{\pgfqpoint{0.000000in}{0.000000in}}{\pgfqpoint{0.000000in}{0.083333in}}{%
\pgfpathmoveto{\pgfqpoint{0.000000in}{0.000000in}}%
\pgfpathlineto{\pgfqpoint{0.000000in}{0.083333in}}%
\pgfusepath{stroke,fill}%
}%
\begin{pgfscope}%
\pgfsys@transformshift{6.458875in}{0.778861in}%
\pgfsys@useobject{currentmarker}{}%
\end{pgfscope}%
\end{pgfscope}%
\begin{pgfscope}%
\definecolor{textcolor}{rgb}{0.552941,0.501961,0.478431}%
\pgfsetstrokecolor{textcolor}%
\pgfsetfillcolor{textcolor}%
\pgftext[x=6.490125in, y=0.509380in, left, base,rotate=90.000000]{\color{textcolor}{\sffamily\fontsize{9.000000}{10.800000}\selectfont\catcode`\^=\active\def^{\ifmmode\sp\else\^{}\fi}\catcode`\%=\active\def%{\%}avg.}}%
\end{pgfscope}%
\begin{pgfscope}%
\definecolor{textcolor}{rgb}{0.552941,0.501961,0.478431}%
\pgfsetstrokecolor{textcolor}%
\pgfsetfillcolor{textcolor}%
\pgftext[x=5.572789in,y=0.378303in,,top]{\color{textcolor}{\sffamily\fontsize{9.000000}{10.800000}\selectfont\catcode`\^=\active\def^{\ifmmode\sp\else\^{}\fi}\catcode`\%=\active\def%{\%}patient}}%
\end{pgfscope}%
\begin{pgfscope}%
\pgfsetbuttcap%
\pgfsetroundjoin%
\definecolor{currentfill}{rgb}{0.552941,0.501961,0.478431}%
\pgfsetfillcolor{currentfill}%
\pgfsetlinewidth{0.803000pt}%
\definecolor{currentstroke}{rgb}{0.552941,0.501961,0.478431}%
\pgfsetstrokecolor{currentstroke}%
\pgfsetdash{}{0pt}%
\pgfsys@defobject{currentmarker}{\pgfqpoint{0.000000in}{0.000000in}}{\pgfqpoint{0.083333in}{0.000000in}}{%
\pgfpathmoveto{\pgfqpoint{0.000000in}{0.000000in}}%
\pgfpathlineto{\pgfqpoint{0.083333in}{0.000000in}}%
\pgfusepath{stroke,fill}%
}%
\begin{pgfscope}%
\pgfsys@transformshift{4.560118in}{0.778861in}%
\pgfsys@useobject{currentmarker}{}%
\end{pgfscope}%
\end{pgfscope}%
\begin{pgfscope}%
\pgfsetbuttcap%
\pgfsetroundjoin%
\definecolor{currentfill}{rgb}{0.552941,0.501961,0.478431}%
\pgfsetfillcolor{currentfill}%
\pgfsetlinewidth{0.803000pt}%
\definecolor{currentstroke}{rgb}{0.552941,0.501961,0.478431}%
\pgfsetstrokecolor{currentstroke}%
\pgfsetdash{}{0pt}%
\pgfsys@defobject{currentmarker}{\pgfqpoint{0.000000in}{0.000000in}}{\pgfqpoint{0.083333in}{0.000000in}}{%
\pgfpathmoveto{\pgfqpoint{0.000000in}{0.000000in}}%
\pgfpathlineto{\pgfqpoint{0.083333in}{0.000000in}}%
\pgfusepath{stroke,fill}%
}%
\begin{pgfscope}%
\pgfsys@transformshift{4.560118in}{1.115231in}%
\pgfsys@useobject{currentmarker}{}%
\end{pgfscope}%
\end{pgfscope}%
\begin{pgfscope}%
\pgfsetbuttcap%
\pgfsetroundjoin%
\definecolor{currentfill}{rgb}{0.552941,0.501961,0.478431}%
\pgfsetfillcolor{currentfill}%
\pgfsetlinewidth{0.803000pt}%
\definecolor{currentstroke}{rgb}{0.552941,0.501961,0.478431}%
\pgfsetstrokecolor{currentstroke}%
\pgfsetdash{}{0pt}%
\pgfsys@defobject{currentmarker}{\pgfqpoint{0.000000in}{0.000000in}}{\pgfqpoint{0.083333in}{0.000000in}}{%
\pgfpathmoveto{\pgfqpoint{0.000000in}{0.000000in}}%
\pgfpathlineto{\pgfqpoint{0.083333in}{0.000000in}}%
\pgfusepath{stroke,fill}%
}%
\begin{pgfscope}%
\pgfsys@transformshift{4.560118in}{1.451601in}%
\pgfsys@useobject{currentmarker}{}%
\end{pgfscope}%
\end{pgfscope}%
\begin{pgfscope}%
\pgfsetbuttcap%
\pgfsetroundjoin%
\definecolor{currentfill}{rgb}{0.552941,0.501961,0.478431}%
\pgfsetfillcolor{currentfill}%
\pgfsetlinewidth{0.803000pt}%
\definecolor{currentstroke}{rgb}{0.552941,0.501961,0.478431}%
\pgfsetstrokecolor{currentstroke}%
\pgfsetdash{}{0pt}%
\pgfsys@defobject{currentmarker}{\pgfqpoint{0.000000in}{0.000000in}}{\pgfqpoint{0.083333in}{0.000000in}}{%
\pgfpathmoveto{\pgfqpoint{0.000000in}{0.000000in}}%
\pgfpathlineto{\pgfqpoint{0.083333in}{0.000000in}}%
\pgfusepath{stroke,fill}%
}%
\begin{pgfscope}%
\pgfsys@transformshift{4.560118in}{1.787971in}%
\pgfsys@useobject{currentmarker}{}%
\end{pgfscope}%
\end{pgfscope}%
\begin{pgfscope}%
\pgfsetbuttcap%
\pgfsetroundjoin%
\definecolor{currentfill}{rgb}{0.552941,0.501961,0.478431}%
\pgfsetfillcolor{currentfill}%
\pgfsetlinewidth{0.803000pt}%
\definecolor{currentstroke}{rgb}{0.552941,0.501961,0.478431}%
\pgfsetstrokecolor{currentstroke}%
\pgfsetdash{}{0pt}%
\pgfsys@defobject{currentmarker}{\pgfqpoint{0.000000in}{0.000000in}}{\pgfqpoint{0.083333in}{0.000000in}}{%
\pgfpathmoveto{\pgfqpoint{0.000000in}{0.000000in}}%
\pgfpathlineto{\pgfqpoint{0.083333in}{0.000000in}}%
\pgfusepath{stroke,fill}%
}%
\begin{pgfscope}%
\pgfsys@transformshift{4.560118in}{2.124341in}%
\pgfsys@useobject{currentmarker}{}%
\end{pgfscope}%
\end{pgfscope}%
\begin{pgfscope}%
\pgfsetbuttcap%
\pgfsetroundjoin%
\definecolor{currentfill}{rgb}{0.552941,0.501961,0.478431}%
\pgfsetfillcolor{currentfill}%
\pgfsetlinewidth{0.803000pt}%
\definecolor{currentstroke}{rgb}{0.552941,0.501961,0.478431}%
\pgfsetstrokecolor{currentstroke}%
\pgfsetdash{}{0pt}%
\pgfsys@defobject{currentmarker}{\pgfqpoint{0.000000in}{0.000000in}}{\pgfqpoint{0.083333in}{0.000000in}}{%
\pgfpathmoveto{\pgfqpoint{0.000000in}{0.000000in}}%
\pgfpathlineto{\pgfqpoint{0.083333in}{0.000000in}}%
\pgfusepath{stroke,fill}%
}%
\begin{pgfscope}%
\pgfsys@transformshift{4.560118in}{2.460711in}%
\pgfsys@useobject{currentmarker}{}%
\end{pgfscope}%
\end{pgfscope}%
\begin{pgfscope}%
\pgfpathrectangle{\pgfqpoint{4.560118in}{0.778861in}}{\pgfqpoint{2.025341in}{1.681851in}}%
\pgfusepath{clip}%
\pgfsetrectcap%
\pgfsetroundjoin%
\pgfsetlinewidth{2.258437pt}%
\definecolor{currentstroke}{rgb}{0.260000,0.260000,0.260000}%
\pgfsetstrokecolor{currentstroke}%
\pgfsetdash{}{0pt}%
\pgfpathmoveto{\pgfqpoint{4.619190in}{1.863818in}}%
\pgfpathlineto{\pgfqpoint{4.619190in}{2.053206in}}%
\pgfusepath{stroke}%
\end{pgfscope}%
\begin{pgfscope}%
\pgfpathrectangle{\pgfqpoint{4.560118in}{0.778861in}}{\pgfqpoint{2.025341in}{1.681851in}}%
\pgfusepath{clip}%
\pgfsetrectcap%
\pgfsetroundjoin%
\pgfsetlinewidth{2.258437pt}%
\definecolor{currentstroke}{rgb}{0.260000,0.260000,0.260000}%
\pgfsetstrokecolor{currentstroke}%
\pgfsetdash{}{0pt}%
\pgfpathmoveto{\pgfqpoint{4.872358in}{2.247366in}}%
\pgfpathlineto{\pgfqpoint{4.872358in}{2.390041in}}%
\pgfusepath{stroke}%
\end{pgfscope}%
\begin{pgfscope}%
\pgfpathrectangle{\pgfqpoint{4.560118in}{0.778861in}}{\pgfqpoint{2.025341in}{1.681851in}}%
\pgfusepath{clip}%
\pgfsetrectcap%
\pgfsetroundjoin%
\pgfsetlinewidth{2.258437pt}%
\definecolor{currentstroke}{rgb}{0.260000,0.260000,0.260000}%
\pgfsetstrokecolor{currentstroke}%
\pgfsetdash{}{0pt}%
\pgfpathmoveto{\pgfqpoint{5.125526in}{1.996380in}}%
\pgfpathlineto{\pgfqpoint{5.125526in}{2.192702in}}%
\pgfusepath{stroke}%
\end{pgfscope}%
\begin{pgfscope}%
\pgfpathrectangle{\pgfqpoint{4.560118in}{0.778861in}}{\pgfqpoint{2.025341in}{1.681851in}}%
\pgfusepath{clip}%
\pgfsetrectcap%
\pgfsetroundjoin%
\pgfsetlinewidth{2.258437pt}%
\definecolor{currentstroke}{rgb}{0.260000,0.260000,0.260000}%
\pgfsetstrokecolor{currentstroke}%
\pgfsetdash{}{0pt}%
\pgfpathmoveto{\pgfqpoint{5.378693in}{2.138724in}}%
\pgfpathlineto{\pgfqpoint{5.378693in}{2.306892in}}%
\pgfusepath{stroke}%
\end{pgfscope}%
\begin{pgfscope}%
\pgfpathrectangle{\pgfqpoint{4.560118in}{0.778861in}}{\pgfqpoint{2.025341in}{1.681851in}}%
\pgfusepath{clip}%
\pgfsetrectcap%
\pgfsetroundjoin%
\pgfsetlinewidth{2.258437pt}%
\definecolor{currentstroke}{rgb}{0.260000,0.260000,0.260000}%
\pgfsetstrokecolor{currentstroke}%
\pgfsetdash{}{0pt}%
\pgfpathmoveto{\pgfqpoint{5.631861in}{1.895126in}}%
\pgfpathlineto{\pgfqpoint{5.631861in}{2.106585in}}%
\pgfusepath{stroke}%
\end{pgfscope}%
\begin{pgfscope}%
\pgfpathrectangle{\pgfqpoint{4.560118in}{0.778861in}}{\pgfqpoint{2.025341in}{1.681851in}}%
\pgfusepath{clip}%
\pgfsetrectcap%
\pgfsetroundjoin%
\pgfsetlinewidth{2.258437pt}%
\definecolor{currentstroke}{rgb}{0.260000,0.260000,0.260000}%
\pgfsetstrokecolor{currentstroke}%
\pgfsetdash{}{0pt}%
\pgfpathmoveto{\pgfqpoint{5.885029in}{1.776384in}}%
\pgfpathlineto{\pgfqpoint{5.885029in}{1.931808in}}%
\pgfusepath{stroke}%
\end{pgfscope}%
\begin{pgfscope}%
\pgfpathrectangle{\pgfqpoint{4.560118in}{0.778861in}}{\pgfqpoint{2.025341in}{1.681851in}}%
\pgfusepath{clip}%
\pgfsetrectcap%
\pgfsetroundjoin%
\pgfsetlinewidth{2.258437pt}%
\definecolor{currentstroke}{rgb}{0.260000,0.260000,0.260000}%
\pgfsetstrokecolor{currentstroke}%
\pgfsetdash{}{0pt}%
\pgfpathmoveto{\pgfqpoint{6.138196in}{1.639115in}}%
\pgfpathlineto{\pgfqpoint{6.138196in}{1.831932in}}%
\pgfusepath{stroke}%
\end{pgfscope}%
\begin{pgfscope}%
\pgfpathrectangle{\pgfqpoint{4.560118in}{0.778861in}}{\pgfqpoint{2.025341in}{1.681851in}}%
\pgfusepath{clip}%
\pgfsetrectcap%
\pgfsetroundjoin%
\pgfsetlinewidth{2.258437pt}%
\definecolor{currentstroke}{rgb}{0.260000,0.260000,0.260000}%
\pgfsetstrokecolor{currentstroke}%
\pgfsetdash{}{0pt}%
\pgfpathmoveto{\pgfqpoint{6.391364in}{1.973218in}}%
\pgfpathlineto{\pgfqpoint{6.391364in}{2.085218in}}%
\pgfusepath{stroke}%
\end{pgfscope}%
\begin{pgfscope}%
\pgfpathrectangle{\pgfqpoint{4.560118in}{0.778861in}}{\pgfqpoint{2.025341in}{1.681851in}}%
\pgfusepath{clip}%
\pgfsetrectcap%
\pgfsetroundjoin%
\pgfsetlinewidth{2.258437pt}%
\definecolor{currentstroke}{rgb}{0.260000,0.260000,0.260000}%
\pgfsetstrokecolor{currentstroke}%
\pgfsetdash{}{0pt}%
\pgfpathmoveto{\pgfqpoint{4.686702in}{1.895541in}}%
\pgfpathlineto{\pgfqpoint{4.686702in}{2.106540in}}%
\pgfusepath{stroke}%
\end{pgfscope}%
\begin{pgfscope}%
\pgfpathrectangle{\pgfqpoint{4.560118in}{0.778861in}}{\pgfqpoint{2.025341in}{1.681851in}}%
\pgfusepath{clip}%
\pgfsetrectcap%
\pgfsetroundjoin%
\pgfsetlinewidth{2.258437pt}%
\definecolor{currentstroke}{rgb}{0.260000,0.260000,0.260000}%
\pgfsetstrokecolor{currentstroke}%
\pgfsetdash{}{0pt}%
\pgfpathmoveto{\pgfqpoint{4.939870in}{2.221833in}}%
\pgfpathlineto{\pgfqpoint{4.939870in}{2.358374in}}%
\pgfusepath{stroke}%
\end{pgfscope}%
\begin{pgfscope}%
\pgfpathrectangle{\pgfqpoint{4.560118in}{0.778861in}}{\pgfqpoint{2.025341in}{1.681851in}}%
\pgfusepath{clip}%
\pgfsetrectcap%
\pgfsetroundjoin%
\pgfsetlinewidth{2.258437pt}%
\definecolor{currentstroke}{rgb}{0.260000,0.260000,0.260000}%
\pgfsetstrokecolor{currentstroke}%
\pgfsetdash{}{0pt}%
\pgfpathmoveto{\pgfqpoint{5.193037in}{1.982557in}}%
\pgfpathlineto{\pgfqpoint{5.193037in}{2.199576in}}%
\pgfusepath{stroke}%
\end{pgfscope}%
\begin{pgfscope}%
\pgfpathrectangle{\pgfqpoint{4.560118in}{0.778861in}}{\pgfqpoint{2.025341in}{1.681851in}}%
\pgfusepath{clip}%
\pgfsetrectcap%
\pgfsetroundjoin%
\pgfsetlinewidth{2.258437pt}%
\definecolor{currentstroke}{rgb}{0.260000,0.260000,0.260000}%
\pgfsetstrokecolor{currentstroke}%
\pgfsetdash{}{0pt}%
\pgfpathmoveto{\pgfqpoint{5.446205in}{2.111929in}}%
\pgfpathlineto{\pgfqpoint{5.446205in}{2.331024in}}%
\pgfusepath{stroke}%
\end{pgfscope}%
\begin{pgfscope}%
\pgfpathrectangle{\pgfqpoint{4.560118in}{0.778861in}}{\pgfqpoint{2.025341in}{1.681851in}}%
\pgfusepath{clip}%
\pgfsetrectcap%
\pgfsetroundjoin%
\pgfsetlinewidth{2.258437pt}%
\definecolor{currentstroke}{rgb}{0.260000,0.260000,0.260000}%
\pgfsetstrokecolor{currentstroke}%
\pgfsetdash{}{0pt}%
\pgfpathmoveto{\pgfqpoint{5.699372in}{1.896135in}}%
\pgfpathlineto{\pgfqpoint{5.699372in}{2.125022in}}%
\pgfusepath{stroke}%
\end{pgfscope}%
\begin{pgfscope}%
\pgfpathrectangle{\pgfqpoint{4.560118in}{0.778861in}}{\pgfqpoint{2.025341in}{1.681851in}}%
\pgfusepath{clip}%
\pgfsetrectcap%
\pgfsetroundjoin%
\pgfsetlinewidth{2.258437pt}%
\definecolor{currentstroke}{rgb}{0.260000,0.260000,0.260000}%
\pgfsetstrokecolor{currentstroke}%
\pgfsetdash{}{0pt}%
\pgfpathmoveto{\pgfqpoint{5.952540in}{1.754385in}}%
\pgfpathlineto{\pgfqpoint{5.952540in}{1.901088in}}%
\pgfusepath{stroke}%
\end{pgfscope}%
\begin{pgfscope}%
\pgfpathrectangle{\pgfqpoint{4.560118in}{0.778861in}}{\pgfqpoint{2.025341in}{1.681851in}}%
\pgfusepath{clip}%
\pgfsetrectcap%
\pgfsetroundjoin%
\pgfsetlinewidth{2.258437pt}%
\definecolor{currentstroke}{rgb}{0.260000,0.260000,0.260000}%
\pgfsetstrokecolor{currentstroke}%
\pgfsetdash{}{0pt}%
\pgfpathmoveto{\pgfqpoint{6.205708in}{1.606345in}}%
\pgfpathlineto{\pgfqpoint{6.205708in}{1.815097in}}%
\pgfusepath{stroke}%
\end{pgfscope}%
\begin{pgfscope}%
\pgfpathrectangle{\pgfqpoint{4.560118in}{0.778861in}}{\pgfqpoint{2.025341in}{1.681851in}}%
\pgfusepath{clip}%
\pgfsetrectcap%
\pgfsetroundjoin%
\pgfsetlinewidth{2.258437pt}%
\definecolor{currentstroke}{rgb}{0.260000,0.260000,0.260000}%
\pgfsetstrokecolor{currentstroke}%
\pgfsetdash{}{0pt}%
\pgfpathmoveto{\pgfqpoint{6.458875in}{1.965020in}}%
\pgfpathlineto{\pgfqpoint{6.458875in}{2.084276in}}%
\pgfusepath{stroke}%
\end{pgfscope}%
\begin{pgfscope}%
\pgfpathrectangle{\pgfqpoint{4.560118in}{0.778861in}}{\pgfqpoint{2.025341in}{1.681851in}}%
\pgfusepath{clip}%
\pgfsetrectcap%
\pgfsetroundjoin%
\pgfsetlinewidth{2.258437pt}%
\definecolor{currentstroke}{rgb}{0.260000,0.260000,0.260000}%
\pgfsetstrokecolor{currentstroke}%
\pgfsetdash{}{0pt}%
\pgfpathmoveto{\pgfqpoint{4.754213in}{1.811914in}}%
\pgfpathlineto{\pgfqpoint{4.754213in}{1.975798in}}%
\pgfusepath{stroke}%
\end{pgfscope}%
\begin{pgfscope}%
\pgfpathrectangle{\pgfqpoint{4.560118in}{0.778861in}}{\pgfqpoint{2.025341in}{1.681851in}}%
\pgfusepath{clip}%
\pgfsetrectcap%
\pgfsetroundjoin%
\pgfsetlinewidth{2.258437pt}%
\definecolor{currentstroke}{rgb}{0.260000,0.260000,0.260000}%
\pgfsetstrokecolor{currentstroke}%
\pgfsetdash{}{0pt}%
\pgfpathmoveto{\pgfqpoint{5.007381in}{2.173206in}}%
\pgfpathlineto{\pgfqpoint{5.007381in}{2.355216in}}%
\pgfusepath{stroke}%
\end{pgfscope}%
\begin{pgfscope}%
\pgfpathrectangle{\pgfqpoint{4.560118in}{0.778861in}}{\pgfqpoint{2.025341in}{1.681851in}}%
\pgfusepath{clip}%
\pgfsetrectcap%
\pgfsetroundjoin%
\pgfsetlinewidth{2.258437pt}%
\definecolor{currentstroke}{rgb}{0.260000,0.260000,0.260000}%
\pgfsetstrokecolor{currentstroke}%
\pgfsetdash{}{0pt}%
\pgfpathmoveto{\pgfqpoint{5.260549in}{1.952648in}}%
\pgfpathlineto{\pgfqpoint{5.260549in}{2.194315in}}%
\pgfusepath{stroke}%
\end{pgfscope}%
\begin{pgfscope}%
\pgfpathrectangle{\pgfqpoint{4.560118in}{0.778861in}}{\pgfqpoint{2.025341in}{1.681851in}}%
\pgfusepath{clip}%
\pgfsetrectcap%
\pgfsetroundjoin%
\pgfsetlinewidth{2.258437pt}%
\definecolor{currentstroke}{rgb}{0.260000,0.260000,0.260000}%
\pgfsetstrokecolor{currentstroke}%
\pgfsetdash{}{0pt}%
\pgfpathmoveto{\pgfqpoint{5.513716in}{2.004433in}}%
\pgfpathlineto{\pgfqpoint{5.513716in}{2.277925in}}%
\pgfusepath{stroke}%
\end{pgfscope}%
\begin{pgfscope}%
\pgfpathrectangle{\pgfqpoint{4.560118in}{0.778861in}}{\pgfqpoint{2.025341in}{1.681851in}}%
\pgfusepath{clip}%
\pgfsetrectcap%
\pgfsetroundjoin%
\pgfsetlinewidth{2.258437pt}%
\definecolor{currentstroke}{rgb}{0.260000,0.260000,0.260000}%
\pgfsetstrokecolor{currentstroke}%
\pgfsetdash{}{0pt}%
\pgfpathmoveto{\pgfqpoint{5.766884in}{1.889195in}}%
\pgfpathlineto{\pgfqpoint{5.766884in}{2.090463in}}%
\pgfusepath{stroke}%
\end{pgfscope}%
\begin{pgfscope}%
\pgfpathrectangle{\pgfqpoint{4.560118in}{0.778861in}}{\pgfqpoint{2.025341in}{1.681851in}}%
\pgfusepath{clip}%
\pgfsetrectcap%
\pgfsetroundjoin%
\pgfsetlinewidth{2.258437pt}%
\definecolor{currentstroke}{rgb}{0.260000,0.260000,0.260000}%
\pgfsetstrokecolor{currentstroke}%
\pgfsetdash{}{0pt}%
\pgfpathmoveto{\pgfqpoint{6.020051in}{1.731276in}}%
\pgfpathlineto{\pgfqpoint{6.020051in}{1.922362in}}%
\pgfusepath{stroke}%
\end{pgfscope}%
\begin{pgfscope}%
\pgfpathrectangle{\pgfqpoint{4.560118in}{0.778861in}}{\pgfqpoint{2.025341in}{1.681851in}}%
\pgfusepath{clip}%
\pgfsetrectcap%
\pgfsetroundjoin%
\pgfsetlinewidth{2.258437pt}%
\definecolor{currentstroke}{rgb}{0.260000,0.260000,0.260000}%
\pgfsetstrokecolor{currentstroke}%
\pgfsetdash{}{0pt}%
\pgfpathmoveto{\pgfqpoint{6.273219in}{1.594797in}}%
\pgfpathlineto{\pgfqpoint{6.273219in}{1.804959in}}%
\pgfusepath{stroke}%
\end{pgfscope}%
\begin{pgfscope}%
\pgfpathrectangle{\pgfqpoint{4.560118in}{0.778861in}}{\pgfqpoint{2.025341in}{1.681851in}}%
\pgfusepath{clip}%
\pgfsetrectcap%
\pgfsetroundjoin%
\pgfsetlinewidth{2.258437pt}%
\definecolor{currentstroke}{rgb}{0.260000,0.260000,0.260000}%
\pgfsetstrokecolor{currentstroke}%
\pgfsetdash{}{0pt}%
\pgfpathmoveto{\pgfqpoint{6.526387in}{1.925474in}}%
\pgfpathlineto{\pgfqpoint{6.526387in}{2.045904in}}%
\pgfusepath{stroke}%
\end{pgfscope}%
\begin{pgfscope}%
\pgfpathrectangle{\pgfqpoint{4.560118in}{0.778861in}}{\pgfqpoint{2.025341in}{1.681851in}}%
\pgfusepath{clip}%
\pgfsetbuttcap%
\pgfsetroundjoin%
\pgfsetlinewidth{1.003750pt}%
\definecolor{currentstroke}{rgb}{0.552941,0.501961,0.478431}%
\pgfsetstrokecolor{currentstroke}%
\pgfsetdash{{3.700000pt}{1.600000pt}}{0.000000pt}%
\pgfpathmoveto{\pgfqpoint{4.560118in}{1.619786in}}%
\pgfpathlineto{\pgfqpoint{6.585459in}{1.619786in}}%
\pgfusepath{stroke}%
\end{pgfscope}%
\begin{pgfscope}%
\pgfsetrectcap%
\pgfsetmiterjoin%
\pgfsetlinewidth{0.803000pt}%
\definecolor{currentstroke}{rgb}{0.552941,0.501961,0.478431}%
\pgfsetstrokecolor{currentstroke}%
\pgfsetdash{}{0pt}%
\pgfpathmoveto{\pgfqpoint{4.560118in}{0.778861in}}%
\pgfpathlineto{\pgfqpoint{4.560118in}{2.460711in}}%
\pgfusepath{stroke}%
\end{pgfscope}%
\begin{pgfscope}%
\pgfsetrectcap%
\pgfsetmiterjoin%
\pgfsetlinewidth{0.803000pt}%
\definecolor{currentstroke}{rgb}{0.552941,0.501961,0.478431}%
\pgfsetstrokecolor{currentstroke}%
\pgfsetdash{}{0pt}%
\pgfpathmoveto{\pgfqpoint{4.560118in}{0.778861in}}%
\pgfpathlineto{\pgfqpoint{6.585459in}{0.778861in}}%
\pgfusepath{stroke}%
\end{pgfscope}%
\begin{pgfscope}%
\definecolor{textcolor}{rgb}{0.552941,0.501961,0.478431}%
\pgfsetstrokecolor{textcolor}%
\pgfsetfillcolor{textcolor}%
\pgftext[x=4.560118in,y=2.544045in,left,base]{\color{textcolor}{\sffamily\fontsize{9.000000}{10.800000}\selectfont\catcode`\^=\active\def^{\ifmmode\sp\else\^{}\fi}\catcode`\%=\active\def%{\%}free VSA}}%
\end{pgfscope}%
\begin{pgfscope}%
\pgfsetbuttcap%
\pgfsetmiterjoin%
\definecolor{currentfill}{rgb}{0.842157,0.553922,0.200980}%
\pgfsetfillcolor{currentfill}%
\pgfsetlinewidth{0.000000pt}%
\definecolor{currentstroke}{rgb}{0.000000,0.000000,0.000000}%
\pgfsetstrokecolor{currentstroke}%
\pgfsetstrokeopacity{0.000000}%
\pgfsetdash{}{0pt}%
\pgfpathmoveto{\pgfqpoint{2.383385in}{0.074306in}}%
\pgfpathlineto{\pgfqpoint{2.633385in}{0.074306in}}%
\pgfpathlineto{\pgfqpoint{2.633385in}{0.161806in}}%
\pgfpathlineto{\pgfqpoint{2.383385in}{0.161806in}}%
\pgfpathlineto{\pgfqpoint{2.383385in}{0.074306in}}%
\pgfpathclose%
\pgfusepath{fill}%
\end{pgfscope}%
\begin{pgfscope}%
\definecolor{textcolor}{rgb}{0.552941,0.501961,0.478431}%
\pgfsetstrokecolor{textcolor}%
\pgfsetfillcolor{textcolor}%
\pgftext[x=2.733385in,y=0.074306in,left,base]{\color{textcolor}{\sffamily\fontsize{9.000000}{10.800000}\selectfont\catcode`\^=\active\def^{\ifmmode\sp\else\^{}\fi}\catcode`\%=\active\def%{\%}tLDA}}%
\end{pgfscope}%
\begin{pgfscope}%
\pgfsetbuttcap%
\pgfsetmiterjoin%
\definecolor{currentfill}{rgb}{0.858824,0.314706,0.223529}%
\pgfsetfillcolor{currentfill}%
\pgfsetlinewidth{0.000000pt}%
\definecolor{currentstroke}{rgb}{0.000000,0.000000,0.000000}%
\pgfsetstrokecolor{currentstroke}%
\pgfsetstrokeopacity{0.000000}%
\pgfsetdash{}{0pt}%
\pgfpathmoveto{\pgfqpoint{3.274953in}{0.074306in}}%
\pgfpathlineto{\pgfqpoint{3.524953in}{0.074306in}}%
\pgfpathlineto{\pgfqpoint{3.524953in}{0.161806in}}%
\pgfpathlineto{\pgfqpoint{3.274953in}{0.161806in}}%
\pgfpathlineto{\pgfqpoint{3.274953in}{0.074306in}}%
\pgfpathclose%
\pgfusepath{fill}%
\end{pgfscope}%
\begin{pgfscope}%
\definecolor{textcolor}{rgb}{0.552941,0.501961,0.478431}%
\pgfsetstrokecolor{textcolor}%
\pgfsetfillcolor{textcolor}%
\pgftext[x=3.624953in,y=0.074306in,left,base]{\color{textcolor}{\sffamily\fontsize{9.000000}{10.800000}\selectfont\catcode`\^=\active\def^{\ifmmode\sp\else\^{}\fi}\catcode`\%=\active\def%{\%}WCBLE}}%
\end{pgfscope}%
\begin{pgfscope}%
\pgfsetbuttcap%
\pgfsetmiterjoin%
\definecolor{currentfill}{rgb}{0.464706,0.320588,0.573529}%
\pgfsetfillcolor{currentfill}%
\pgfsetlinewidth{0.000000pt}%
\definecolor{currentstroke}{rgb}{0.000000,0.000000,0.000000}%
\pgfsetstrokecolor{currentstroke}%
\pgfsetstrokeopacity{0.000000}%
\pgfsetdash{}{0pt}%
\pgfpathmoveto{\pgfqpoint{4.307916in}{0.074306in}}%
\pgfpathlineto{\pgfqpoint{4.557916in}{0.074306in}}%
\pgfpathlineto{\pgfqpoint{4.557916in}{0.161806in}}%
\pgfpathlineto{\pgfqpoint{4.307916in}{0.161806in}}%
\pgfpathlineto{\pgfqpoint{4.307916in}{0.074306in}}%
\pgfpathclose%
\pgfusepath{fill}%
\end{pgfscope}%
\begin{pgfscope}%
\definecolor{textcolor}{rgb}{0.552941,0.501961,0.478431}%
\pgfsetstrokecolor{textcolor}%
\pgfsetfillcolor{textcolor}%
\pgftext[x=4.657916in,y=0.074306in,left,base]{\color{textcolor}{\sffamily\fontsize{9.000000}{10.800000}\selectfont\catcode`\^=\active\def^{\ifmmode\sp\else\^{}\fi}\catcode`\%=\active\def%{\%}XDAWNCov+TS+LDA}}%
\end{pgfscope}%
\end{pgfpicture}%
\makeatother%
\endgroup%

\end{figure}
\todo{check if percentages or 0-1 accuracy and roc-auc}

For PB2 and PC4 performing covertly, was free decoding higher than covert
decoding?

\subsection{Cross-condition calibration}

| indication of fatigue, cinsistency within free case
| closer to overt or to covert by looking at eye tracker
|
| will learn
| split up per target
|
| Findings:
| 1. Some patients prefer to do covert (check performance results for them),
|    analyze distance to target
| 2. covert attention can be improved using WCBLE for all subjects, but not to
| the
|    level of overt covert attention works best when training in overt
| attention
| s this also true for patients preferring covert?
| 3. free performs on par (in all subjects?), opens debate of necessity of
|    gaze-independence BCIs
| 4. covert attention works best when training in overt attention. Is this also
|    true for patients preferring covert?

2024-09-18 04:39:35 PM can more overt attention epochs closer to
| mention problem can't know if patient shifted or eye tracker shifted
| (difficult mobile gaze tracker)
| overt lower because can't perform

\section{Discussion}

patients 2 are progressing towards 3

Limitations

Limited sample of patients, FA maybe not most interesting group to represent
real BCI users

interpreting eye tracker results hard because of inaccuracy eye tracker for eye
motor impaired patients. Redo with more advanced stationary eye tracker and
head rest.

advanced eye tracking equipment is necessary

user comfort was not objectively measured using established questionnaires

Further work
Comparing with a sample of non-gaze impaired patients with  neuromuscular
disability DMD
On-line session with these patients

In a follow up study, we aim to revisit these patients with an on-line system
to be able to calculate a realistic \ac{itr} and measure their user experience.

\section{Conclusion}

\todo{https://sci-hub.ru/https://ieeexplore.ieee.org/abstract/document/1642754}
\todo{https://www.sciencedirect.com/science/article/abs/pii/B9780444639349000020}
\todo{BCI principles and practice chapter 19}
\todo{https://www.nature.com/articles/s44222-024-00239-5}
