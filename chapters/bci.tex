
\todo{Shortly describe challenges from the patient point of view, related to
this thesis}

tetraplegic, locked-in syndrome
Regaining capabilities and improving QoL
Start from wolpaw examples, for spelling
Cite State of the art examples (walk again, speech avatar Metzger, handwriting
willet, neuralink)

\section{A direct interface to the brain}

Brain-computer interfaces (BCIs) read your mind.
BCIs are an entertaining subject with many futuristic and more established
applications.
Devices processing direct inputs from the central nervous system can be
useful for rehabilitation, medical diagnosis or treatment.
They have always been especially promising as assistive technology for
paralyzed patients limited in their communication ability. \todo{expand more on this last
sentence, put patients first in introduction}
Furthermore, they bring a new paradigm to human-computer interaction,
especially when paired with virtual or augmented reality.

Simply put, a BCI records the user's brain activity, then extracts some
relevant output from this brain activity and couples this output to a function
of a device.
Optionally, the user can then observe the action of the device and adapt their
behaviour or brain activity accordingly, closing the loop
\footnote{
  \textcite{BCISociety2024} has recently formalized this into the
  following definition:
  \it``A brain-computer interface is a system that measures brain activity and
  converts it in (nearly) real-time into functionally useful outputs to replace,
  restore, enhance, supplement, and/or improve the natural outputs of the brain,
  thereby changing the ongoing interactions between the brain and its external or
  internal environments. It may additionally modify brain activity using targeted
  delivery of stimuli to create functionally useful inputs to the
  brain.''
}.

\fullpagefig{figures/bci/bci_loop.pdf}{The Brain-Computer Interfacing loop.}{fig:bci-loop}

Let's break this definition down and focus on its separate parts.
First of all, we need to identify a signal that is a direct representation of
what's going on in the brain.
One might for instance measure the fluctuations in blood flow to specific brain
regions that are more or less active at a given time, i.e. the
\emph{hemodynamic signal} with a scanner.
However, this signal reacts to slowly to reflect the real-time activity that is
of interest for our purposes, and it carries little information other than the
brain regions where it is originating.
A better candidate is the neuronal \emph{action potential}.
The brain consists 86 billion neurons, which are highly interconnected cells that are
the smallest units of information processing.
The action potential is an electrical pulse occurring when a neuron receives
input from another neuron and is activated.\todo{kort door de bocht}
A neuron's action potential, or the combined action potentials of groups of
synchronized neurons, thus generate an electrical field in and near the brain,
which can be measured using electrodes.
Fluctuations in the brain's electrical potential can be faster and more
informative.\todo{ref section}

The recorded brain waves form only one part of the interaction scheme presented
in Figure~\ref{fig:bci-loop}.
It would make little sense to start digging around at random in all the brain
activity to try to extract the desired output.
It would help greatly if we knew what we are looking for.
For this reason, a BCI operates using a defined \emph{paradigm}, the manner in
which interaction is performed.
This usually means that the user is instructed to perform a specific task, like
attending a specific flickering stimulus.
We can now exploit background knowledge from prior neuroscientific research
that the brain response evoked by attended stimuli is different of those
unattended,
and determine which one of a set of targets was attended.
In turn, this can be coupled to an action, like typing the letter shown on the
stimulus.
The BCI paradigm comprises, if applicable, the manner of stimulation and the
exact task performed by the user, and is often closely linked to the manner of
feedback in the case of a closed-loop system.
As shown in Section~\ref{sec:bci-paradigms}, there is a wide variety of BCI
paradigms based on different systems in the brain.

The previous points to another component of a successful BCI: the specific
brain signals related to our paradigm need to be identified within the recorded
activity.
This is commonly referred to as \emph{feature extraction}
In other words, the signal needs to be interpreted \emph{in function of} the
task we want our BCI to perform.
The electrical activity of the brain is relatively weak compared to that of the
environment around us, containing electronic devices and other sources of
interference.
Furthermore, the brain is continuously processing information and carrying out
`background tasks', all of which generate their own electrical signals.
On the other hand, the brain activity we are interested in exploiting often
originates from a specific functional system within the brain, and its activity
might not be easily discerned from all other neural and environmental activity,
akin to trying to listen to a single speaker in a noise environment where
everyone is shouting over each other.
Some obvious sources of interference can be easily filtered out, but to obtain
a good measure reflecting only the activity that is relevant for a given BCI
problem.
This problem is called \emph{decoding} and is often solved through supervised
machine learning.
Section~\label{sec:bci-decoding} takes a closer look at some common and
state-of-the art techniques.
A machine learning algorithm can use some training data for which the performed
tasks are known, and learn to recognize these tasks in new data.
The training data can be either obtained from the BCI user themselves or from
other users.
In the first case, the user would perform a short calibration session before
using the BCI, where they are instructed to perform a known task.
This calibration session can be eliminated by training the decoder on
preexisting labeled data from other sessions and users, but this is harder due
to the variability between subjects and sessions.\todo{make shorter, move part
to section}

Finally, the loop can be closed by coupling this output can be coupled to a
device or actuator.
There are two aspects to this.
On the one hand, the BCI gains its function by allowing the user to interact
with their environment.
On the other hand, the actions performed by the BCI can themselves influence
the user's brain activity, creating an adaptive system.
A direct form of closed-loop BCI is the use of a neurostimulator.
Examples of this are deep brain stimulation to mitigate the symptoms of
Parkinson's disease, or the cochlear implant as an auditive prosthesis.
More indirectly modulation can be achieved by sensory stimulation.
This can involve specific, paradigm-related sensory input, like haptic feedback when a
movement BCI detects an intended action, or simply presenting selected actions
back to the user.
The user's brain will then adapt through reinforcement learning, causing
changes in behavior or strategy.
Due to neuroplasticity, the brain's ability to adapt, this can have a positive
impact on BCI performance and opens the pathway to rehabilitation applications.

\todo{wrap-up}


\todo{check excercise session presentation}

\section{Recording technologies}
\label{sec:bci-recording}
The brain's activity can be recorded using various neuroimaging
technologies.
These can range from brain scans~\cite{Weiskopf2004} using functional magnetic resonance imaging
(fMRI) to more portable technologies like accoustic signals obtained by functional ultrasound imaging
(fUS)~\cite{Zheng2023} and near-infrared spectroscopy
(fNIRS)~\cite{Borgheai2020}.
These technologies all measure the hemodynamic signal, which only indirectly
reflects brain activity.
Because of the desirable high temporal resolution of
electrophysiology~\cite{Easttom2021} and
because of their practicality, we will focus only on methods to record
electromagnetical activity originating from neuronal action potentials.
The magnetical field of the brain can be measured using magnetoencephalography
(MEG)~\cite{Mellinger2007}.
While MEG is non-invasive and results in high-quality signals, it is relatively
impractical due to the size of the equipment needed.
Recent advances have been made using optically pumped magnetometers
(OPM-MEG)~\cite{Wittevrongel2021}, but this technology still falters outside of
lab conditions.

As a consequence, the BCI field relies heavily on technologies measuring the
electrical activity of the brain.
As mentioned earlier with MEG, an important concept in determining which technology within this group is
suited for a BCI application is the invasiveness of the technology.
single-unit recordings
microelectrode arrays (utah-array, neuralink)
ecog/µecog
eeg

EEG cheap and practical, non-invasive, high-temporal resolution, but noise, suffers from volume
conduction hence low spatial resolution
what does EEG measure?





different recording technologies and level of invasiveness + infographic
cost, ease of use
\section{BCI Paradigms}
\label{sec:bci-paradigms}
\fullpagefig{figures/bci/paradigms.eps}{caption}{fig:bci-paradigms}

\section{Decoders}
\label{sec:bci-decoders}
evaluated on
information transfer rate,
roc auc,
accuracy

Calibration, transfer learning, riemannian geometry, (multi)-linear techniques,
paper fabien lotte

active/reactive/passive + infographic


\section{A case study: the visual oddball ERP paradigm}
\subsection{stimulation}
\subsection{The event-related potential}
\todo{figure of P3 matrix}
P300 and other components
\todo{figure of ERP components}
\subsection{Decoding ERPs}
tLDA, beamforming
Linear decoding techniques
