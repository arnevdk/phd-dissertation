\chapter{Gaze-independent BCIs}
\todo{mention partially published}
\chapter{Gaze-independent BCIs}
\section{Oculomotor deficits in Locked-in Syndrome}
\section{Gaze-independent visual BCIs}
\section{Benefits \& drawbacks of covert attention in BCI operation}
\section{Electrophysiological correlates of covert attention}
\section{Compensating for covert attention}
\todo{spatially vs temporally organized interfaces}

Gaze-independent ERP-based BCIs~\cite{Riccio2012} can be realized in three
ways.
Firstly, non-visual stimulation paradigms such as auditory and somatosensory paradigms
do not rely on gaze redirection but often result in lower information transfer
rates, increased mental effort and user-dependent variability~\cite{Reichert2020b}.
Secondly, visual stimulation can be optimized, e.g. such that the stimuli are always
present in the field of view~\cite{Treder2010,Pires2011,Lees2018}.
Non-spatial visual attention (feature attention) can also be exploited, such as
attention to stimulus color, shape or symbol~\cite{Zhang2010,Treder2011,Hwang2015}.
Alternative visual stimulation paradigms can modulate specific ERP components
that are more sensitive to stimulation in the visual periphery~\cite{Schaeff2012,Xu2022}.
%These methods, However, can suffer from reduced information
%transfer rates~\cite{Chennu2013}.
However, they still rely to some extent on
eye motor control, often necessitating central gaze fixation.

Firstly, by avoiding visual stimulation entirely and opting for auditory or
somatosensory stimulation instead.
However, these alternatives often result in lower information transfer
rates, increased mental effort and
user-dependent variability~\cite{Reichert2020b}.
Secondly, gaze-independence can be achieved through tailored stimulus presentation.
This can involve enforcing overt VSA by presenting stimuli that fill the field
of view of the user,
exploiting visual non-spatial attention, or a combination thereof~\cite{Treder2011}.
The most common example of non-spatial attention is feature attention, where
the user directs their attention to an expected property of some of the
presented stimuli, such as its color, its shape or the symbol it represents~\cite{Zhang2010,Hwang2015,Reichert2020a}.
In covert VSA scenarios, stimulus paradigms are often fine-tuned to modulate
specific ERP responses~\cite{Schaeff2012,Xu2022}, increasing the
signal-to-noise ratio (SNR).
Some noteworthy examples include the GIBS Block Speller~\cite{Pires2011},
the GeoSpell interface~\cite{Aloise2012a}, and the RSVP
speller~\cite{Acqualagna2011,Lees2018}.
These systems typically require users to focus on a central fixation point while
selecting peripheral targets.
However, they often suffer from reduced information
transfer rates~\cite{Chennu2013}.
Moreover, they still rely to some extent on
eye motor control, often necessitating central gaze fixation.

Thirdly, stimuli can be presented in a standard BCI paradigm, but visuospatial attention
can be decoded separately from gaze direction.
Earlier work has made advances in decoding lateralized covert
VSA harnessing the N2pc ERP component~\cite{Thiery2016,Reichert2020b,Wang2022},
decoding covert VSA shifts from spectral content~\cite{Tonin2013}, or hybrid
stimulation paradigms~\cite{Egan2017}.
Since these methods often require a slower stimulation pace than the classical
ERP BCI paradigm, gaze-independent decoding in fast-paced ERP paradigm stimulation
remains underexplored.
\cite{Aloise2012b} aimed to bridge the performance gap between covert and
overt VSA decoding performance.
They compared classical linear and non-linear ERP classifiers on a covert
attention P3 ERP component dataset.
The results revealed no significant performance improvement in covert VSA
decoding for any of the investigated decoders.
\cite{Arico2014} observed higher variability in single-trial P3 peak
latencies relative to stimulus onset during covert VSA compared to overt VSA.
This latency variability contributes to reduced covert VSA decoding performance.
While they proposed an analysis and performance prediction method, they did not provide a
decoding solution.
They suggested that compensating for latency jitter could enhance covert VSA
decoding, but did not actually verify this hypothesis directly.
Additionally, \cite{hardiansyah2020single} developed a classifier for
covert VSA ERPs, exploiting single-trial latency features in combination with
amplitude features for classification with a support vector machine.
They demonstrated the positive influence of single-trial ERP component latency
features on covert VSA inference yet did not attempt to correct the amplitude
features for these latencies.
\cite{Frenzel2011} introduced a similar protocol to our split VSA
setting.
They showed that it is possible to perform split VSA and that, in this
case, visuospatial attention and gaze direction can separately be decoded using
classical ERP techniques.
To the best of our
knowledge, this is the sole study that investigated split attention in ERP-based BCIs.
However, \cite{Frenzel2011} considered their interface only for the case
where the user actively intends to select both targets determined by the gaze
and the VSA.
In the split VSA setting considered in our work, we rather instruct the participant to ignore
the distractor and only attend the cued target, since we are interested in
decoding the visuospatial attention only.

P3 latency generally falls between 350ms and 600ms~\cite{Luck2014}, but this
value is heavily dependent on the subject and the task, and can vary from trial
to trial~\cite{Ouyang2017}.
The work of \cite{Arico2014}
illustrates that the variation in single-trial P3 latencies is important in
gaze-independent decoding and has been hampering covert VSA decoding performance.
In this work we aim to reprise their hypothesis stating that jitter compensation improves
covert VSA performance and extend it by developing a decoder and evaluating it
in a broader range of gaze-independent settings, including split VSA.
\todo{Check connection}
