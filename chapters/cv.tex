% Structure

\newenvironment{cvlist}{%
	\begin{description}[font=\normalfont]%\setlength\itemsep{0em}
}{%
  \end{description}
}

\newcommand{\cvitem}[2]{\item[#1,] #2}


%Preamble


%\begin{minipage}[b]{.7\linewidth}
%	\Large \textbf{Arne Van Den Kerchove}\\
%	\normalsize Doctoral Researcher Computational Neuroscience\\
%	KU Leuven, University of Lille
%  \bigskip
%
%  \texttt{arne.vandenkerchove@kuleuven.be \\
%  arne.vandenkerchove@univ-lille.fr \\
%  arne@vandenkerchove.com \\
%  +32 473 32 78 71 \\
%	\url{https://arne.vandenkerchove.com}\\
%	\url{https://orcid.org/0000-0002-9367-2986} \\
%  \url{https://linkedin.com/in/arnevdk} \\}
%
%%	\begin{tabular}{@{}c l}
%%		\faAt        & arne.vandenkerchove@kuleuven.be             \\
%%		\faAt        & arne.vandenkerchove@univ-lille.fr           \\
%%		\faAt        & arne@vandenkerchove.com                     \\
%%		\faPhone     & +32 473 32 78 71                            \\
%%		\faMapMarker & Leuven, Belgium                             \\
%%		\faMapMarker & Laboratory for Neuro- and Psychophysiology  \\
%%		             & KU Leuven-Campus Gasthuisberg             \\
%%		             & ON II-Herestraat 49-box 1021            \\
%%		             & BE-3000 Leuven                              \\
%%		\faMapMarker & UMR CRIStAL                                 \\
%%		             & Université de Lille-Campus scientifique   \\
%%		             & Bâtiment ESPRIT-Avenue Henri Poincaré     \\
%%		             & FR-59655 Villeneuve d'Ascq                  \\
%%		\faGlobe     & \url{https://arne.vandenkerchove.com}       \\
%%		\aiOrcid     & \url{https://orcid.org/0000-0002-9367-2986} \\
%%		\faLinkedin  & \url{https://linkedin.com/in/arnevdk}
%%	\end{tabular}
%\end{minipage}%
%\begin{minipage}[b]{.29\linewidth}
%	\includegraphics[width=\linewidth]{photo.jpg}
%\end{minipage}

% Curriculum vitae =============================================================
\chapter*{Curriculum vit\ae}
Arne Van Den Kerchove \\
\texttt{arne@vandenkerchove.com}

\subsection*{About me}
I am a researcher at KU Leuven Laboratory for Neuro- and Psychophysiology
and the University of Lille Research Center for Informatics, Signal Processing,
and Control Science. My research is centered on developing brain-computer
interface (BCI) technologies with a primary focus on assisting individuals with
severe disabilities, especially those with motor and eye movement impairments.
My goal is to enhance their quality of life by creating accurate and
user-friendly BCIs. Combining my expertise in machine learning, signal
processing, and user interface design with insights from neurophysiology, my
PhD centers around the development of an event-related potential BCI system
using EEG for real-time communication. I strongly believe in the potential of
innovative neurotechnology to empower individuals and patients, enabling them
to lead more independent and fulfilling lives.

\subsection*{Keywords}
brain-computer interfaces, event-related
potentials, (multi)linear classification, covert visual attention

\subsection*{Education}

\begin{cvlist}
	\cvitem{PhD\ in Biomedical Sciences}{KU Leuven, 2024}
	\cvitem{PhD\ in Control Science and Signal Processing}{University of
		Lille, 2024}
	\cvitem{M.Sc.\ in Engineering Science: Computer Science}{KU Leuven, 2024}
	\cvitem{B.Sc.\ in Informatics}{KU Leuven, 2017}
\end{cvlist}

\subsection*{Publications}
\begin{refsection}
\setmaxbibnames{9999}
\nocite{%
	VanDenKerchove2020,
	Libert2022,
	VanDenKerchove2022,
	VanDenKerchove2024,
	VanDenKerchove2024a,
}
\printbibliography[heading=none, resetnumbers=true]
\end{refsection}

\subsection*{Awards \& funding}
\begin{cvlist}
	\cvitem{Clinical Trial Sponsorship Program}{Zeto, 2024 \\
		Our research project titled ``A gaze-independent Visual Brain-Computer
		Interface for use by patients with limited or no eye control'' won a spot in
		the Zeto Clinical Trial Sponsorship Program, granting us the resources to conduct research
		with the Zeto EEG headset.
	}
	\cvitem{Student BCI competition 4th place worldwide}{NeuroTechX, 2022 \\
		Our NeuroTechLeuven BrainBrowsR project, a plug-and-play BCI system that lets you control
		social media applications  trough SSVEP-BCI was awarded 90 out of 100 points by
		the jury of the NeuroTechX international student club competition.
	}
	\cvitem{Global PhD Partnership Grant}{KU Leuven and University of Lille, 2021--2024}
\end{cvlist}

\subsection*{Conferences and presentations}
\begin{cvlist}
	\cvitem{CORTICO Days}{CORTICO, Nancy, 2024 \\
		Poster presentation on covert visual attention BCIs for patients with
		oculomotor impairment.}
	\cvitem{10th International BCI Meeting}{BCI Society, Brussels, 2023\\
		Poster presentation on classifier-based latency estimation for
		gaze-independent BCIs.}
	\cvitem{Leuven.AI Scientific Workshop 2022}{Leuven.AI, Leuven, 2022\\
		Poster presentation on (multi-)Kronecker-structured linear discriminant
		analysis for low sample size event-related potential classification.}
	\cvitem{CORTICO Days}{CORTICO, Grenoble, 2022\\
		Presentation on Kronecker-structured LCMV-beamforming for event-related potential
		classification.}
	\cvitem{CORTICO Days}{CORTICO, 2021\\
		Presentation on a multi-component approach
		to spatiotemporal beamforming decoding of event-related potentials.}
	\cvitem{BCI \& Neurotechnology Spring School 2024}{g.tec, 2024}
	\cvitem{Closed Loop Neurotechnologies Autumn School}{NeurotechEU, Lille, 2023}
	\cvitem{CORTICO Days}{CORTICO, Paris, 2023}
	\cvitem{BCI \& Neurotechnology Spring School 2023}{g.tec, 2023}
	\cvitem{BCI \& Neurotechnology Masterclass Belgium}{g.tec, 2022}
	\cvitem{BCI \& Neurotechnology Spring School 2022}{g.tec, 2022}
\end{cvlist}

\subsection*{Teaching Experience}
\subsubsection*{Classes}
\begin{cvlist}
	\cvitem{Teaching Assistant Brain-Computer Interfaces}{KU Leuven, 2022--2024\\
		Teaching exercise sessions in BCI design and signal processing to students in
		the Master of Bioengineering and Advanced Master of Artificial
		Intelligence.}
	\cvitem{Teaching Assistant Fundamentals of Computer Science}{KU Leuven,
		2022 \\
		Teaching exercise sessions in Python programming and algorithmic reasoning to
		students in the Bachelor of	Engineering Sciences.}
\end{cvlist}

\subsubsection{Master theses and internships supervised}
\begin{cvlist}
	\cvitem{Eye-tracker and ERP data fusion for gaze-independent visual
		BCI}{\ \\ Juliette  Meunier, University of Lille, 2024}
	\cvitem{LDA in	the combined
		space-time-frequency domain for BCI decoding}{\ \\Lunkyadi Sucipto, KU Leuven,
		2023--2023}
	\cvitem{Tackling the Midas Touch problem in eye tracking with a
		BCI}{\ \\Reniflal Ebenezer Sundaralal, KU Leuven, 2023--2023}
	\cvitem{An EEG connectivity analysis of Alzheimer’s Disease
		and Frontotemporal Dementia}{Zoe Barinaga, KU Leuven, 2023--2024}
	\cvitem{Single-trial ERP latencies as a predictor for the mode
		of visual attention}{\ \\Yildiz Dilara Parry, KU Leuven, 2023}
	\cvitem{A hybrid P300-gaze BCI alternative for navigating virtual
		spaces}{\ \\Gijs Claes, KU Leuven, 2021--2022}
\end{cvlist}

\subsubsection*{Extracurricular}
\begin{cvlist}
	\cvitem{Project supervisor}{NeuroTech Leuven, 2022--ongoing\\
		Project supervisor and technical advisor for extracurricular
		neurotechnology student projects and a student team competing in the
		annual NeuroTechX BCI competition.}
	\cvitem{PAL tutor Principles of Computer Programming}{KU Leuven, 2015--2016\\
		Organizing and teaching peer-assisted learning sessions in Python programming for first year students.}
\end{cvlist}

\subsection*{Professional experience}
\begin{cvlist}
	\cvitem{Ambulance EMT}{Red Cross, FAST	vzw, 2022-ongoing}
	\cvitem{Python developer}{Mindspeller, 2019
		Python Flask developer in a KU Leuven spin-off that provides neuromarketing services based on
		scientifically validated neuroscience and AI research.}
	\cvitem{Freelance full-stack web developer}{self-employed,
		2014--2020}
\end{cvlist}

\subsection*{Licenses and certifications}
\begin{cvlist}
	\cvitem{ICH Good Clinical Practice}{E6 TransCelerate BioPharma Inc.}
	\cvitem{Ambulance EMT}{FOD Volksgezondheid}
\end{cvlist}

\subsection*{Languages}
\begin{cvlist}
	\cvitem{Dutch}{native}
	\cvitem{English}{fully proficient}
	\cvitem{French}{advanced}
	\cvitem{German}{intermediate}
\end{cvlist}
