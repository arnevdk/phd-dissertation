\chapter*{R\'esum\'e}
Les individus ayant des handicaps moteurs sévères, tels que ceux atteints de \ac{lis} ou de dysfonctionnement oculomoteur, rencontrent des défis considérables lors de l'utilisation d'interfaces cerveau-ordinateur \acp{bci} traditionnelles.
Ces systèmes nécessitent que les utilisateurs dirigent leur regard vers des cibles spécifiques, une tâche qui devient impraticable pour les personnes ayant un contrôle oculaire limité ou inexistant.
Ce travail aborde cette limitation en développant des méthodes de \ac{bci} indépendantes du regard, en se concentrant sur l'amélioration de la \ac{vsa} cachée, où les utilisateurs peuvent diriger leur attention vers une cible sans mouvements oculaires correspondants.
Une contribution clé de ce travail est la compensation pour le jitter de latence ERP, une variabilité qui impacte négativement la performance de décodage dans la \ac{vsa} cachée.
En gérant ce jitter, la méthode proposée améliore la précision de la communication sans nécessiter de contrôle du regard, rendant les \acp{bci} plus utilisables pour les personnes ayant des handicaps moteurs.

Au-delà de l'indépendance du regard, le travail fait également progresser les techniques générales de décodage \ac{erp} en affinant la structure des estimateurs linéaires et multilinaires.
Ces méthodes améliorent la précision de classification dans une gamme de conditions \ac{bci}, en particulier lorsque les données d'entraînement sont limitées.
L'introduction de régularisation structurée dans les modèles linéaires et multilinaires améliore l'interprétabilité des classificateurs et réduit le temps d'entraînement ainsi que la complexité computationnelle.
Cela permet un entraînement plus efficace et une meilleure performance sur des données non vues, contribuant à des systèmes plus fiables qui s'adaptent à divers contextes et besoins des utilisateurs.

Les méthodes proposées ont été validées lors d'expériences impliquant à la fois des individus en bonne santé et 7 individus avec des handicaps physiques sévères et un contrôle oculomoteur altéré.
Les données ont été enregistrées à partir d'un groupe de contrôle, ainsi que de sept individus ayant des impairments du regard.
Ces expériences ont démontré la robustesse des nouvelles méthodes de décodage, montrant que le système pouvait efficacement décoder l'attention cachée même lorsque le regard direct était impossible.

\paragraph{Mots-clés :} interface cerveau-ordinateur, indépendance du regard, attention visuo-spatiale cachée, potentiels évoqués liés aux événements, décodage linéaire, décodage multilinéraire, handicap moteur, handicap du regard.
