\chapter*{Abstract}
\addcontentsline{toc}{chapter}{Abstract}

Individuals with severe motor impairments, such as those with \ac{lis}
or eye-motor dysfunction, face substantial challenges when using traditional
gaze-dependent \acp{bci}.
These systems require users to direct their gaze toward specific targets, a
task that becomes unfeasible for individuals with limited or no eye control.
This work addresses this limitation by developing gaze-independent
\ac{bci} methods, focusing on improving covert \ac{vsa}, where users can direct
their attention to a target without corresponding eye movements.
A key contribution of this work is the compensation for ERP latency jitter, a
variability that negatively impacts decoding performance in covert \ac{vsa}.
By handling this jitter, the proposed method enhances communication accuracy
without requiring gaze control, making \acp{bci} more usable for individuals with
motor impairments.

Beyond gaze independence, the work also advances general \ac{erp} decoding
techniques by refining the structure of linear and multilinear estimators.
These methods improve classification accuracy across a range of \ac{bci}
conditions, particularly when training data are limited.
The introduction of structured regularization in both linear and multilinear
models enhances the interpretability of the classifiers and reduces training
time and computational complexity.
This allows for more efficient training and contributes to generalizability
allowing for more reliable systems that are adaptable to various contexts and
user needs.

The proposed methods were validated in experiments involving both healthy
individuals and seven individuals with severe physical impairment and impaired eye-motor control.
These experiments demonstrated the robustness of the novel decoding methods,
showing that the system could effectively decode covert attention even when
direct gaze was impossible.

\paragraph{Keywords:} brain-computer interface, electroencephalography,
event-related potentials, visuospatial attention, (multi)linear
decoding, physical impairment
