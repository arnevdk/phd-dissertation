% Math declarations
\newtheorem{property}{Property}

\DeclareMathAlphabet\mathbfcal{OMS}{cmsy}{b}{n}
\newcommand{\ten}[1]{\underline{\mathbf{#1}}} % tensor
\newcommand{\mat}[1]{\mathbf{#1}} % matrix
\renewcommand{\vec}[1]{\mathbf{#1}} % vector
\newcommand{\mpr}[2]{\times_{#2} {#1}_{#2}} % tensor mode product
\newcommand{\mmpr}[1]{\times\left\{#1\right\}} % tensor multi-mode product
\newcommand{\mmpri}[1]{\times\{#1\}} % inline tensor multi-mode product
\newcommand{\mmprs}[2]{\times_{-#2}\left\{#1\right\}} % tensor multi-mode product skip
\newcommand{\mmprsi}[2]{\times_{-#2}\{#1\}} % inline tensor multi-mode product skip
\newcommand{\ev}[2]{\text{E}\left[#1\right]_{#2}} % expected value
\newcommand{\R}[1]{\mathbb{R}^{#1}} % real number space
\DeclareMathOperator*{\argmax}{arg\,max} % argmax
\DeclareMathOperator*{\argmin}{arg\,min} % argmin
\DeclareMathOperator{\Tr}{Tr} % trace
\DeclareMathOperator{\vect}{vec} % vectorization (flattening)
\DeclareMathOperator{\nmse}{NMSE} % normalized mean square error
\DeclareMathOperator{\var}{Var} % normalized mean square error
\DeclareMathOperator{\ITR}{ITR} % information transfer rate
