\documentclass{letter}
\usepackage[backend=biber]{biblatex}
\usepackage{todonotes}
\setuptodonotes{inline}

\addbibresource{references.bib}

\newcommand{\reply}[1]{%
	\textbf{Response to review comments made by prof. #1}

}

\begin{document}

\signature{
	Arne Van Den Kerchove \\
	\texttt{arne.vandenkerchove@kuleuven.be} \\
	+32 473 32 78 71 \\
}
\address{
	KU Leuven \\
	Laboratory for Neuro- and Psychophysiology \\
	Campus Gasthuisberg, O\&N 2 \\
	Herestraat 49 bus 1021 \\
	BE-3000 Leuven
}

\begin{letter}{
	prof. John Creemers \\
	KU Leuven \\
	Doctoral School of Biomedical Sciences \\
	Campus Gasthuisberg, O\&N 4 \\
	Herestraat 49 bus 700
	BE-3000 Leuven

}
\opening{Dear chair and members of the reading committee}
\todo{fix todos in document}

I would like to extend my sincere gratitude for time and effort you have taken
to review the manuscript of my PhD dissertation.
Additionally, I would like to thank you for the insightful, constructive and
detailed comments you all have provided.
The following reply contains a point-by-point response to your review comments.
Attached, you will also find an updated draft of the manuscript and a version
with indicated changes for your approval.

\reply{Andrea K\"ubler}
Major comments:
\begin{enumerate}
	\item
\end{enumerate}
Minor comments:
\begin{enumerate}
	\item
\end{enumerate}

\reply{Fabien Lotte}
Major comments:
\begin{enumerate}
	\item Part of the confusion between active and passive BCI might be due
	to the poor choice of phrasing ``paradigms that require little active
  participation from the user, like redirecting attention or initiating imagined actions, are classified as passive BCI.''.
	This has been rewritten to ``in contrast to paradigms that require the user's active participation, for
	instance through redirecting attention or initiating imagined actions,
	paradigms free of this requirement are classified as passive bci,''
	for clarity.
	Neurofeedback has been reclassified as active BCI in the text and in
	figure 1.3.
	\item \todo{Why XDAWN+RG no shrinkage (performed worse), fair
	comparison of sample rate}
  \item Appendix 1.A now fully reports all obtained accuracies to give a better
    insight in effect sizes. Wherever the significant effects have been mentioned
    in the text, the effect size has now been reported along with it.
	\item \todo{report actual accuracies performance stbf in table}
	\item In the results section of chapter 4, statistical comparisons were
	only performed between methods for which we evaluated the performance
	ourselves (HODA, PARAFACDA and BTTDA).
	Since the MOABB benchmarking framework~\cite{Aristimunha2023} allows for
	fair comparison of performance metrics reported across different works,
	the analyses for other comparison methods were not re-run
	by us as reported, but performances were
	taken from the MOABB benchmark database~\cite{Chevallier2024}.
	However, this source provided only the aggregated performance metrics
	and not the metrics per subject, session and cross-validation fold,
	hence we could not perform relevant statistical comparison with these
	comparison models.
	\item \todo{include table with absolute performances for transfer
	conditions}
\end{enumerate}
Minor comments:
\begin{enumerate}
	\item According to the annotations, typographical errors have been
	corrected in text and figures, and imprecise or inaccurate statements have been adjusted.
	The recommended references have been discussed and, where necessary,
	statements were updated accordingly.
  Figure 1.5 was corrected and matched with the correct caption.
\end{enumerate}

\reply{Reinhold Scherer}
Major comments:
\begin{enumerate}
	\item \todo{quantify qualitative statements}
	\item  Mathematical equations and derivations have thoroughly
	been checked for correctness and notational consistency, and have been
	corrected throughout the manuscript where necessary.
	\item As indicated by the reviewer, true on-line experiments with live
	feedback have indeed not yet be conducted, but were foreseen to take
	place within the framework of this doctoral study.
	I agree that these experiments would be valuable to supplement the
	presented conclusions and are the logical next step.
	Indeed, the start of experimentation with healthy control participants
	as well as patient groups, was heavily delayed by the COVID-19 pandemic, as the
	preparatory research started in September 2019 and I officially
	enrolled in the the doctoral training programme in January 2020.
	This restricted not only our access to experiment subjects for a
	significant period at the start of the PhD, and even longer for
	vulnerable patient groups, but also shifted priorities for the medical
	staff at our partner institutions and patient centers.
	Hence, work early in the PhD mainly focused on the development of
	computational methods.
	We are currently starting up true on-line experimentation in cooperation
	with our partner patient centers, with an adapted interface and experimental protocol.
	\todo{indicate in manuscript covid}
	\item Recorded datasets could not be made publicly available since the
	ethical protocol (Ethics Commission of University Hospital Leuven --
	S62547 v1.4) for the conducted studies did not include provisions
	publishing the data.
\end{enumerate}
Minor comments:
\begin{enumerate}
	\item
\end{enumerate}

\reply{Maarten De Vos}
Major comments:
\begin{enumerate}
	\item Mathematical equations and derivations have thoroughly
	been checked for correctness and notational consistency, and have been
	corrected throughout the manuscript where necessary.
	The paragraph on Block-Toeplitz linear discriminant analysis (tLDA) in
	section 1.6.4 has also been corrected, and it was expanded for clarity.
	Figures 3.1 and 3.2 have been regenerated and relabeled for consistent
	formatting.
	\item \todo{replace wilcoxon by delong}
\end{enumerate}
Minor comments:
\begin{enumerate}
	\item Unsupported claims regarding generalisability have been omitted
	from the abstract.
	\item Throughout the manuscript, different EEG band-pass filtering
	settings have been used.
	Chapter 1 indicates that ERPs are usually band-pass filtered between
	0.5 and 16 Hz.
	These were also the preferred cut-off frequencies for other
	(unpublished) analyses used during the course of the thesis work, an
	those used for the analyses in chapter 3.
	However, ERP datasets evaluated in chapter 4 were band-pass filtered
	between 1 and 24 Hz, and in chapter 6 between 0.1 and 20 Hz.
	While there is no general consensus in the field about the optimal
	cut-off frequencies for P300 detection and ERP analysis in general,
	these values all fall well within the accepted
	bounds~\cite{Bougrain2012}.
	Decisions to deviate from this were made consciously.
	In chapter 3, filtering between 1 and 24 Hz was applied for consistency
	with the recommended ERP preprocessing pipeline of the MOABB
	benchmarking framework~\cite{Aristimunha2023}.
	This allows for a fair comparison under equal circumstances of the
	performance of the proposed algorithm with performances of other
	algorithms reported by the MOABB benchmarking
	database~\cite{Chevallier2024}.
	In chapter 6, the low-pass frequency was increased from 16 Hz to 20 Hz,
	since the analyses in this explicitly involve the short-time effects of
	latencies of ERP components, and the analysis of faster, early ERP
	components. A low low-pass frequency would harm the precision of
	latency effects in the filtered signal.
	A lower high-pass filter of 0.1 Hz was used here to reveal the
	influence of the proposed alignment approach on potential
	slower late components.
	\item Typographical errors have been corrected throughout the
	manuscript.
\end{enumerate}

\reply{Adalberto Simeone}
Major comments:
\begin{enumerate}
	\item \todo{discuss in limitations the impact of letters and future
	work experiments with an actual usable interface}
	\todo{mention here plans to go to patients with actual speller
	interface}
\end{enumerate}
Minor comments:
\begin{enumerate}
	\item \todo{correct si measurements Hz, mm, cm, V, µV, $\mu$ V}
	\item \todo{do proper check in textcite, especially in chapter 3}
	\item \todo{check for contractions and remove we've, isn't, wouldn't}
\end{enumerate}


\textbf{References}

\printbibliography[heading=none]

\closing{Kind regards and thank you in advance,}
\end{letter}
\end{document}
