\chapter*{Samenvatting}
\addcontentsline{toc}{chapter}{Dutch abstract}

Personen met ernstige motorische beperkingen, bijvoorbeeld het
insluitings\-syndroom, ervaren aanzienlijke uitdagingen bij het gebruik van
traditionele visuele brein-computer interfaces (BCI's).
Deze systemen vereisen gerichte oogbewegingen wat onpraktisch is voor personen
met beperkte of zonder oogbewegingscontrole.
Dit werk ontwikkelt BCI-technieken onafhankelijk van de blik, met nadruk
op bedekte visuospati\"ele aandacht (VSA) -- aandacht zonder overeenkomstige oogbewegingen.
Een belangrijke bijdrage is het compenseren van variabiliteit in de
timing van event-gerelateerde potentialen (ERP's), een factor met een negatieve
impact op de decodering van bedekte VSA.
Door dit aan te pakken, verbetert de nauwkeurigheid van communicatie die
onafhankelijk is van oogcontrole.
wat BCI's bruikbaarder maakt voor personen met oog-motorische beperkingen.

Naast onafhankelijkheid van oogbewegingen, verbetert dit werk ook algemene
ERP decodering door de structuur van lineaire en multilineaire technieken te verfijnen.
Deze methoden verhogen de nauwkeurigheid in verschillende BCI-toepassingen,
vooral wanneer kalibratiegegevens beperkt zijn.
Ge\-struc\-tu\-reerde regularisatie van lineaire en
multilineaire modellen verhoogt de interpretateerbaarheid
en verlaagt de trainingstijd en computationele complexiteit.
Dit draagt bij aan efficiëntere training en generalisatie, wat
de ontwikkeling van betrouwbaardere systemen toelaat, aangepast aan
verschillende contexten en gebruikersbehoeften.

De voorgestelde methoden zijn gevalideerd in experimenten met gezonde
deelnemers en zeven individuen met een ernstige fysieke beperking en verminderde
controle over oogbewegingen.
De resultaten wijzen op de robuustheid van de nieuwe decodeermethoden,
waarbij werd aangetoond dat het systeem bedekte VSA effectief kan decoderen,
zelfs zonder directe blik.

\bigskip

\paragraph{Trefwoorden:}
brein-computer interface,
electroencephalografie,
event-related potentials,
visuospati\"ele aandacht,
(multi)lineaire decodering,
fysieke beperkingen
