\chapter*{Samenvatting}

Personen met ernstige motorische beperkingen, zoals degenen met het
Locked-in Syndroom (LIS) of oog-motorische disfuncties, ondervinden
aanzienlijke uitdagingen bij het gebruik van traditionele visuele brein-computer interfaces (BCI's).
Deze systemen vereisen dat gebruikers hun blik richten op specifieke doelen, een taak die
onuitvoerbaar wordt voor personen met beperkte of
geen controle over oogbewegingen.
Dit werk richt zich op deze beperking door
BCI-methoden onafhankelijk van oogbewegingen te ontwikkelen, met een focus op
het verbeteren van de bedekte visuospatiële aandacht (VSA), waarbij gebruikers
hun aandacht kunnen richten op een doel zonder overeenkomstige oogbewegingen.
Een belangrijke bijdrage van dit werk is het compenseren van variabiliteit in
de timing van event-gerelateerde potentialen (ERP's), wat de decodeerprestaties
van bedekte VSA negatief beïnvloedt.
Door deze variabiliteit aan te pakken, verbetert de voorgestelde methode de
nauwkeurigheid van de communicatie zonder afhankelijkheid van oogcontrole,
waardoor BCI's bruikbaarder worden voor personen met motorische beperkingen van
de ogen.

Naast onafhankelijkheid van oogbewegingen, bevordert dit werk ook de algemene
ERP-decodeertechnieken door de structuur van lineaire en multilineaire
decodeertechnieken te verfijnen.
Deze methoden verbeteren de classificatienauwkeurigheid in een breed scala aan
BCI-condities, met name wanneer trainingsgegevens beperkt zijn.
De introductie van gestructureerde regularisatie in zowel lineaire als
multilineaire modellen verhoogt de interpretateerbaarheid van de classificators
en vermindert de trainingstijd en computationele complexiteit.
Dit maakt efficiëntere training mogelijk en zorgt voor betere prestaties bij
ongeziene data, wat bijdraagt aan betrouwbaardere systemen die zich kunnen
aanpassen aan verschillende contexten en gebruikersbehoeften.

De voorgestelde methoden zijn gevalideerd in experimenten met zowel gezonde
individuen als zeven individuen met een ernstige fysieke beperking en verminderde
oog-motorische controle.
Deze experimenten toonden de robuustheid van de nieuwe decodeermethoden aan,
waarbij werd aangetoond dat het systeem bedekte aandacht effectief kon decoderen,
zelfs wanneer directe blik onmogelijk was.

\paragraph{Trefwoorden:} brein-computer interface, electroencephalografie,
event-gerelateerde potentialen, bedekte visuospatiale attentie, (multi)lineaire
decodering, zware fysieke beperkingen
