
\chapter{Conclusions \& recommendations}
\section{General discussion}
\section{Limitations}
\section{Future directions}
\section{Working with patients}
\section{Conclusion}

\section{Obtained results}

\subsection{Regularized spatiotemporal beamforming}
We introduced a covariance estimator using adaptive shrinkage (STBF-shrunk) and an estimator
exploiting prior knowledge about the spatiotemporal nature of the EEG signal
(STBF-struct).
We compared these estimators with the original formulation of the
spatiotemporal (STBF-emp)
beamformer and a state-of-the-art Riemannian Geometry method (xDAWN+RG) in an off-line P3 detection task on
an existing dataset.
Our results show that the structured estimator results in higher accuracy in
general, and specifically when training data are sparsely available.
Results can be computed faster and with
substantially less memory usage.
Since these algorithms are not paradigm-specific, the conclusions can be
generalized to other ERP-based BCI settings.
These results have been published in~\cite{VanDenKerchove2022}.

We have tested our proposed methods and similar methods in gaze-independent and
covert VSA settings, but we did not observe a relative increase in performance
in these settings.

%* Preliminary results: BTTDA better than HODA, on par with other state of the
%art methods and in some situations better (data availability, noise structure)

\subsection{Healthy control study}
The CVSA-ERP dataset consists of recordings of 15 healthy participants, mean age
$26.38\pm3.15$ years.
This dataset gives us insight in the ERP dynamics in overt, covert and the
novel split VSA condition, and confirms our hypothesis that P3 jitter has a
significant impact on performance in covert and split VSA, as well as that the
amplitude effect observed for covert VSA also holds for split VSA.

We have evaluated our proposed wCBLE algorithm, the baseline CBLE algorithm and
a state-of-the-art algorithm (tLDA)~\cite{Sosulski2022} that uses similar techniques as our
proposed STBF-struct method on this dataset as well as a publicly available
dataset~\cite{Aloise2012} that also contains the overt and covert VSA conditions.
We evaluated the BCI decoding performance in a single-trial classification experiment,
as well as in a target selection experiment reflecting BCI operation.
Performance was significantly different between decoders, but this
result was significantly dependent on the VSA condition and the dataset.
While CBLE performed overall on par with the state-of-the art method for both datasets, wCBLE was
outperformed by tLDA in overt VSA decoding but outperformed
tLDA for covert VSA.
For the split attention conditions in the CVSA-ERP
dataset, wCBLE yielded a significant improvement over CBLE and the
state-of-the-art method only for d = 1 and d = 3.
These results were corroborated by analyzing selection accuracy and information
transfer rate, which showed similar behavior for both datasets, except in overt
VSA, where accuracy and information transfer rate were not harmed by the lower
single-trial selection performance of wCBLE.

To further study the gaze-independent performance of these algorithms, transfer
learning between VSA conditions is studied to simulate conditions where a patient
can end up in different VSA settings within a BCI operation session due to their
lack of eye motor control.
When trained and evaluated on overt VSA data, our proposed wCBLE algorithm results in a significant decrease in performance
over the state-of-the art tLDA decoder, consistent with the within VSA
condition results.
For all other pairs of training and evaluation VSA condition, however, wCBLE
never yields a significant decrease in performance and often significantly
outperforms the other methods.

These results show that there is an interest in developing a new class of ERP-BCI
interfaces for patients that prefer to rest their eyes on a chosen point on the
screen, so as to avoid the effort of redirecting their eye gaze at different spatial locations on
the stimulation screen. Furthermore, the performance gain in the split VSA conditions
and in the between-VSA condition transfer settings are promising for patients with even
less eye motor control, such as patients experiencing involuntary saccades or
fatigue. An accurate decoder would allow them to comfortably operate a BCI while
resting their eyes on whichever portion of the screen they prefer, even when there is
another target present at that location or this location varies during the course of the
experiment. Our results show that whenever the eye gaze is directed at a different
spatial location other than that of the mentally attended target, the wCBLE decoder can
more accurately discern the locus of mental attention, promoting in this way accurate
gaze-independent decoding. This comes, however, at the cost of a decreased performance
compared to overt VSA operation.
