\chapter{Interface design \& experimental setup}
\section{The Hex-o-Spell interface}
\section{Visuospatial attention conditions}
frenzel
\todo{part about split attention etc from chapter on covert alignment to here}


EEG for the CVSA-ERP dataset was recorded using a SynAmps RT amplifier
(Compumedics Neuroscan, Australia) at 2048Hz and 62 Ag/AgCl active electrodes arranged in the
international 10-10 layout fitted to a standard electrode cap (EASYCAP GmbH,
Germany), with electrodes located at AFz and FCz as ground and reference respectively.
Using electrolyte gel, electrode impedances were brought below 5k$\Omega$.
Electrodes TP9 and TP10, used for off-line re-referencing, were directly
attached to the skin using stickers for better contact.
The power line frequency in Belgium is 50 Hz.
Participant's eye gaze was registered using an EyeLink 1000 Plus eye tracker (SR Research,
Canada) in non-fixation mode.

Participants signed the informed consent form and were seated at a distance of
60 cm before a CRT-emulating monitor (VPixx
Technologies, Canada) operating at a refresh rate of 120Hz, displaying 6
circular white targets with a diameter of 4.15° visual angle and laid out on a hexagon
with a radius of 12.28° of visual angle centered on the midpoint of the screen,
conforming to the interface proposed by \citeasnoun{Treder2010} (\fref{fig:layout_targets}).
A hexagonal layout interface with an empty center and a low number of targets
counteracts target crowding and, as long as the subject’s gaze is within the hexagon of
targets, no other target can be between the subject’s gaze and a covertly
attended target.
Targets are full-contrast white and were intensified by scaling them to a
larger size (5.60° of visual angle, \fref{fig:layout_intense}) instead of changing the contrast to avoid Troxler-fading\footnote{The optical illusion of disappearing unchanging stimuli
experienced when visually fixating~\cite{Troxler1804}.}~\cite{Treder2010} in the
peripheral visual field.
Stimuli were presented using PsychopPy (version 2023.1.3)~\cite{Peirce2019}.


The participant was instructed to press the space bar when ready for a block
of stimulations.
Then, one target was indicated as the cue and the participant
was instructed to count the number of intensifications of the cued target
during the following block of stimulations.
After pressing the space bar again, a blue crosshair appeared, and the subject
was instructed to fixate their gaze on the blue crosshair for the duration of
the stimulation block (\fref{fig:layout_cross_overt} and \fref{fig:layout_cross_covert}).
The position of this crosshair determined the VSA condition for this trial:
overt VSA when the crosshair was at the same location as the cued target,
covert VSA when the crosshair appeared in the center of the screen, and split
VSA when the crosshair appeared on a different target than the cued one.
After pressing the space bar again and a delay of 5 seconds, the stimulation block
starts.
All targets were intensified for a duration of 100 ms, in pseudorandom order.
The inter-stimulus-interval (ISI), the time between the onsets of subsequent intensifications,
was variable and consisted of a fixed 300ms interval (of which 100ms with an intensified target onscreen)
with 200ms uniform jitter added, resulting in an ISI
between 200 and 400 ms.
Inter-stimulus intervals were jittered to counteract steady-state effects and residue in averaging. A
longer inter-stimulus interval will increase component amplitude and aid in counteracting temporal
autocorrelation for a higher statistical test precision.
In a block of stimulations, each target was intensified a pseudorandom number of times between 10 and 15.
This led to stimulation blocks with an average duration of 26.25 seconds. After a block of stimulations, an
input prompt appeared to enter the mentally counted number of intensifications.
After inputting this number, the subject was allowed to pause until pressing the space bar again.
In total, six blocks were presented for overt VSA, six blocks for covert VSA, 12 blocks for
split ($d=1$) VSA, 12 blocks for split ($d=2$) VSA and 6 blocks for split
($d=3$) VSA, covering all possible combinations of VSA conditions, cued targets and
crosshair locations.
The experiment started with five non-recorded practice stimulation blocks, one for
each of the 5 VSA conditions.
During these practice blocks, the participant received feedback about their gaze
position and counting accuracy.
Counting the instructions and the participant's response to the
input prompts, a block lasted about 30
seconds. In sum, the experiment featured approximately 45 minutes of
stimulation time.
After blocks 14 and 28, the participant was allowed to take a longer break.
Including these longer breaks, the experiment lasted approximately one hour.
